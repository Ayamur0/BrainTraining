\newcommand{\graphic}[2]{
\captionsetup{type=figure}
\begin{center}
\includegraphics[width=0.95\textwidth,height=0.75\textheight,keepaspectratio]{images/#1}
\end{center}
\caption{#2}
\bigskip
}

\documentclass[oneside]{Ausarbeitung}
\setcounter{tocdepth}{3}
\setcounter{secnumdepth}{3}
\bibliography{latexlit}


\usepackage{listings}
\lstset{language=csh, literate={-}{-}1}
% ----------------------------------------------------------------------

\begin{document}

%--- Sprachauswahl
% Erlaubte Werte:
%   \selectlanguage{english}
%   \selectlanguage{ngerman}
\selectlanguage{ngerman}

%--- Art der Arbeit
% Erlaubte Werte:
%   \Praxissemesterbericht
%   \Projektbericht
%   \Bachelorarbeit
%   \Seminararbeit
%   \Masterarbeit

\Projektbericht

%--- Studiengang:
% Erlaubte Werte:
%   \Informatik
%   \Elektronik
%   \DataScience
\Informatik

\title{Lernspiel für Grundschulkinder}

\author{Vincent Ugrai, Tim Staudenmaier}
\matrikelnr{76389, 75981}

%--- Ist der Erstbetreuer (\examinerA) an der Hochschule ein Professor?
% Erlaubte Werte:
%   \examinerIsAProfessortrue   % Ja
%   \examinerIsAProfessorfalse  % Nein
\examinerIsAProfessortrue   % Ja

%--- Betreuer
\examinerA{Dr. Mark Hermann}
%\examinerB{Prof.~Dr.~Ulrich~Klauck}

%--- Einreichungsdatum
\date{28. Februar 2022}

%--- Angaben zur Firma
% Auskommentieren, wenn die Arbeit nicht bei einer ext. Firma gemacht wurde.
% \companyname{Beispielfirma}
% \industrialsector{Beispielbranche}
% \department{Beispielabteilung}
% \companystreet{Beispielstr. 1}
% \companycity{12345 Musterstadt}

%--- Angaben zum Betreuer bei dieser Firma
% \advisorname{Name des Betreuers}
% \advisorphone{(01234) 567-890}
% \advisoremail{name@company.xxx}

%--- Titelseite Anzeigen
\maketitle
\cleardoublepage

%---
\pagenumbering{roman}
\setcounter{page}{1}

%--- Firmendaten Anzeigen
% Auskommentieren, wenn die Arbeit nicht bei einer ext. Firma gemacht wurde.
% \makeworkplace
% \cleardoublepage

%--- Eidesstattliche Erklärung anzeigen
\makeaffirmation
\cleardoublepage

%---
\begin{abstract}
  Das Thema dieser Arbeit ist die Erstellung eines Lernspiels für Grundschulkinder.
  Das Lernspiel soll dabei zwei große Teile beeinhalten: Spiele mit denen die Kinder    wichtige Grundlagen in unter anderem Mathematik und Deutsch lernen können und Rätsel.
  Die Lernspiele beinhalten 6 einzelne Spiele:
  \begin{itemize}
  	\item Blitzblick
  	\item Grundrechenarten  (Addition, Subtraktion, Division, Multiplikation)
  	\item Mengen vergleichen mit Objekten und Zahlen
  	\item Vervollständigen die Reihe
  	\item Rechenmauer
  	\item Wortsalat
  \end{itemize}
  %TODO vielleicht als Fließtext
  Zudem sind 23 verschiedene Rätsel vorhanden, die von einfachen Rätseln, bei denen eine Zahl als Ergebnis eingegeben werden muss, bis zu Rätseln bei denen Ziegen auf einem Feld verteilt werden müssen, reichen.\\

  Das Spiel wird in 2D umgesetzt und mithilfe der Unity Engine programmiert.
  In diesem Bericht wird der gesamte Erstellungsprozess eines 2D Spiels in Unity, vom Erstellen des Projekts, über den Aufbau des Codes, bis zum fertigen Spiel, beschrieben.

\end{abstract}
%-----------------------------------------------------------------------
\cleardoublepage
\tableofcontents

%---
\listoffigures

%---
%\listoftables

%---
\listoflistings

%---


\cleardoublepage
\pagenumbering{arabic}
\setcounter{page}{1}

% ----------------------------------------------------------------------
\chapter{Einleitung}
\label{cha:einleitung}

In diesem Bericht werden zuerst die Grundlagen des Programmierens von Spielen in Unity erklärt. Alle Wissen, das notwendig ist um ein einfaches 2D Spiel in Unity zu entwickeln, sind im Kapitel \ref{cha:grundlagen} zu finden.\\
Zuerst wurden die zwei Teile des Spiels (Rätsel und Lernspiele) einzeln entwickelt und am Ende durch ein Menü verbunden.\\
Für die Rätsel wird eine Rätsel-Szene erstellt, welche drei Elemente beeinhaltet:
\begin{itemize}
	\item Die Beschreibung des Rätsels
	\item Bereich in dem das Rätsel zu sehen ist (hier werden zB die Antwortmöglichkeiten gezeigt, aus denen bei manchen Rätseln eine oder mehrere auszuwählen sind %TODO Screenshot)
	\item Bereich für die Eingabe einer Lösung
\end{itemize}
Des weiteren werden für die Rätsel eine Szene für die Rätselauswahl, sowie eine Szene für die Auflösung der eingegebenen Lösung erstellt. Die Implementierung dieser Rätsel findet sich im Kapitel \ref{sec:raetsel}\\

Auch für die Lernspiele wird ein Menü erstellt, über welches alle Spiele ausgewählt werden können. Zudem können hier vor dem Starten für bestimmte Spiele noch Variationen und die Schwierigkeit gewählt werden. \\
Zudem wird für jedes Lernspiel eine Szene benötigt, welche das eigentliche Spiel realisiert. Die Implementierung hierfür wird im Kapitel \ref{sec:lernspiele} beschrieben.\\

\section{Motivation}
\label{sec:motivation}

Durch die Lernspiele sollen die Kinder spielend an das Lernen herangeführt werden. Denn vor allem viele Grunschüler/-innen lernen zu beginn nicht gerne. Viele dieser Kinder lernen mithilfe solcher Spiele schneller, da sie mehr Spaß am lernen haben. \\
Die Lernspiele dienen dabei dazu, den Kindern einfache Grundlagen, wie zB die Grundrechenarten und Rechtschreibung einfacher Worte, spielend beizubringen. Die Rätsel dienen vor allem dazu, das logische Denken der Kinder zu fördern. Darüber hinaus werden in manchen Rätseln auch grundlegende Mathematik Kenntnisse benötigt, wodurch auch dieser weiter gefestigt werden.

\section{Problemstellung und -abgrenzung}
\label{sec:problemstellung}

Das Menü des Spiels muss eine für Kinder ansprechendes und möglichst einfaches UI Design besitzen. Zudem müssen die Kinder durch zB das Sammeln von Punkten motiviert werden die Spiele und Rätsel abzuschließen.\\

In Hinsicht auf den Code des Projekts, muss für die Rätsel ein Grundgerüst erstellt werden, welches die verschiedenen Rätsel laden und darstellen kann, um möglichst viel wiederzuverwenden und zu verhindern, dass jedes Rätsel von neuem erstellt werden muss.\\

Für die Lernspiele müssen die Aufgaben für jedes Spiel möglichst zufällig erstellt werden, damit das Spiel nicht auswendig gelernt werden kann und bei mehrmaligem Spielen noch immer neue Aufgaben vorkommen.

\section{Ziel der Arbeit}
\label{sec:ziel}

Es wird ein 2D-Spiel erstellt, welches über 20 verschiedene Rätsel und 6 verschieden Lernspiele mit unterschiedlichen Variationen und Schwierigkeiten beinhalten. Alle diese Rätsel müssen über ein möglichst simples und für Kinder ansprechendes Menü auswählbar sein.\\
Des weiteren muss ein Punktesystem eingebaut werden, um die Kinder zu motivieren möglichst viele Rätsel zu lösen und möglichst viele Spiele zu spielen. Die Punkte müssen dabei auch zwischen den Spielstarts gespeichert werden und es muss möglich sein diese wieder zu löschen.

\section{Vorgehen}
\label{sec:vorgehen}

Um das Ziel des Projektes zu erreichen, wurde mehrere Meilensteine festgelegt. Um den ersten Meilenstein zu erreichen, muss das erste Level für die Rätsel- und Lernspiele vorhanden sein. Um die Lernspiele zu erstellen, musste erst Recherche betrieben werden, danach wurde die Szene erstellt und die Logik dafür implementiert. Damit das erste Rätsel erstellt wird, wurde eine Szene kreiert in die alle Rätsel geladen werden. Danach wurden die Klasse und XML Datei des Rätsel erstellt, als letztes wurde die Klasse die die Rätsel lädt erstellt. Um den zweiten Meilenstein zu erreichen, sollten die restlichen Level erstellt werden. Für die Lernspiele wurden die restlichen Szenen und Skripte der Lernspiele erstellt. Um die restlichen Rätsel einzubauen, wurden die Klassen und die XML Dateien erzeugt und danach die restlichen Skripte für die Rätsel implementiert. Im dritten Meilenstein wurde das Menü erstellt. Dafür wurde zuerst eine Menüszene für die Rätsel- und Lernspiele erstellt. Danach wurde ein Hauptmenü das in die einzelnen Menüs weiterleitet erstellt. Als letztes wurde ein Savegame angelegt, um die Punktzahl der Rätsel und die Sterne der Lernspiele zu speichern. Um den letzten Meilenstein zu erreichen, muss nur noch die Dokumentation geschrieben werden. Der Ablauf ist in der Abbildung 1.1 zu sehen.

\begin{figure}[htbp]
  \centering
  \includegraphics[height=0.9\textheight]{images/loesungskonzept}
  \caption{Vorgehen beim Erstellen des Spiels}
  \label{fig:1}
\end{figure}


% ---
\chapter{Grundlagen}
\label{cha:grundlagen}

Da das Spiel mit Unity erstellt wird, werden in diesem Kapitel einige wichtige Grundlagen des Unity Editors und der Skripte von Unity erklärt.

\section{Unity Grundlagen}
\label{sec:grundlagenunity}

\subsection{Szenen Editor}
\label{sub:szenenEditor}
Im Editor von Unity können die Spielszenen erstellt werden.
\begin{figure}[htbp]
  \centering
  \includegraphics[width=0.95\textwidth,height=0.75\textheight,keepaspectratio]{images/unityEditor}
  \caption{Szenen Editor von Unity}
  \label{fig:unityEditor}
\end{figure}
Der Editor besteht standardmäßig aus 4 Teilen, wie in der Abbildung zu sehen.
\begin{itemize}
	\item \textbf{Teil 1:} Hier wird die Struktur der Szene dargestellt. In dieser Liste sind alle in der Szene vorhandenen Elemente aufgeführt. Elemente die dabei weiter unten liegen, liegen in der Szene hinter Elemente die darüber liegen. Zudem können Elemente weitere Elemente beinhalten (Child Elemente), die in der Szene dann ebenfalls in dem Element liegen, das sie beinhaltet. Über diese Liste können mittels Rechtsklick auch neue Elemente erstellt werden.
	\item \textbf{Teil 2:} Hier ist zu sehen, wie die Szene momentan aussieht. Durch einen Wechsel vom 'Scene' Tab in den 'Game' Tab, lässt sich auch genau sehen, wie die Szene im Spiel aussehen würde. Hier können die einzelnen Elemente zB verschoben, vergrößert oder gedreht werden.
	\item \textbf{Teil 3:} Hier können die Attribute des aktuell ausgewählten Elements verändert werden. Je nach Element sind hier andere Komponenten enthalten, für ein Image Objekt lässt sich hier zB das dargestellte Bild ändern. Hier können auch weitere Komponenten hinzugefügt werden oder bereits vorhandene deaktiviert oder entfernt werden.
	\item \textbf{Teil 4:} Dieser Teil ist ein einfacher Dateiexplorer, um zB Bilddateien auszuwählen, Skripte zu erstellen bzw. öffnen oder neue Szenen zu erstellen.
\end{itemize}

\subsection{Prefabs}
\label{sub:prefab}
Prefabs sind Objekte, die aus mehreren Szenen-Elementen bestehen.
\begin{figure}[htbp]
  \centering
  \includegraphics[width=0.95\textwidth,height=0.75\textheight,keepaspectratio]{images/unityPrefab}
  \caption{Prefab in Unity}
  \label{fig:unityPrefab}
\end{figure}
In der obigen Abbildung wurde somit ein Zahlen Input Feld erstellt, welches aus einem Image ('NumberInput'), zwei Input Feldern ('tens' und 'ones') und zwei Texten ('Text') besteht. Dieses Prefab kann dann als ein Objekt in Szenen eingefügt werden, wodurch dieser Zahlen Input an mehreren Stellen verwendet werden kann, ohne ihn jedes mal neu zu erstellen.

\subsection{Skripte}
\label{sub:skripte}
Die Skripte von Unity werden in C\# geschrieben. Erstellt man ein neues Skript, so sieht dieses zuerst wie folgt aus:
\begin{lstlisting}[language=csh, caption={Generiertes Unity Skript}]
using System.Collections;
using System.Collections.Generic;
using UnityEngine;

public class TestScript : MonoBehaviour {
    // Start is called before the first frame update
    void Start() {
    }

	// Update is called once per frame
    void Update() {
    }
}
\end{lstlisting}
Jedes Skript, welches Funktionen von Unity verwendet, erbt von der Klasse MonoBehaviour.
Diese erlaubt es zB die beiden Methoden Start und Update zu verwenden, welche beim erstellen eines neuen Skripts auch direkt generiert werden.\\
Die Start Methode wird dabei aufgerufen, bevor der erste Frame geladen wird, in dem dieses Skript vorhanden ist. Ist das Skript also zB die Komponente eines Würfels, welcher nun in die Szene geladen wird, so wird die Start Methode einmal aufgerufen, bevor der Würfel in der Szene auftaucht.\\
Die Update Methode wird hingegen jeden Frame aufgerufen, solange das Skript in der Szene vorhanden ist. \\

Hier können noch viele weitere Funktionen verwendet werden, wie zB onCollisionEnter(), wenn ein Collider vorhanden ist. Weitere Informationen sind in der Unity Dokumentation zu finden \autocite{Unity:Doc}.

\subsubsection{UnityEngine.SceneManagement}
\label{Grundlagen:SceneManagement}
Wenn aus Skripten Szenen geladen werden sollen, muss UnityEngine.SceneManagement eingebunden werden.\\
Dann können mithilfe der LoadScene() Funktion des SceneManagers neue Szenen geladen werden.
\begin{lstlisting}[language=csh, caption={Laden einer Szene im Skript}]
SceneManager.LoadScene("RiddleSolution", LoadSceneMode.Additive)
\end{lstlisting}
Dabei muss der Name der Szene als string übergeben werden. Außerdem kann ein Modus gewählt werden, standardmäßig wird die aktuelle Szene durch die neu geladene ersetzt. Im obigen Beispiel wird der Modus 'Additive' verwendet, wodurch die neue Szene über die alte gelegt wird, so kann man die neu geladene Szene wieder löschen und zur vorherigen zurückkehren.

\subsubsection{Coroutines}
\label{Grundlagen:Coroutines}
Coroutines werden verwendet, um Code zu parallelisieren. Methoden die als Coroutine aufgerufen werden sollen, müssen dabei den Returntyp 'IEnumerator' besitzen. Des weiteren muss die Methode an mindestens einer Stelle eine Wartezeit zurückgeben, zB 'yield return new WaitForSeconds(1);.\\
Wird die Methode dann mittels 'StartCoroutine(Methode()) aufgerufen, wird sie parallel ausgeführt und wartet dabei an den Stellen, an denen eine Wartezeit returned wird, ohne die Ausführung des aufrufenden Codes zu blockieren.\\
Es ist auch möglich in einer Schleife die Wartezeiten zurückzugeben:
\begin{lstlisting}[language=csh, caption={Laden einer Szene im Skript}]
void IEnumerator count(int counter, int amount){
	for(int i = 0; i < amount; i++){
		counter++
		yield return new WaitForSeconds(1);
	}
}
\end{lstlisting}
Diese Methode würde die Variable counter jede Sekunde um 1 hochzählen, bis sie amount mal erhöht wurde.

%---
\chapter{Problemanalyse}
\label{cha:problemanalyse}

\section{Rätsel}
Die Rätsel sollen größtenteils aus XML Dateien geladen werden. Somit muss dann nur eine Szene für alle Rätsel erstellt werden. Die XML Dateien sollen so viele Informationen wie möglich beinhalten, so dass möglichst wenig Code wie möglich für die Rätsel benötigt wird. \\
Für jedes Rätsel existiert zudem eine Klasse, welche die zu ladende XML Datei, das Rätsel Prefab und die benötigten Bilder beinhaltet. Das Rätsel Prefab ist dabei ein Prefab, das in die Szene geladen wird, um Rätsel, die mehr Interaktion erlauben (wie unter anderem das verschieben von Objekten), in derselben Szene darstellen zu können. \\
Außerdem werden allgemeine Klassen benötigt, die in mehreren Rätseln zum Einsatz kommen. Unter anderem ein Skript das dem Spieler erlaubt Objekte mit der Maus zu verschieben und ein Zahleninput sowie Textinput. Zudem wird ein Manager benötigt, der aus den vorhandenen Informationen die Rätsel in die Szene laden kann.\\
Auch müssen werden die Szene für die Rätselauswahl und das Ergebnis des Rätsel benötigt.

\section{Lernspiele}

In diesem Teil des Projektes wurden Spiele für Grundschulkinder entwickelt. Damit das Projekt vernünftig wird, müssen die folgenden Probleme bedacht werden.
\begin{enumerate}
    \item Welche Spiele sind für Grundschulkinder der ersten bis vierten Klasse pädagogisch sinnvoll?
    \item Was ist pädagogisch wertvoll für Kinder?
    \item Was steht in ihrem Lehrplan?
    \item Was ist gut umsetzbar?
    \item Welches Umgebung um das Projekt umzusetzen soll verwendet werden?
    \item Wird das Spiel Multiplayer Elemente beinhalten?
    \item Auf welcher Plattform soll das Spiel erscheinen?
    % \item Fragen warum wir das Projekt so umgesetzt haben
\end{enumerate}

% Probleme dann auf Text ein wenig ausweiten nachdem sie genannt wurden oben

Das erste große Problem war, welche Spiele für Grundschulkinder der ersten bis vierten Klasse pädagogisch sinnvoll sind. Also müssen sich Spiele angesehen und bewertet werden. Dabei ist wichtig darauf zu achten, dass die Spiele im Schwierigkeitsgrad nicht zu schwer und nicht zu einfach sind. Die Spiele sollten zum Lernplan passen und den Kindern auch weiterhelfen. Die Kinder sollten die Spiele gerne spielen wollen und gleichzeitig ihre schulischen Leistungen verbessern.
\\

Das zweite Problem ist, was heißt pädagogisch wertvoll für Kinder? Was bedeutet dies für die Kinder und für die Umsetzung des Projektes? Wie kann gewährleistet werden, das das Projekt pädagogisch wertvoll Umgesetzt wird, sodas die Kinder mithilfe des Projektes ihre Leistungen verbessern können.
\\

Das dritte Problem wäre Spiele aufgrund des aktuell geltenden Lehrplans zu finden. Die Spiele sollten also den Stoff des Lehrplans aufgreifen und für die jeweilige Klassenstufe nicht zu anspruchsvoll sind, sodass die Kinder gefördert werden und sich durch die Spiele verbessern können.
\\

Nach der Lösung dieser drei Probleme folgt, dass vierte Problem, welche Spiele mit den davor genannten Kriterien gut umzusetzen sind? Also sollte überlegt werden, in welcher Art das Spiel dargestellt wird. Wäre es besser es in 2D oder in 3D zu gestalten? Wie müssen die Level aussehen, damit sie die Kinder nicht ablenken, aber sie trotzdem ansprechen?
\\

Im fünften Problem sollte behandelt werden, in welchem Framework das Projekt umgesetzt wird. Also sollten die verschiedenen Möglichkeiten betrachtet werden. Diese wären die Unity3D Engine, die CryEngine und die Unreal Engine. Von diesen müssen die Vor- und Nachteile gegenübergestellt werden und dann das Beste Framework ausgesucht werde.
\\

Als sechstes Problem sollte beachtet werden, ob das Spiel im Singleplayer oder im Multiplayer verfügbar sein soll und welche Option für die Schüler/Schülerin die sinnvollste Umsetzung ist. Dabei sollte beachtet werden, ob es für den Schüler/die Schülerin Vorteile gäbe, wenn er seinen Lernfortschritt mit anderen Schülern/Schülerinnen vergleichen kann oder sogar mit diesen gemeinsam lernen kann. Es sollten aber auch eventuelle Nachteile die Online stattfinden, bedacht werden.
\\
Als letztes stellt sich die Frage für welche Plattform das Projekt entwickelt werden soll. Für die folgenden Plattformen kann die Anwendung entwickelt werden. Auf dem Computer, einem Handy oder auf einer Konsole. Es muss auch in diesem Fall überlegt werden, welche Vor- und Nachteile die jeweilige Plattform bietet.

%---
\chapter{Lösungskonzept}
\label{cha:loesungskonzept}

\section{Rästel}
Ein wichtiger Aspekt für die Rätsel ist, dass sie nicht nur Zahlen oder Text als Lösung haben. Denn wenn alle Rätsel nur Zahlen oder Text als Lösung haben, dann werden diese schnell repetitiv und langweilig für Kinder. Rätsel, die mehr Interaktion bieten, motivieren die Kinder mehr dazu diese zu lösen und sich damit zu beschäftigen.\\
Zudem sind auch aussagekräftige Bilder notwendig, um die Rätsel für Kinder leichter verständlich zu machen. Die Bilder für dieses Spiel wurden alle aus Layton’s Mystery Journey: Katrielle und die Verschwörung der Millionäre für Nintendo Switch entnommen.\\

\graphic{classDiagramRiddle}{Klassenstruktur zum Laden der Rätsel}

Im obigen Klassendiagramm ist die Struktur zu sehen, welches verwendet wird, um die Rätsel zu laden. Die Prefabs für die Rätsel verwenden dabei verschiedene Skripte, um Interaktionen, wie das Verschieben von Objekten zu ermöglichen.\\
Die R001 - R023 Klassen enthalten alle Informationen zum Laden der Rätsel. Dabei besitzt jede dieser Klassen eine Instanz der RiddleInfo Klasse, welche die aus den XML Dateien geladenen Informationen beinhaltet. Zu den aus der XML Datei geladenen Informationen, werden auch manche Infos aus dem Savegame geladen, wie zB ob das Rätsel bereits abgeschlossen wurde.\\
Der RiddleManager verwendet dann die R001 - R023 Klassen um die Rätsel zu laden und die SolutionScene um die Lösung anzuzeigen.

\section{Lernspiele}
Für das erste Problem, welche Spiele für Grundschulkinder der ersten bis vierten Klasse pädagogisch wirkungsvoll sind, sollte erst überlegt werden, was \textit{pädagogisch sinnvoll} für Kinder ist und was im Lehrplan steht.\\
Deshalb sollte das zweite und dritte Problem zuerst gelöst werden. Um dieses zu beantworten, wird zuerst die Begrifflichkeit \textit{pädagogisch sinnvoll} erklärt.  Zunächst wird erklärt, was man unter \textit{pädagogisch} versteht. Der Begriff \textit{pädagogisch} lässt sich in vier Aufgabenbereiche unterteilen,\textit{Betreuung, Erziehung, Bildung und Lernen}. Die ersten drei Begriffe sind nicht essenziell für das Spiel, also wird sich auf den Begriff des Lernens fokussiert. Dieser wird definiert als "\ Im Familien- und schulergänzenden Kontext kommt dem Lernen in seinen spielerischen, sensomotorischen, entdeckenden und selbst-gesteuerten Aspekten Bedeutung zu. Das begriffliche und kognitive Lernen steht zwar nicht im Vordergrund, spielen aber ebenfalls eine Rolle"\footnote{Vgl. Päda.logics! (2019, o. S.)}. %TODO Umschreiben Direktes Zitat nicht geht, Quellenverzeichnis überprüfen
In dem Fall des Projektes sollten die Spiele für die Kinder spielerisch sein und ihr kognitives Lernen fördern. Zeitgleich sollten die Spiele nicht über dem Niveau der ersten bis vierten Klasse stehen.\\

Das dritte Problem, also was im Lehrplan steht, ist leicht zu lösen. Da das Projekt in Baden-Württemberg bearbeitet wurde, wurde der Lehrplan aus diesem Bundesland gewählt.\\
Im Fach Deutsch geht es in den ersten beiden Klassen, also Klasse 1/2 darum, von den Schülern die Handschrift zu entwickeln, sie Texte schreiben zu lassen und sie im Lesen zu fördern. Unter anderem wird den Schülern/Schülerinnen auch beigebracht zu präsentieren und Texte zu verstehen. Im Bereich Sprache sollen Kinder die Unterschiede zum Schreiben kennenlernen. Sie sollen auch lernen, wie Sprache als Mittel der Kommunikation wirkt und grundlegende Strukturen und Mittel in der Sprache kennenlernen. In der 3./4. Klasse werden die vorherigen genannten Bereiche erweitert und verbessert, sodass das Kind in der Lage ist Texte noch besser zu schreiben, verstehen und zu lesen. Das Kind lernt auch noch besser in der deutschen Sprache zurechtzufinden\footnote{Vgl. Ministerium für Kultus, Jugend und Sport Baden Württemberg (2016, S. 13 - 36)}.\\
Im Bereich der Mathematik lernen Kinder der Klassen 1/2, Zahlen und Operatoren, Raum und Formen, Größen und Messen und Daten, Häufigkeiten und Wahrscheinlichkeit kennen.\\
Im Bereich Zahlen und Operatoren lernen Kinder Zahlen und Zahlenbeziehungen kennen. Sie sollen auch verstehen, was Rechenoperatoren sind und wie diese anzuwenden sind. Im Bereich Raum und Form sollen die Kinder die Fähigkeit sich im Raum zu orientieren bekommen, sie sollten Flächen legen und auslegen können. Sie sollen auch einfache Geometrische Figuren und Abbildungen erkennen und benennen.\\
Die Kinder lernen auch Größen zu erkennen und mit diesen in den geeigneten Situationen umzugehen.\\
Im letzten Bereich lernen Kinder Daten aus einfachen Situationen auszuwerten und darzustellen. Sie lernen auch einfache Zufallsexperimente kennen.
\\
In der Klasse 3/4 werden die oben genannten Bereiche erweitert und verbessert, so müssen Kinder zum Beispiel jetzt weitere geometrische Figuren erkennen und nennen können.
\\
Nachdem diese Probleme nun gelöst wurden, können Spiele, die den oben genannten Kriterien entsprechen, ausgesucht werden. Es werden nun Spiele ausgesucht, die die verschiedenen Bereiche des Lehrplans abdecken und gleichzeitig die Kinder fördern und ihr Wissen verbessern\footnote{Vgl. Ministerium für Kultus, Jugend und Sport Baden Württemberg (2016, S. 12 - 36)}.\\
Um Ideen für Lernspiele zu sammeln wurde im Internet recherchiert welche Spiele andere Anwendungen zum Lernen anbieten. Es wurde hierzu auf unterschiedlichen Seiten nach Spielen gesucht. Auf den Online-Portalen für Spiele wie zum Beispiel Spielzwerg, findet man überwiegend Sodokus, Puzzles und ein paar andere Arten von Spielen, die das Gehirn trainieren sollen. Die meisten sind für Kinder der ersten bis vierten Klasse aber nicht sinnvoll, da diese Spiele nicht alle den Inhalten des Lehrplans entsprechen.\\
Also wurden Spielideen auf der Basis des Lehrplans gesammelt, in welchen die Kinder die Grundrechenarten, Mengen, Wahrscheinlichkeiten kennenlernen, sowie deutsche Wörter schreiben und lesen können. Daraus bilden sich folgende Ideen, einmal die Grundrechenarten, das Vergleichen von Mengen mit Gegenständen, sowie das Vergleichen von Zahlen in der symbolischen Art, Rechenmauer lösen, das Spiel Blitzblick, die Länger von Objekten vergleichen, Zahlenreihen erkennen und fortführen können. Damit Kinder in der Sprache Deutsch besser werden können, sollten die Kinder Wörter lesen und schreiben können. Die Kinder sollten außerdem ganze Texte schreiben und verstehen.\\
Aus den oben genannten Spielen wurde sechs ausgewählt, darunter die Grundrechenarten, da diese für Schulkinder unbedingt notwendig sind. Des Weiteren wurden der Blitzblick und der Mengenvergleich ausgewählt. Beim Blitzblick müssen Kinder schnell erfassen, wie viele Objekte sie sehen und lernen dadurch Mengen zu erfassen und die Anzahl dann auch zu nennen, ohne die Objekte nachzuzählen. Beim Mengenvergleich in dem die Schüler/Schülerinnen die Mengen vergleichen sollen, gibt es zwei Varianten, den Vergleich von Objekten in der ikonischen Schreibweise und den Vergleich von Zahlen in der symbolischen Art. Bei der Variante mit Zahlen, lernen die Kinder die Zahlen miteinander zu vergleichen und somit welche Zahl größer oder kleiner ist. Die Objekten animieren die Kinder dazu Mengen zunächst zu erfassen und anschließend miteinander zu vergleichen sollen. Es wurde sich auch für das Spiel entschieden Zahlenreihen zu erkennen und fortzuführen, da das Erkennen von Mustern und Strukturen ebenfalls Teil des Lehrplans ist. %TODO Quelle hinzufügen
Die Schüler/Schülerinnen, sollten in der Lage sein die Zahlen um die die Reihe erweitert wird zu erkennen und dadurch die Reihe fortzuführen. Die Rechenmauer bietet den Kindern die Möglichkeit die Addition noch einmal zu verbessern. Das Prinzip der Rechenmauer ist das folgende, jeweils zwei Steine der Rechenmauer, die in der gleichen Ebene werden addiert und das Ergebnis wird in den Stein darüber geschrieben. Die Subtraktion kann als Erweiterung noch eingebaut werden. Bei der Rechenmauer sollen die Schülerinnen und Schüler wie bei dem Spiel \textit{Grundrechenarten - Addition} addieren, aber in diesem Fall können die Zahlen wesentlich größer werden. Als letztes Spiel wurde sich für ein Spiel entschieden, was Wörter für Grundschulkinder nimmt, diese mischt und dem Kind anzeigen. Die Aufgabe des Kindes ist es zu erkennen welches Wort gemischt wurde und es richtig in ein Textfeld zu schreiben. Hierbei lernt das Kind Wörter, die es gelernt hat, zu erkennen und zu schreiben. Für die Aufgaben, in denen das Kind die Länge von Objekten unterscheiden kann, wurde sich nicht entschieden, da der Bildschirm nur eine gewisse Größe hat und es sehr ähnlich zu dem Mengenvergleichen ist. Die Objekte würden in Unity lang gezogen werden und das Kind müsste sehen welches länger ist. Außerdem ist die enaktive Umsetzung mit realen Objekten bei dem Vergleich von Längen besser geeignet. Im Fach Deutsch wurde sich nur für ein Spiel entschieden, da ein Spiel mit langen Texten schwer umsetzbar wäre. Das Kind müsste entweder lange Texte schreiben oder lesen und darauf hin müsste dieser lange Text überprüft werden. In diesem Fall gibt es auch keine gute Möglichkeit dies umzusetzen, sodass es dem Kind Spaß machen würde.
\\
Nachdem die Spiele des Projektes festgelegt sind, muss geklärt werden, welches Framework verwendet werden soll. Hierzu muss festgelegt sein, was es für ein Projekt sein soll. Soll die Anwendung in 2D oder in 3D sein? Für ein Projekt, das aus Spielen und Rätseln besteht, bietet sich es in 2D mehr an. Das liegt daran, dass die Kinder in diesen Spielen mehr schreiben und Objekte bewegen müssen. In 3D ist die Umgebung wahrscheinlich zusätzlich ablenkend und nicht dem Ziel förderlich. Wenn Lernspiele im Internet betrachtet werden, so sind diese immer in 2D.\\
Um die Wahl des Frameworks einzugrenzen, muss klar sein, ob das Spiel online oder offline spielbar ist und auf welcher Plattform dieses Projekt erscheinen soll.\\
Darum muss sich erst mit den Problemen Nummer sechs und sieben auseinandergesetzt werden.
Im sechsten Problem ist die Frage, ob das Projekt online oder offline, also ob es nur für einen Schüler/Schülerinn sichtbar ist wie er lernt und die Rätsel löst oder ob es für seine/ihre Freunde auch sichtbar sein kann. Der Vorteil von Online Ranglisten ist die Möglichkeit sich zu vergleichen und vielleicht die Motivation zu entwickeln mehr zu lernen, um besser als seine/ihre Freunde zu sein. Andererseits, kann dies auch ein negativer Grund sein, wenn man  nicht so schnell wie seine Freunde lernt oder nicht so gut ist, besteht die Gefahr, die Lust an der Anwendung zu verlieren. Wenn die Möglichkeit besteht, mit seinen Freunden zu lernen, bringt das den Schülern/Schülerinnen den Vorteil, dass sie motivierter sind. Der Nachteil hierbei ist, dass die Kinder die Anwendung nutzen sollten, weil sie es wollen und nicht weil ein Freund diese auch verwendet. Also wird das Projekt zunächst nur für einen Spieler ausgelegt. Die Kinder können es sich herunterladen und mithilfe des Punktesystems bei den Rätseln und den gezählten Sternen bei den Lernspielen untereinander vergleichen. So können sie für sich üben, ihren Fortschritt anzeigen und, wenn sie es wollen, diesen mit Freunden vergleichen.\\
Zu dem Problem auf welcher Plattform das Projekt entwickelt werden soll gibt es vier Möglichkeiten. Es kann im Browser, auf einer Konsole, wie der Xbox, Playstation oder Switch erscheinen, auf dem Computer oder dem Handy entwickelt werden. Der Browser bietet gute Möglichkeiten für eine Kombination mit einem Multiplayer, einer Rangliste oder dem  gemeinsamen Lernen, für die Umsetzung des Projektes bis jetzt ist das aber nicht notwendig.\\
Auf einer Konsole wäre die Umsetzung möglich, aber die Art wie die Lernenden ihr Lösungen bei den Lernspielen eintragen, gestalltet sich als komplizierter, da die meisten Konsolen keine Tastatur um die Lösung einfach einzutragen besitzen. Zum Schluss bleibt noch die Möglichkeit eine Handy-Anwendung oder eine Computer-Anwendung zu entwickeln. Auf dem Handy kann man das Projekt für einen Spieler oder mehrere entwickeln, auch die Art wie die Lösung eingetragen wird ist simpel, da Handys eine Tastatur besitzen. Das Problem dabei ist, dass nicht alle Schüler/Schülerinnen ein Handy oder Tablet besitzen.\\
Damit bleibt für unser Projekt der Computer. In den meisten Haushalten gibt es einen Computer, der in der Lage ist ein Lernspiel mit nicht so hohen Anforderungen in der Grafik und in der Leistung ausführen zu können. Der Computer hat eine Tastatur und wenn später das Projekt einen online Modus bekommen soll, so kann dieser problemlos hinzugefügt werden. Des Weiteren ist die Möglichkeit der Hilfestellung durch die Eltern am Computer durch einen größeren Bildschirm besser möglich.

Nachdem diese Probleme geklärt wurden, sollte überlegt werden, welche Engine diese Voraussetzungen erfüllt. Die Frameworks, die genutzt werden können, um das Projekt zu realisieren, wären Unity3D Engine, die Unreal Engine und die CryEngine. Die CryEngine ist für Virtual Reality Spiele eine sehr gute Engine und für das Projekt einer 2D Entwicklung daher nicht geeignet. Die Unreal Engine ist eine gute Engine mit vielen Möglichkeiten, diese legt ihren Schwerpunkt aber auf Anwendungen, die in 3D sein sollen. Die Unity Engine ist in der Lage Anwendungen in 2D und 3D zu entwickeln, sie wird laut Bewertungen aber bevorzugt für 2D-Anwendungen verwendet und wurde unter anderem deswegen ausgewählt. Ein weiterer Vorteil, der für die Unity Engine spricht ist, dass in den vorherigen Semestern an der Hochschule bereits mit Unity gearbeitet wurde.
%---
\chapter{Implementierung}
\label{cha:implementierung}
\section{Rätsel}
\label{sec:raetsel}
\subsection{Rätsel Szene}
\graphic{riddleScene}{Szenenaufbau für Rätsel}
In der obigen Abbildung ist der generelle Aufbau der Rätsel Szene zu sehen. Oben links gibt es ein Textfeld, in das die Rätselbeschreibung eingetragen wird. Darunter ist Platz für ein Eingabefeld für das Ergebnis, oder ein Bild, falls das Rätsel keine Ergebniseingabe benötigt.\\ 
In der rechten Hälfte findet sich der Hauptteil des Rätsel. Hier werden Rätsel gelöst, die zB verschiebbare Objekte haben. Für Rätsel die lediglich eine Beschreibung und dann eine Lösungseingabe benötigen, wird hier nur ein Bild eingefügt.

\subsection{Laden der Rätsel in die Szene}
Damit nicht für jedes Rätsel eine eigene Szene erstellt werden muss, sollen die Rätsel dynamisch in diese eine Szene geladen werden können.\\
Dafür müssen die Daten für jedes Rätsel in Dateien gespeichert werden, welche dann geladen werden.

\subsubsection{Rätseldaten}
Für jedes Rätsel werden verschiedene Assets benötigt:
\begin{itemize}
	\item \textbf{info.xml} mit den Daten für das Rätsel
	\item \textbf{main.png} das Hintergrundbild für die rechte Hälfte
	\item \textbf{R001.prefab} das Prefab, das in die rechte Hälfte geladen wird und Funktionen beinhaltet die das Rätsel zum lösen bietet
	\item \textbf{solution.png} das Bild, das in der Lösung angezeigt wird
	\item optional: \textbf{result.png} falls keine Lösungseingabe benötigt wird, ein Bild für die untere linke Ecke
	\item Eventuell weitere für das Prefab benötigte Assets
\end{itemize}

Die Daten für die Rätsel werden in XML-Dateien gespeichert. Diese enthalten alle notwendigen Informationen um ein Rätsel laden zu können.
\begin{lstlisting}[language=xml, caption={Aufbau einer Rätsel XML-Datei}]
<?xml version="1.0"?>
<RiddleInfo>
    <id>1</id>
    <title>Titel</title>
    <maxPoints>35</maxPoints>
    <minPoints>5</minPoints>
    <descriptionParagraphs type="array">
        <value>Absatz 1</value>
        <value>Absatz 2 $roter Text$</value>
    </descriptionParagraphs>
    <solutionParagraphs>
        <value>Absatz 1</value>
        <value>Absatz 2</value>
    </solutionParagraphs>
    <autoSubmit>false</autoSubmit>
</RiddleInfo>
\end{lstlisting}
In der XML-Datei ist die ID enthalten, welche ein Rätsel eindeutig identifiziert. Der Titel wird später in der Rätselauswahl angezeigt, die Punkte werden beim abschließen des Rätsels verrechnet und die Beschreibung wird in der Rätsel Szene angezeigt. 
\\Die Lösung wird nach lösen des Rätsel in einem Popup angezeigt.
\\Wird autoSubmit auf true gesetzt, so wird der 'Bestätigen' Button aus der Szene entfernt und das Rätsel wird automatisch bestätigt, sobald die korrekte Lösung eingeben wurde.\\

\subsubsection{RiddleInfo Klasse zum Laden der XML-Dateien}
\label{sub:RiddleInfo}
\begin{lstlisting}[language=csh, caption={In der Klasse gespeicherte Daten}]
	[XmlElement]
    public int id;
    [XmlElement]
    public string title;
    public bool completed;
    public int points;
    [XmlElement]
    public int maxPoints;
    [XmlElement]
    public int minPoints;
    public int pointReduction;
    [XmlArrayItem("value")]
    public string[] descriptionParagraphs;
    public string description = "";
    [XmlArrayItem("value")]
    public string[] solutionParagraphs;
    public string solution = "";
    [XmlElement]
    public bool autoSubmit;
\end{lstlisting}
Alle mit '[XmlElement]' markierten Variablen sind unter demselben Namen in den XML-Dateien zu finden. Diese Variablen können dann direkt mithilfe des XML-Serialisierers von C\# aus der Datei gelesen werden. Die anderen Variablen werden dann von den aus der Datei gelesenen Variablen abgeleitet.\\
\begin{lstlisting}[language=csh, caption={Laden der Daten mit dem XML-Serialisierer}, escapeinside={(*}{*)}]
private static XmlSerializer serializer = new XmlSerializer(typeof(RiddleInfo));

public static RiddleInfo loadFromXML(int id) {
    RiddleInfo info = new RiddleInfo();
    info.id = id;
    XmlDocument xmldoc = new XmlDocument();
   TextAsset textAsset = Resources.Load<TextAsset>($"RiddleAssets/{info.id.ToString("000")}/info");
    xmldoc.LoadXml(textAsset.text);
    using (XmlReader reader = new XmlNodeReader(xmldoc)) {
        info = serializer.Deserialize(reader) as RiddleInfo;
    }
    (*...*)
\end{lstlisting}
Die Variablen 'completed' und 'points' werden anschließend aus der SaveFile gelesen, da diese Variablen beinhalten, ob das Rätsel schon einmal abgeschlossen wurde bzw. wie viele Punkte vom Rätsel noch erhalten werden können.\\
Die Variable 'pointReduction' lässt sich aus 'maxPoints' und 'minPoints' berechnen, da immer fünfmal Punkte abgezogen werden sollen, bevor die minimalen Punkte erreicht sind.\\
Für die string 'solution' und 'description' werden die string in den Arrays aneinandergehängt, wobei nach jedem string aus dem Array jeweils zwei Zeilenumbrüche folgen, um die Paragraphen zu trennen. Für die Beschreibung wird zudem noch eine Regex angewendet, da Text der in der XML-Datei zwischen zwei \$-Zeichen steht, in rot angezeigt werden soll:
\begin{lstlisting}[language=csh, caption={Regex um Text zwischen \$-Zeichen rot zu färben}]
temp = Regex.Replace(temp, @"\$([^$]+)\$", "<color=#b83e39>$1</color>");
\end{lstlisting}
Der Text der zwischen den \$-Zeichen stand steht im string dann zwischen zwei HTML color Tags, welche das Unity Text Objekt interpretieren kann.\\

Zusätzlich zur obigen Ladefunktion gibt es noch eine gekürzte Variante dieser, welche weniger Daten lädt. Diese ist zB für die Rätselliste nützlich, da in dieser für alle Rätsel ID, Titel, verbleibende Punkte, maximale Punkte und der Abgeschlossen-Status benötigt werden. In dieser gekürzten Ladefunktion werden nur diese Daten geladen, damit nicht unnötig alle anderen Daten, die von der Rätselliste nicht benötigt werden geladen werden müssen, was je nach Gesamtanzahl der Rätsel für eine schnellere Ladezeit sorgt.

\subsubsection{Rätselklassen}
Neben den Daten existiert für jedes Rätsel eine Klasse, welche unter anderem die vorhandenen Assets lädt, so dass diese dann verwendet werden können.\\
Die Klassen für die Rätsel erben dabei alle von der abstrakten Klasse Riddle.cs:
\begin{lstlisting}[language=csh, caption={Riddle.cs base Klasse}]
	public RiddleInfo info;
    public Sprite resultAreaImage;
    public Sprite solutionSprite;
    public GameObject interactiveArea;

    private static XmlSerializer serializer = new XmlSerializer(typeof(RiddleInfo));

    virtual public void OnEnable() { }

    public void init(int id) {
        info = RiddleInfo.loadFromXML(id);
        solutionSprite = Resources.Load<Sprite>($"RiddleAssets/{idToString(id)}/solution") as Sprite;
        interactiveArea = Instantiate(Resources.Load<GameObject>( $"RiddleAssets/{idToString(id)}/R{idToString(id)}")) as GameObject;
        resultAreaImage = Resources.Load<Sprite>($"RiddleAssets/{idToString(id)}/result") as Sprite;
    }

    virtual public bool checkResult() { return false; }

    virtual public bool isResultValid() { return false; }

    public void reducePoints() {
        info.points = Mathf.Clamp(info.minPoints, info.points - info.pointReduction, info.maxPoints);
    }

    public void Destroy() {
        Destroy(interactiveArea);
    }

    private string idToString(int id) {
        return id.ToString("000");
    }
\end{lstlisting}
Diese Klasse muss von jeder Rätselklasse implementiert werden. In der OnEnable() Funktion muss immer zuerst 'base.init(id)' aufgerufen werden, danach können weitere für das Rätsel benötigte Assets geladen werden.\\
In der init() Funktion werden die Assets geladen, die jedes Rätsel benötigt: Die Infos aus der XML-Datei, das Bild für die Lösung, das Prefab für die rechte Seite und das Bild für die untere linke Ecke (falls nicht vorhanden steht hier null).\\
Zudem muss jedes Rätsel die Funktionen checkResult() und isResultValid() implementieren. Erstere soll dabei true zurückgeben, wenn die richtige Lösung eingegeben wurde und isResultValid() muss true zurückgeben, wenn die eingegebene Lösung syntaktisch Korrekt ist und bestätigt werden könnte (nicht korrekt wäre zB bei einer Texteingabe mit 6 Zeichen nur 4 Zeichen einzugeben).\\

Für ein einfaches Rätsel, mit nur einer Zahleneingabe als Lösung, sieht die Klasse dann wie folgt aus:
\begin{lstlisting}[language=csh, caption={Klasse für Rätsel 002 die von Riddle.cs erbt}]
public class R002 : Riddle {
    public override void OnEnable() {
        base.init(2);
    }

    public override bool checkResult() {
        return interactiveArea.GetComponent<NumberInput>().GetInput() == 14;
    }

    public override bool isResultValid() {
        return interactiveArea.GetComponent<NumberInput>().IsInputValid();
    }
}
\end{lstlisting}

\subsubsection{RiddleLoader Klasse zum Laden der Rätsel in die Szene}
Um die Klassen der Rätsel in die Rätselszene zu laden wird der RiddleManager verwendet.
Dieser hält in Variablen folgende Elemente aus der Szene:
\begin{itemize}
	\item \textbf{mainArea:} Das Canvas für die rechte Hälfte des Rätsels, in welches das Prefab geladen wird
	\item \textbf{resultImage:} Das Bild in der unteren linken Ecke, in welches entweder in Hintergrund für die Ergebniseingabe kommt oder, falls diese nicht benötigt wird, ein Bild
	\item \textbf{description:} Text Element für die Beschreibung
	\item \textbf{solution:} Canvas des Popups in dem die Lösung angezeigt wird
	\item \textbf{solutionText:} Text im Canvas für die Lösung, in welchen der Lösungstext geschrieben wird
	\item \textbf{solutionImage:} Image Element im Canvas für die Lösung, in welchem das Bild für die Lösung angezeigt wird
	\item \textbf{submitButton:} Button, mit welchem die Eingabe bestätigt wird und überprüft wird, ob das Rätsel richtig gelöst wurde
\end{itemize}
Zudem wird aus dem resultImage eine Instanz einer privaten Klasse angelegt, welche das wechseln zwischen Ergebniseingabe-Hintergrund und Rätselbild vereinfacht.\\

Um Rätsel zu laden, bietet der RiddleManager die LoadRiddle Funktion, welcher die ID des zu ladenden Rätsels übergeben wird:

\begin{lstlisting}[language=csh, caption={LoadRiddle() Methode des RätselManagers}]
void LoadRiddle(int id) {
        solution.enabled = false;
        if (currentRiddle != null)
            currentRiddle.Destroy();
        currentRiddle = ScriptableObject.CreateInstance("R" + id.ToString("000")) as Riddle;
        description.text = currentRiddle.info.description;
        if (currentRiddle.info.autoSubmit)
            submitButton.gameObject.SetActive(false);
        else
            submitButton.gameObject.SetActive(true);
        if (currentRiddle.resultAreaImage != null)
            resultArea.SetImage(currentRiddle.resultAreaImage);
        else
            resultArea.SetStandardImage();
        RectTransform rectTransform = currentRiddle.interactiveArea.GetComponent<RectTransform>();
        rectTransform.sizeDelta = mainArea.GetComponent<RectTransform>().rect.size;
        rectTransform.position = Vector3.zero;
        currentRiddle.interactiveArea.transform.SetParent(mainArea.transform, false);
    }
\end{lstlisting}
Zuerst wird hier das Lösungspopup deaktiviert, falls dieses noch offen ist. Ist noch ein anderes Rätsel geladen, wird dieses ebenfalls gelöscht.\\
Danach wird die statische Variable 'currentRiddle' auf die Klasse des Rätsels gesetzt, das geladen werden soll, damit andere Klassen Zugriff auf das aktuelle Rätsel haben.\\
Dann werden die Elemente der Szene gesetzt: Der Beschreibungstext wird aktualisiert, falls das Rätsel automatisch bestätigt wird der Bestätigen-Button deaktiviert und falls ein Bild für die untere linke Ecke vorhanden ist, wird dieses gesetzt, ansonsten wird das Standard Bild gesetzt.\\
Dann wird noch das Prefab in das Canvas der rechten Hälfte geladen. Dabei wird das Prefab auf die Größe des Canvas skaliert, mittels des sizeDeltas und dann in diesem zentriert, indem die Position auf die Mitte des Parents (0,0) gesetzt wird.\\
Danach ist das Rätsel vollständig geladen und der Spieler kann es lösen.\\

In der Update() Methode des RiddleManagers, welche jeden Frame aufgerufen wird, wird überprüft ob das momentane Ergebnis syntaktisch korrekt ist. Dazu wird die isResultValid() Funktion des aktuellen Rätsels verwendet. Solange die Lösung nicht syntaktisch korrekt ist, wird der Bestätigen-Button ausgegraut und ist nicht klickbar.\\

Des weiteren gibt es eine SubmitSolution() Funktion, welche nach klicken des Bestätigen-Button überprüft ob die aktuell eingegebene Lösung korrekt ist. Ist die Lösung korrekt wird die Erfolgsszene geladen, ansonsten die Szene für falsche Lösungen. Die Szenen werden dabei im Additiven Modus geladen (siehe \ref{Grundlagen:SceneManagement}) damit danach zum Rätsel zurückgekehrt werden kann. Zudem wird, bei einem richtigen Ergebnis, mit der ShowSolution() Funktion noch das Popup mit der Lösung angezeigt. Die ShowSolution() Funktion setzt dabei den Text und das Bild für die Lösung auf die Daten aus der Rätselklasse. Das Popup mit der Lösung ist dann zu sehen, sobald die geladene Erfolgsszene wieder verschwunden ist.
\graphic{solutionPopup}{Popup das die Lösung anzeigt}

Die Methode LoadMenu() lädt bei drücken des Menü Buttons die Szene für die Rätselauswahl.

\subsection{Ergebnisszene}
Wird diese Szene geladen, so wird zuerst der gesamte Bildschirm für ca 2 Sekunden dunkelgrau. Währenddessen wird einer von zwei Sounds abgespielt, je nachdem ob die Lösung korrekt war oder nicht.\\
Danach wird die graue Fläche aus der Szene entfernt. Bei Erfolg ist dann die folgende Szene zu sehen:
\graphic{riddleSuccess}{Szene bei korrekt gelöstem Rätsel}
\begin{lstlisting}[language=csh, caption={Methode um Unity Texte hoch oder runter zu zählen}]
 private IEnumerator count(int amount, bool up, Text toUpdate) {
        float interval = 2f / amount;
        for (int i = 0; i < amount; i++) {
            if (up)
                toUpdate.text = (Int32.Parse(toUpdate.text) + 1).ToString();
            else
                toUpdate.text = (Int32.Parse(toUpdate.text) - 1).ToString();
            yield return new WaitForSeconds(interval);
        }
    }
\end{lstlisting}
Der obige Code wird verwendet um die Punktzahlen hoch oder runter zu zählen. Die Methode wird als Coroutine aufgerufen, da die Variablen langsam hoch bzw runter gezählt werden (für Coroutines siehe {\ref{Grundlagen:Coroutines}). Die Methode benötigt immer 2 Sekunden um die Punktzahlen hoch oder runter zu zählen, unabhängig davon, wie weit hoch oder runter gezählt wird. Lediglich die Geschwindigkeit in der die Zahlen sich ändern wird schneller bzw langsamer.\\

Im Falle der richtigen Lösung (siehe obige Szene) werden dann die verdienten Punkte auf den Punktestand gezählt, während die verbleibenden Punkte herunter gezählt werden. Dafür wird die oben beschriebene count() Methode verwendet. Danach wird die Szene zerstört, wodurch die dahinter liegende Rätsel Szene wieder zu sehen ist, welche das Popup mit der Lösung beinhaltet.\\

Wurde das Rätsel falsch gelöst, so ist der folgende Screen zu sehen:
\graphic{riddleFailure}{Szene bei falsch gelöstem Rätsel}
Bei falscher Lösung wird in dieser Szene mit obiger count() Methode die verbleibende Punktzahl verringert. Wie weit die Punktzahl verringert wird, hängt dabei vom Rätsel ab. Ist die Punktzahl bereits auf dem minimalen Punktestand für dieses Rätsel, so wird sie nicht weiter verringert.
\begin{lstlisting}[language=csh, caption={Methode für die Szene bei falschen Lösungen}]
private IEnumerator ExecuteFailure() {
        yield return new WaitForSeconds(2);
        BackgroundFailure.enabled = true;
        yield return new WaitForSeconds(1);
        bool remainingAboveMin = Int32.Parse(RemainingPoints.text.Split('/')[0]) > RiddleManager.currentRiddle.info.minPoints;
        if (remainingAboveMin)
            StartCoroutine(count(RiddleManager.currentRiddle.info.pointReduction, false, RemainingPoints));
        yield return new WaitForSeconds(5);
        if (remainingAboveMin) {
            int remaining = Int32.Parse(RemainingPoints.text.Split('/')[0]);
            SaveDataManager.RiddleSaveData.UpdateRemainingPoints(RiddleManager.riddleId, remaining);
            SaveDataManager.SaveGame();
        }
        SceneManager.SetActiveScene(SceneManager.GetSceneByName("Riddle"));
        SceneManager.UnloadSceneAsync("RiddleSolution");
    }
\end{lstlisting}
Diese Methode realisiert die Funktion im Falle einer falschen Lösung. Wie oben beschrieben, wird zuerst für 2 Sekunden der graue Bildschirm angezeigt, bevor dann das Canvas mit obiger Szene für die falsche Lösung aktiviert wird. Nach einer weiteren Sekunde wird dann die count() Methode verwendet um die Punktzahl, falls noch möglich, zu verringern. Nach weiteren 5 Sekunden wird die neue Punktzahl noch im Savegame gespeichert und dann die Szene zerstört, wodurch die dahinter liegende Rätsel Szene wieder zu sehen ist.

\subsection{Rätselauswahl}
In dieser Szene findet sich eine Liste, die alle Rätsel beinhaltet. Hierüber kann jedes Rätsel ausgewählt und gespielt werden.
\graphic{riddleList}{Rätselauswahl}
In der Liste ist jeweils die Nummer des Rätsels zu sehen, der Name, die Punkte die noch verdient werden können, wenn das Rätsel gelöst wird und die maximale Punktzahl die möglich gewesen wäre. Wurde das Rätsel bereits gelöst, wird zudem ein Haken auf der rechten Seite angezeigt.\\
Um die scrollbare Liste zu erzeugen, wird die in Unity vorhandene Scrollview verwendet.
Fügt man diese in einer Szene hinzu, erhält man folgende Struktur:
\graphic{scrollView}{Aufbau der in Unity vorhandenen Scrollview}
Für die hier verwendete Liste wurde das 'Scrollbar Horizontal' Element entfernt, da die Liste nur vertikal gescrollt werden soll. Die Elemente, die in der Liste auftauchen sollen, müssen dann nur als Child Elemente des 'Content' Elements erstellt werden. Dafür wurde ein Prefab für ein Listen Element erstellt, welches die Texte für ID, Titel und Punkte sowie ein Bild für den Haken und den Hintergrund enthält. Dieses wird dann in der Start() Methode der Szene für jedes Rätsel initialisiert und als Child vom 'Content' Element eingefügt:
\begin{lstlisting}[language=csh, caption={Start Methode der Rätsel Liste, welche die Listenelemente erstellt}]
void Start() {
        for (int i = 1; i <= RiddleInfo.RiddleAmount; i++) {
            RiddleInfo.SimpleRiddleInfo info = RiddleInfo.loadSimpleInfoFromXML(i);
            GameObject temp = Instantiate(listElement);
            temp.GetComponent<Button>().onClick.AddListener(() => loadRiddleScene(info.id));
            temp.transform.GetChild(0).GetComponent<Text>().text = info.id.ToString("000");
            temp.transform.GetChild(1).GetComponent<Text>().text = info.title;
            temp.transform.GetChild(2).GetComponent<Text>().text = info.points + "/" + info.maxPoints;
            temp.transform.GetChild(3).GetComponent<Image>().enabled = info.completed;
            temp.transform.SetParent(contentContainer);
            temp.transform.localScale = Vector2.one;
        }
    }
\end{lstlisting}
Hier wird für jedes Rätsel zuerst die gekürzte Info aus der XML-Datei gelesen (siehe \ref{sub:RiddleInfo}). Die gekürzte Info enthält alle für die Liste notwendigen Daten.
Dann wird eine Instanz des Listenelement-Prefabs erstellt, und diesem eine onClick() Funktion zugewiesen, welche beim Klick auf das Element dann die loadRiddleScene(id) Methode aufruft. Diese setzt im RiddleManager die ID für das zu ladende Rätsel und lädt dann die Rätselszene, in welcher der RiddleManager dann das Rätsel lädt.\\
Dann werden noch die Daten für das Rätsel in die Texte geschrieben und das Bild des Hakens aktiviert bzw deaktiviert, je nachdem ob das Rätsel schon gelöst wurde. Am Ende wird dann noch der Parent auf das 'Content' Element gesetzt und die Größe des Objekts auf 1 gesetzt, um den kompletten Parent und damit die komplette Breite der Liste zu füllen.\\

In der oberen linken Ecke der Szene ist zudem noch ein Zurück-Button zu finden, welcher die Hauptmenü Szene lädt.

\subsection{Speicherdaten für die Rätsel}
Damit der Fortschritt gespeichert bleibt, muss der Gesamtpunktestand in einem Savegame gespeichert werden. Zudem muss gespeichert werden, welche Rätsel gelöst wurden und wie viele Punkte bei jedem Rätsel noch erhalten werden können, wenn dieses gelöst wird.\\

Die Speicherdaten werden in einem Binärformat gespeichert, um zu verhindern dass sie einfach vom Spieler manipuliert werden können.\\
Dafür wird zuerst eine serialisierbare Klasse angelegt:
\begin{lstlisting}[language=csh, caption={Serialisierbare Speicherdaten für die Rätsel}]
[Serializable]
public class SaveData {
    [Serializable]
    public class RiddleData {
        public int RemainingPoints;
		public bool completed;

        public RiddleData(int r, bool c) {
            RemainingPoints = r;
            completed = c;
        }
    }

    public int TotalPoints;

    public Dictionary<int, RiddleData> RiddleDataDict = new Dictionary<int, RiddleData>();

    public SaveData() {
		TotalPoints = 0;
        for (int i = 1; i <= RiddleInfo.RiddleAmount; i++) {
            RiddleInfo.SimpleRiddleInfo info = RiddleInfo.loadSimpleInfoFromXML(i);
            RiddleDataDict.Add(i, new RiddleData(info.maxPoints, false));
        }
    }
\end{lstlisting}
Hier wird in einem Dictionary für jede RätselID gespeichert, ob das Rätsel bereits abgeschlossen wurde und wie viele Punkte verbleiben. Beim Initialisieren werden dabei alle Rätsel auf nicht abgeschlossen gesetzt und der maximale Punktestand aus den XML-Dateien wird für die verbleibenden Punkte gesetzt.\\
Die Klasse bietet folgende Methoden um die Speicherdaten zu lesen oder verändern:
\begin{itemize}
\item \textbf{bool IsCompleted(int id)} gibt true zurück wenn das Rätsel schon abgeschlossen wurde
\item \textbf{void complete(int id)} speichert für die angegebene ID, dass das Rätsel abgeschlossen wurde
\item \textbf{int GetRemainingPoints(int id)} gibt die verbleibenden Punkte für das Rätsel zurück
\item \textbf{void UpdateRemainingPoints(int id, int points)} speichert die verbleibenden Punkte für das Rätsel
\item \textbf{void UpdateTotalPoints(int points)} speichert einen neuen Gesamtpunktestand
\end{itemize}

Um die Speicherdaten zu speichern und laden wird die SaveDataManager Klasse verwendet.
Diese besitzt eine statische Variable der SaveData Klasse, über welche alle anderen Klassen auf die Speicherdaten zugreifen können. Mithilfe der SaveGame() Methode werden diese Daten dann in einer Datei gespeichert:
\begin{lstlisting}[language=csh, caption={Methode um SaveData in eine Datei zu speichern}]
public static void SaveGame() {
        if (!Directory.Exists(Directory.GetCurrentDirectory() + "/Saves"))
            Directory.CreateDirectory(Directory.GetCurrentDirectory() + "/Saves");
        FileStream file = File.Create(RiddleSavePath);
        BinaryFormatter bf = new BinaryFormatter();
        bf.Serialize(file, RiddleSaveData);
        file.Close();
        Debug.Log("Game saved");
    }
\end{lstlisting}
Zuerst wird überprüft, ob der 'Saves' Ordner bereits existiert, wenn nicht wird dieser angelegt. Dann wird eine neue Datei und ein BinaryFormatter angelegt. Mithilfe des BinaryFormatters werden die Daten dann in Binärdaten serialisiert und diese in die Datei geschrieben, welche danach geschlossen wird.\\

Um Speicherdaten wieder aus der Datei zu laden, kann die LoadGame() Methode verwendet werden, welche wiederum mithilfe eines BinaryFormatters die Daten aus der Datei liest, deserialisiert und in die statische Variable schreibt. Wird hier keine Datei gefunden, so wird eine neue Instanz der SaveData Klasse angelegt, welche in die Variable geschrieben wird.\\

Des weiteren gibt es eine ResetData() Methode, welche die Speicherdatei löscht und die statische Variable auf eine neue Instanz der SaveData Klasse setzt.

\subsection{Skripte für die Rätselprefabs}
Für Rätsel, die mehr Interaktionsmöglichkeiten bieten, als nur einen Text oder eine Zahl als Ergebnis einzugeben, werden weitere Skripte benötigt. Diese können weitere Funktionen für die Rätsel implementieren, um zB Elemente im rechten Bereich mit der Maus verschieben zu können. In diesem Abschnitt werden alle zusätzlichen Skripte, die in den Prefabs der Rätsel verwendet werden, beschrieben.

\subsubsection{Ergebnis Inputs}
Diese Skripte werden von Rätseln verwendet, die als Lösung eine einfache Text oder Zahlen Eingabe haben. \\
\begin{lstlisting}[language=csh, caption={Skript für einen zweistelligen Zahlen Input}]
public class NumberInput : MonoBehaviour {
    public InputField tens;
    public InputField ones;

    public bool IsInputValid() {
        return ones.text.Length > 0 || tens.text.Length > 0;
    }

    public int GetInput() {
        if (tens.text.Length == 0)
            return int.Parse(ones.text);
        if (ones.text.Length == 0)
            return int.Parse(tens.text);
        return int.Parse(tens.text) * 10 + int.Parse(ones.text);
    }
}
\end{lstlisting}
Der Zahleninput ist dabei auf maximal zwei Stellen beschränkt. Im Editor werden die zwei InputFelder aus der Szene für die Einer und Zehner Variablen zugewiesen. Diese InputFelder müssen im Editor auf nur Zahlen und maximal ein Zeichen begrenzt werden.\\
Das Skript bietet dann eine Methode, die überprüft ob die Eingabe syntaktisch korrekt ist, was sie in diesem Fall ist, sobald eines der beiden Felder befüllt wurde. Eine Eingabe von 1 im Zehner Feld und nichts im Einer Feld wird hierbei als korrekt gewertet, da diese als eine Eins ausgewertet wird.\\
Mit der GetInput() Methode kann die momentan eingegebene Zahl ausgelesen werden.\\

Das Skript für den Textinput ist sehr ähnlich aufgebaut:
\begin{lstlisting}[language=csh, caption={Skript für einen zweistelligen Zahlen Input}]
public class TextInput : MonoBehaviour {
    public InputField[] letters;

    public bool IsInputValid() {
        foreach (InputField i in letters) {
            if (i.text.Length == 0)
                return false;
        }
        return true;
    }

    public string GetInput() {
        string s = "";
        foreach (InputField i in letters)
            s += i.text;
        return s.ToLower();
    }
}
\end{lstlisting}
Hier muss für jeden Buchstaben ein einzelnes Input Feld angelegt werden, welche dann im InputField Array 'letters' gespeichert werden. Auch hier sollten die Input Felder auf Buchstaben und maximal ein Zeichen begrenzt werden. Der Text zählt als syntaktisch korrekt, sobald in jedem Feld ein Buchstabe steht.\\
Die GetInput() Methode liefert hier alle Buchstaben der InputFelder als string verkettet und in Kleinbuchstaben.

\subsubsection{StaticSelection}
Dieses Skript erlaubt es, aus mehreren Feldern eine bestimmte Anzahl zu selektieren.\\
Dabei müssen folgende Variablen im Editor gesetzt werden:
\begin{itemize}
\item \textbf{public Button[] buttons} hier werden die Felder gespeichert, die selektiert werden können, die Felder müssen dabei Buttons sein, damit sie anklickbar sind. Diese Buttons benötigen als erstes Child Element ein Image, welches aktiviert wird, sobald das Feld ausgewählt ist
\item \textbf{public Text counter} in diesen Text wird die Anzahl, an noch selektierbaren Feldern, geschrieben
\item \textbf{public int maxAllowedSelections} hiermit wird angegeben wie viele Felder maximal gleichzeitig ausgewählt werden dürfen
\end{itemize}
In der Variable 'public List<int> selectedIndices' wird dann vom Skript gespeichert, welche Buttons aus dem Array momentan selektiert sind.\\
\begin{lstlisting}[language=csh, caption={Start Methode des StaticSelection Skripts}]
void Start() {
        int x = 0;
        foreach (Button b in buttons) {
            int tempx = x;
            b.onClick.AddListener(() => OnClick(tempx));
            x++;
        }
    }
\end{lstlisting}
In der Start Methode wird für jedes Feld (welche Buttons sind) ein onClick Listener festgelegt. Das Feld ruft bei einem Klick dann die OnClick(int index) Funktion des Skripts, mit seinem Index im 'buttons' Array als Parameter, auf.
\begin{lstlisting}[language=csh, caption={OnClick Methode des StaticSelection Skripts}]
public void OnClick(int index) {
        Image selectedImage = buttons[index].transform.GetChild(0).GetComponent<Image>();
        if (selectedIndices.Contains(index)) {
            selectedAmount--;
            selectedIndices.Remove(index);
            selectedImage.enabled = false;
        } else if (selectedAmount < maxAllowedSelections) {
            selectedAmount++;
            selectedIndices.Add(index);
            selectedImage.enabled = true;
        }
    }
\end{lstlisting}
Ist der Index des angeklickten Felds bereits in der 'selectedIndices' Liste enthalten, ist das Feld bereits selektiert. Dann muss das Feld wieder deselektiert werden, dazu wird der Index aus der Liste entfernt und das Bild deaktiviert, das anzeigt dass das Feld selektiert wurde. Außerdem wird die Anzahl der momentan selektierten Felder verringert.\\
Ist der Index hingegen nicht in der Liste enthalten, wird er zur Liste hinzugefügt, das Bild aktiviert und die Anzahl der selektierten Felder erhöht.\\

Um zu überprüfen ob die Lösung korrekt ist, muss überprüft werden ob in der Liste alle Indizes der Felder enthalten sind, die für die korrekte Lösung selektiert sein müssen.\\

Das Skript kommt zum Beispiel in Rätsel 3 zum Einsatz:
\graphic{r003}{Rätsel 3}
Hier müssen 2 der 9 Personen ausgewählt werden. Dabei ist jedes Personenfeld ein Button, welcher als Child das Bild des Kreises hat. Links findet sich der Text für den Counter, wie viele Felder noch angeklickt werden können. Das StaticSelection Skript ist dabei auf dem Hintergrund. Wird auf eine der Personen geklickt, so erscheint der Kreis um diese Person (solange noch ein Feld ausgewählt werden darf). Wird eine Person angeklickt, bei der der Kreis bereits sichtbar ist, so verschwindet dieser wieder.\\
In der R003 Klasse, werden dann die folgenden Methoden verwendet, um das Ergebnis auszuwerten:
\begin{lstlisting}[language=csh, caption={Methoden in der Klasse von Rätsel 3}]
public override bool checkResult() {
        List<int> selected = interactiveArea.GetComponent<StaticSelection>().selectedIndices;
        return selected.Contains(1) && selected.Contains(2);
    }

    public override bool isResultValid() {
        return interactiveArea.GetComponent<StaticSelection>().selectedIndices.Count == 2;
    }
\end{lstlisting}
In checkResult(), die Methode die prüft, ob das Ergebnis korrekt ist, wird dann überprüft, ob die Buttons mit Index 1 und 2 (die Personen oben in der Mitte und oben links), angeklickt sind. Das wird gemacht, indem geprüft wird ob die beiden Indizes in der 'selectedIndices' Liste des Skript enthalten sind.\\
Die Methode isResultValid(), die auf syntaktische Korrektheit testet, prüft nur ob zwei Personen ausgewählt wurden, also ob die Liste eine Länge von 2 hat.

\subsubsection{DynamicSelection}
Mit dem DynamicSelection Skript kann ein Bild, das einen Bereich selektiert (zB ein Kreis) überall in einem vorgegebenen Bereich platziert werden. Um den Kreis an einer Stelle zu platzieren muss dabei nur an diese Stelle geklickt werden.
\begin{lstlisting}[language=csh, caption={DynamicSelection Skript}]
    public Image marker;
    private Vector2 areaMin;
    private Vector2 areaMax;

	void Start() {
        Rect rect = GetComponent<RectTransform>().rect;
        areaMin = transform.TransformPoint(new Vector2(rect.xMin, rect.yMin));
        areaMax = transform.TransformPoint(new Vector2(rect.xMax, rect.yMax));
    }

    public override void OnPointerDown(PointerEventData eventData) {
        float x = Input.mousePosition.x;
        float y = Input.mousePosition.y;
        if (x < areaMax.x && x > areaMin.x && y > areaMin.y && y < areaMax.y)
            marker.transform.position = new Vector2(x, y);
    }
\end{lstlisting}
Zuerst werden die Koordinaten des erlaubten Bereichs in zwei Vektoren gespeichert. Diese werden durch die Größe des Elements bestimmt. Das Skript muss also auf das Element gesetzt werden, in welchem der Marker platziert werden darf. \\
Wird dann eine Stelle angeklickt, wird falls die Koordinaten im Bereich liegen, die Position des marker Images auf diese Stelle gesetzt.\\
Um das Ergebnis zu überprüfen, kann dann eine Distanz zwischen der Position des Markers und der korrekten Position berechnet werden. Liegt diese unter einem Wert, so wird die Lösung als korrekt gewertet.\\

Dieses Skript wird zB von Rätsel 20 verwendet:
\graphic{r020}{Rätsel 20}
Hier muss eine Wand markiert werden, die durchbrochen werden muss. Wird an eine Stelle im Labyrinth geklickt, wird der rote Kreis an dieser Stelle platziert. In der R020 Klasse wird dann das Ergebnis überprüft:
\begin{lstlisting}[language=csh, caption={checkResult Methode in der Klasse von Rätsel 3}]
public override bool checkResult() {
        Image marker = interactiveArea.transform.GetChild(0).GetComponent<DynamicSelection>().marker;
        return Vector2.Distance(marker.transform.localPosition, result) < 7;
    }
\end{lstlisting}
In dieser Methode wird der Abstand der Position des Markers, aus dem DynamicSelection Skript, und der Position der korrekten Lösung berechnet (diese ist im Vector 'result' gespeichert). Ist dieser Abstand kleiner als 7 (ca. die halbe Größe des Kreises), so wird die Lösung als korrekt gewertet.\\
Der Test auf syntaktische Korrektheit gibt hier immer true zurück, da der Kreis von Anfang an im Bild platziert ist und nur umplatziert wird.

\subsubsection{ToggleButtonController}
Mit diesem Skript können mehrere ToggleButtons gesteuert werden. Es kann verwendet werden wenn zB 2 von 4 Objekten ausgewählt werden sollen.\\
Das Skript beinhaltet folgende Variablen:
\begin{itemize}
\item \textbf{public Sprite offSprite} der Sprite für nicht angeklickte Buttons
\item \textbf{public Sprite onSprite} der Sprite für angeklickte Buttons
\item \textbf{public Button[] buttons} ein Array mit allen Buttons
\item \textbf{public bool[] buttonStates} ein Array, das bools enthält welche Buttons angeklickt sind (true) und welche nicht (false), das Array wird zu Beginn mit false initialisiert
\end{itemize}

Im Editor wird für jeden Button die OnClick Funktion gesetzt, welche dann die ChangeImage(int id) Funktion des Skript mit dem jeweiligen Index aufruft.
\graphic{toggleButtonSetOnClick}{Setzen der OnClick Funktion im Editor}

In der ChangeImage(int id) Funktion des Skripts wird dann der boolean an der entsprechenden Stelle im Array geändert und der Sprite des Buttons wird getauscht (von on zu off bzw von off zu on). Um das Ergebnis zu überprüfen, werden die Booleans aus dem Array verwendet, um zu bestimmen, ob die richtigen Buttons angeklickt sind.\\

Rätsel 6 verwendet dieses Skript:
\graphic{r006}{Rätsel 6}
Hier muss ausgewählt werden, welche der Totems eine bestimmte Silhouette ergeben. Im Bild sind momentan Antwort B und C selektiert.
\begin{lstlisting}[language=csh, caption={checkResult Methode in der Klasse von Rätsel 6}]
public override bool checkResult() {
        bool[] selected = interactiveArea.GetComponent<ToggleButtonController>().buttonStates;
        return selected[0] && !selected[1] && selected[2] && selected[3] && !selected[4];
    }
\end{lstlisting}
Um zu bestimmen, ob die Lösung richtig ist, wird geprüft indem getestet wird, ob die Booleans für Buttons 0, 2 und 3 true sind (also diese Buttons gedrückt wurden) und die für Buttons 1 und 4 false sind (also diese Buttons nicht gedrückt sind). Ist das der Fall, ist die Lösung korrekt. Wurde ein falscher Button angeklickt oder ein Button vergessen anzuklicken, so wird hier false zurückgegeben.

\begin{lstlisting}[language=csh, caption={isResultValid Methode in der Klasse von Rätsel 6}]
    public override bool isResultValid() {
        bool[] selected = interactiveArea.GetComponent<ToggleButtonController>().buttonStates;
        foreach (bool b in selected) {
            if (b)
                return true;
        }
        return false;
    }
\end{lstlisting}
Für die Überprüfung auf syntaktische Korrektheit wird hier über das Array iteriert und sobald ein boolean den Wert true hat, ist das Ergebnis syntaktisch korrekt, da dann mindestens ein Button angeklickt wurde.

\subsubsection{SnapDragController}
Der SnapDragController wird verwendet, wenn Objekte mit der Maus verschoben werden können sollen. Dabei können diese Objekte nur an bestimmten Position abgelegt werden. Wenn die Position bereits belegt ist oder die Maustaste losgelassen wird, während keine mögliche Position in der Nähe ist, springt das Objekt an seine Startposition zurück.\\
Folgende Variablen müssen für dieses Skript im Editor gesetzt werden:
\begin{itemize}
\item \textbf{ public Image[] tileImages} in dieses Array müssen alle Image Objekte eingefügt werden, die verschiebbar sein sollen
\item \textbf{public Vector2[] locations} hier müssen alle Positionen angegeben werden, an denen die Bilder abegelegt werden können (ohne die Startpositionen der Bilder)
\item \textbf{public bool enableRotation} Wird dieser boolean auf true gesetzt, so können die Bilder durch einen Klick gedreht werden, steht hier false, so können die Bilder nicht gedreht werden
\end{itemize}
Für die Locations könnte man anstelle eines Vector2 Arrays auch ein GameObject Array verwenden. Damit könnten in der Szene leere GameObjects an allen Stellen angelegt werden, an die die Images gesetzt werden dürfen. Dann können die Koordinaten für die Punkte aus den GameObjects gelesen werden und nicht von Hand eingetragen werden. Für Rätsel die eventuell sehr viele mögliche Position haben, wäre dieser Ansatz sinnvoller, er würde aber mehr Objekte in der Szene benötigen.

\begin{lstlisting}[language=csh, caption={Start Methode des SnapDragControllers}]
public void Start() {
        foreach (Vector2 v in locations)
            possibleLocations.Add(new Location(v));
        tiles = new Tile[tileImages.Length];
        for (int i = 0; i < tileImages.Length; i++)
            tiles[i] = new Tile(tileImages[i]);
        Rect rect = GetComponent<RectTransform>().rect;
        areaMin = transform.TransformPoint(new Vector2(rect.xMin, rect.yMin));
        areaMax = transform.TransformPoint(new Vector2(rect.xMax, rect.yMax));
    }
\end{lstlisting}
Zuerst werden aus den Vektoren für die möglichen Positionen, an denen die Bilder abgelegt werden, Objekte der Klasse Location erstellt. Diese Klasse beinhaltet neben dem Vektor für die Position noch einen Boolean, der angibt ob die Position bereits belegt ist. Die Locations werden dann in einer Liste gespeichert.\\
Auch für die verschiebbaren Bilder wird jeweils ein Objekt erzeugt, hier von der Klasse Tile. In dieser Klasse wird zusätzlich zum Image noch die Startposition des Images sowie der Index der Location, die das Image momentan belegt, gespeichert.\\
Dann wird hier, wie bei der DynamicSelection, noch die Größe des Bereichs, in welchem die Objekte verschoben werden dürfen, aus dem Element gelesen.\\

Für jedes der Images wird im Editor ein EventTrigger für die Events 'Pointer Down' und 'Pointer Up' festgelegt, so dass die entsprechende Funktion im Skript mit dem Index des Images aufgerufen wird:
\graphic{dragEventTrigger}{Im Editor gesetzte Event Trigger}

\begin{lstlisting}[language=csh, caption={OnPointerDown Methode des SnapDragControllers}]
 public void OnPointerDown(int tile) {
        if (!Input.GetMouseButtonDown(0))
            return;
        pointerDownTime = Time.time;
        dragging = tile;
        if (tiles[dragging].occupiedLocationIndex != -1)
            possibleLocations[tiles[dragging].occupiedLocationIndex].occupied = false;
        tiles[dragging].occupiedLocationIndex = -1;
    }

\end{lstlisting}
Im Skript wird in der OnPointerDown zuerst überprüft, ob die gedrückte Maustaste die linke Maustaste, also Maustaste 0 ist. Die Zeit des Klicks wird gespeichert, um einen kurzen Klick (zum Drehen) von gedrückt halten (zum Bewegen) unterscheiden zu können. Dann wird dragging auf den Index des Tiles gesetzt, das momentan bewegt wird.\\
Ist für das bewegte Tile momentan eine Location gespeichert (occupiedLocationIndex != -1), dann muss diese Location wieder als leer markiert werden. Dazu wird im Locations Array an der Stelle, die vorher vom Tile belegt wurde, occupied wieder auf false gesetzt und im bewegten Tile wird occupiedLocationIndex auf -1 gesetzt, um zu zeigen, dass dieses Tile momentan keine der möglichen Locations belegt.\\

\begin{lstlisting}[language=csh, caption={Update Methode des SnapDragControllers}]
void Update() {
        if (dragging == -1)
            return;
        float x = Input.mousePosition.x;
        float y = Input.mousePosition.y;
        if (x < areaMax.x && x > areaMin.x && y > areaMin.y && y < areaMax.y)
            tiles[dragging].image.transform.position = new Vector2(x, y);
    }
\end{lstlisting}
Wenn gerade ein Tile bewegt wird (dragging != -1 ist) dann wird die Position des Images von diesem Tile auf die momentane Mausposition gesetzt. Ist die Mausposition außerhalb des erlaubten Bereichs, in dem das Tile bewegt werden darf, wird die Position nicht verändert und das Image bleibt an der vorherigen Stelle.\\

\begin{lstlisting}[language=csh, caption={Ausschnitt der OnPointerUp Methode des SnapDragControllers}, escapeinside={(*}{*)}]
public void OnPointerUp() {
        // need to reset dragging first, else Update will set position while this function is executed
        if (!Input.GetMouseButtonUp(0))
            return;
        int lastDrag = dragging;
        dragging = -1;
        if (enableRotation && Time.time - pointerDownTime < 0.2) {
            tiles[lastDrag].image.transform.Rotate(0.0f, 0.0f, 90.0f, Space.Self);
        }
        (*...*)
\end{lstlisting}
Wird die Maustaste dann wieder losgelassen, wird zuerst die dragging Variable auf -1 gesetzt, da während diese Methode ausgeführt wird, bereits die Update Funktion wieder aufgerufen könnte, welche das Tile bewegen würde. Um innerhalb dieser Methode trotzdem zu wissen, welches Tile bewegt wurde wird der Wert der dragging Variable vorher noch in einer lokalen Variable gespeichert.\\
Ist Rotation im Skript aktiviert, so wird das Image um 90° nach rechts gedreht, falls zwischen PointerDown und PointerUp Zeit weniger als 0.2 Sekunden lagen.\\
\begin{lstlisting}[language=csh, caption={Setzen der Position in der OnPointerUp Methode}, escapeinside={(*}{*)}]
		(*...*)
 		float x = Input.mousePosition.x;
        float y = Input.mousePosition.y;

        if (!(x < areaMax.x && x > areaMin.x && y > areaMin.y && y < areaMax.y)) {
            tiles[lastDrag].image.transform.position = tiles[lastDrag].basePos;
        } else {
            bool locationFound = false;
            foreach (Location l in possibleLocations) {
                if (!l.occupied && Vector2.Distance(l.position, transform.InverseTransformPoint(Input.mousePosition)) < 25) {
                    tiles[lastDrag].image.transform.position = transform.TransformPoint(l.position);
                    tiles[lastDrag].occupiedLocationIndex = possibleLocations.IndexOf(l);
                    l.occupied = true;
                    locationFound = true;
                    break;
                }
            }
            if (!locationFound)
                tiles[lastDrag].image.transform.position = tiles[lastDrag].basePos;
        }
\end{lstlisting}
In der zweiten Hälfte der Methode wird dann die Position, an die das Image gesetzt werden muss, bestimmt. Ist die Mausposition momentan außerhalb des erlaubten Bereichs, so wird das Image zurück an seine Startposition (Tile.basePos) gesetzt.\\
Ansonsten werden alle möglichen Locations, an die das Tile platziert werden darf, getestet. Falls eine dieser Locations eine Distanz kleiner als 25 zur momentanen Position der Maus hat und sie noch nicht belegt ist, wird das Image an diese gesetzt. Dann wird im Tile noch der Index der Location die es jetzt belegt gesetzt und in der Location wird der occupied boolean auf true gesetzt.\\
Wurde keine mögliche Location gefunden, so wird das Image an seine Startposition zurückgesetzt.\\

Dieses Skript wird unter anderem von Rätsel 8 verwendet. In diesem müssen einzelne Platten zu einem Labyrinth zusammengesetzt werden:
\graphic{r008}{Rätsel 8}
Das Skript liegt hier auf dem Hintergrund und im Editor wurden die Images sowie die Locations als Variablen gesetzt:
\graphic{SnapDragControllerVariables}{Skript im Editor}
Auf den einzelnen Bildern müssen dann nur noch die EventTrigger für Pointer Down und Pointer Up wie oben beschrieben, gesetzt werden.\\
Um in der Rätsel Klasse zu überprüfen, ob die Lösung syntaktisch korrekt ist, wird überprüft ob für alle Tiles der occupiedLocationIndex gesetzt ist, also nicht -1 ist. Das heißt damit das Rätsel bestätigt werden darf, müssen alle Tiles platziert werden, also nicht mehr auf ihrer Startposition liegen.\\
\begin{lstlisting}[language=csh, caption={Überprüfen der Lösung eines SnapDragControllers}]
public override bool checkResult() {
        SnapDragController.Tile[] tiles = interactiveArea.transform.GetComponent<SnapDragController>().tiles;
        for (int i = 0; i < result.Length; i++) {
            if (tiles[i].occupiedLocationIndex != result[i])
                return false;
        }
        return true;
    }
\end{lstlisting}
Um zu prüfen ob die Lösung korrekt ist, wird für jedes Tile überprüft, ob es an der korrekten Location liegt, indem der Index der Location an der es liegt, mit dem Index der korrekten Location verglichen wird. Liegt ein Tile nicht an der richtigen Location, ist die Lösung falsch und false wird zurückgegeben.

\subsubsection{UIElementDragger}
Der UIElementDragger verhält sich sehr ähnlich zum SnapDragController, mit dem Unterschied, dass die Objekte hier überall im Bereich abgelegt werden können, nicht nur an festgelegt Stellen.\\
Dieses Skript muss auf alle Elemente gesetzt werden, die bewegt werden sollen.\\
\begin{lstlisting}[language=csh, caption={Berechnen der Breite des beweglichen Objekts und des Bereichs in dem es bewegt wird}]
void Start() {
        parent = transform.parent.gameObject;
        Rect rect = parent.GetComponent<RectTransform>().rect;
        parentMin = parent.transform.TransformPoint(new Vector2(rect.xMin, rect.yMin));
        parentMax = parent.transform.TransformPoint(new Vector2(rect.xMax, rect.yMax));
        RectTransform rectTransform = GetComponent<RectTransform>();
        Vector2 thisMax = transform.TransformPoint(new Vector2(rectTransform.rect.xMax, rectTransform.rect.yMax));
        Vector2 thisMin = transform.TransformPoint(new Vector2(rectTransform.rect.xMin, rectTransform.rect.yMin));
        width = thisMax.x - thisMin.x;
        height = thisMax.y - thisMin.y;
    }
\end{lstlisting}
In der Start Methode wird zuerst die minimale und maximale x und y Koordinate des parent Objekts berechnet und in den Vektoren parentMin und parentMax gespeichert.\\
Dann wird noch die Höhe und Breite des (zu bewegenden) Objekts berechnet und gespeichert.\\
\begin{lstlisting}[language=csh, caption={OnPointerDown Methode des UIElementDraggers}]
public override void OnPointerDown(PointerEventData eventData) {
        dragging = true;
        offset = eventData.position - new Vector2(transform.position.x, transform.position.y);
    }
\end{lstlisting}
Wird das Objekt angeklickt, wird diese Funktion aufgerufen, welche dann nur die dragging Variable auf true setzt und das Offset des Klicks zur Mitte des Objekts speichert. Dieses Offset wird benötigt, damit das Objekt, wenn man es an der linken Seite anklickt, nicht mit der Mitte an die Maus springt beim setzen der Position.\\
In der OnPointerUp Methode wird dann lediglich der dragging Boolean wieder auf false gesetzt.\\

In der Update Methode des Skript werden dann die Move und Rotate Methoden aufgerufen, hier wird zuerst die Move Methode beschrieben:
\begin{lstlisting}[language=csh, caption={Methode zum Bewegen des Objekts im UIElementDraggers}]
void Move() {
        float x = Input.mousePosition.x - offset.x;
        if (x + width / 2 > parentMax.x) {
            x = parentMax.x - width / 2;
        }
        if (x - width / 2 < parentMin.x) {
            x = parentMin.x + width / 2;
        }
        float y = Input.mousePosition.y - offset.y;
        if (y + height / 2 > parentMax.y) {
            y = parentMax.y - height / 2;
        }
        if (y - height / 2 < parentMin.y) {
            y = parentMin.y + height / 2;
        }
        transform.position = new Vector2(x, y);
    }
\end{lstlisting}
Die x und y Koordinaten des Klicks werden beide um das Offset verschoben, um die Koordinaten zu erhalten, an welche die Mitte des Objekts gesetzt werden muss, damit die Maus an derselben Stelle auf dem Objekts bleibt.\\
Dann wird noch geprüft, ob das Objekt an der neuen Position über den erlauben Bereich herausragen würde. Das heißt wenn die Mitte des Objekts + die halbe Breite außerhalb des Bereichs wäre, wird der x Wert so gesetzt, dass das gesamte Objekt noch im Bereich liegt. ZB wäre der größtmögliche x Wert somit beim Rand des Bereichs - die halbe Breite des Objekts, dann liegt der rechte Rand des Objekts genau am Rand des Bereichs.\\

\begin{lstlisting}[language=csh, caption={Methode zum Drehen des Objekts im UIElementDraggers}]
void Rotate() {
        if (Input.GetKeyDown(KeyCode.Q)) {
            transform.Rotate(0.0f, 0.0f, 90.0f, Space.Self);
            float d = height;
            height = width;
            width = d;
        }
        if (Input.GetKeyDown(KeyCode.E)) {
            transform.Rotate(0.0f, 0.0f, -90.0f, Space.Self);
            float d = height;
            height = width;
            width = d;
        }
    }
\end{lstlisting}
Ist die Q Taste gedrückt, so wird das Objekt um 90° im Uhrzeigersinn gedreht. Nach dem Drehen müssen dann noch Breite und Höhe des Objekts vertauscht werden. \\
Mit der E Taste wird das Objekt um 90° gegen den Uhrzeigersinn gedreht. Auch hier werden Breite und Höhe des Objekts vertauscht.\\

Dieses Skript wird nur von Rätsel 1 verwendet, hier muss aus den Dreiecken ein K gebildet werden:
\graphic{r001}{Rätsel 1}
Der Bereich, in dem die Dreiecke bewegt werden dürfen, ist die weiße Fläche. Deshalb sind die Dreiecke alle child Elemente, des Canvas das genau die Größe der weißen Fläche hat. Denn im Skript ist der Bereich, in dem die Objekte bewegt werden dürfen, auf das Parent Element begrenzt.\\
Auf jedes der Dreiecke wird dann das UIElementDragger Skript gesetzt, Variablen müssen hier im Editor nicht gesetzt werden.\\

Um in der Rätsel Klasse zu testen, ob die Lösung korrekt ist, wird zuerst überprüft ob alle Dreiecke die richtige Rotation haben. Sind alle Dreiecke richtig rotiert, wird für jedes Dreieck die Distanz zur korrekten Position berechnet. Ist dann die Summe aller dieser Distanzen kleiner als 30, so wird die Lösung als korrekt akzeptiert. Sind also zwei Dreiecke perfekt platziert, so darf das dritte Dreieck weiter von der perfekten Position weg sein.\\
Als syntaktisch korrekt zählt das Rätsel immer, da die Dreiecke von Anfang an im erlaubten Bereich liegen und auch nur innerhalb von diesem bewegt werden können.
\newpage
\section{Lernspiele}
\label{sec:lernspiele}
% \graphic{riddleScene}{Szenenaufbau für Rätsel}
% Text für das Design der Lernspiel szenen von der ersten bis zur letzten.
% Liste mit allen Szenen machen, von den Grundrechenarten bis zum endscreen
\subsection{Szene - Menü}

%TODO Variablen in den Skripten Checken
\subsubsection{Menü - Design}

\begin{figure}[htbp]
  \centering
  \includegraphics[width=0.65\textwidth,height=0.55\textheight,keepaspectratio]{images/menuLearning.PNG}
  \caption{Auswahl der Level}
  \label{MenüLernen01}
\end{figure}


In der Abbildung \ref{MenüLernen01} ist die Auswahl der sechs unterschiedlichen Level zu sehen. Die Szene besteht aus einem Zurück-Button und sechs weiteren für die Lernspiele. Mit dem zurück Button kommt man wieder in das Hauptmenü. Die restlichen Buttons leiten den Nutzer ein Menü weiter. In dem nächsten Menü werden die Alternativen und die Schwierigkeiten des Spiels festgelegt. Die Buttons wurden im Programm \textit{Inkscape vector graphics} erstellt. Damit die beiden Teile des Projektes zusammenpassen, wurde ein einheitliches Design überlegt. Dieses unterscheidet sich in den Farben und in den Buttons. Der Hintergrund, die Art der Buttons und die Schrifftart sind die gleichen. So das das Kind anhand der Farben einen Unterschied in den Abschnitten Rätseln und Lernspiele erkennen kann. In dieser Szene wurde für jeden Button ein PNG entworfen. Wie dies im Programm \textit{Inkscape} aussieht ist in der Abbildung \ref{ButtonInkscape} zu sehen.\\
\begin{figure}[htbp]
  \centering
  \includegraphics[width=0.65\textwidth,height=0.55\textheight,keepaspectratio]{images/InkscapeButtonGreen.PNG}
  \caption{Button-Design}
  \label{ButtonInkscape}
\end{figure}
Der Button besteht aus mehreren Objekten, welche in Abbildung \ref{ButtonInkscapeSingleTiles} zu sehen sind. Für den Hintergrund des Buttons wurden drei Rechtecke erzeugt, ein weißes, ein dunkelgrünes und ein hellgrünes. Diese werden immer kleiner und anschließend übereinander gelegt. Von allen Rechtecken werden die Kanten abgerundet. Als letztes werden noch zwei Textfelder eingefügt. Jedes der Textfelder ist unterschiedlich gestaltet, so dass am Ende die gewünschte Schrift rauskommt. 
\begin{figure}[htbp]
  \centering
  \includegraphics[width=0.65\textwidth,height=0.55\textheight,keepaspectratio]{images/InkscapeButtonSingleTiles.PNG}
  \caption{Button Einzelteile}
  \label{ButtonInkscapeSingleTiles}
\end{figure}
Nachdem das Design für die einzelnen Buttons fertig sind, werden diese in Unity auf die \textit{Image} Komponente des Buttons gezogen. Dies ist in der Abbildung \ref{UnityButtonImgChange} zu sehen.
In dieser Abbildung, markiert ein grüner Pfeil die Stelle ab der das Image des Buttons ausgewählt wird.
\begin{figure}[htbp]
  \centering
  \includegraphics[width=0.65\textwidth,height=0.55\textheight,keepaspectratio]{images/buttonImgChange.PNG}
  \caption{Button Image ändern in Unity}
  \label{UnityButtonImgChange}
\end{figure}
\subsubsection{Menü - Skript}

In diesem Abschnitt wird das Skript erklärt, welches die Auswahl der Level verwaltet.
Das Skript wird hierfür in kleine Teile zerlegt und anschließend erklärt.\\

\begin{lstlisting}[language=csh, caption={MenuPickLevel.cs Klasse Menü Imports}]
using System.Collections;
using System.Collections.Generic;
using UnityEngine;
using UnityEngine.UI;
using UnityEngine.SceneManagement;
using System;
\end{lstlisting}

Zu Beginn des Skriptes werden verschiedene Bibliotheken eingebunden. Diese werden für die Skripte in den nachfolgenden Abschnitten sehr ähnlich sein oder sich gar nicht unterscheiden. Mithilfe des Befehls \textit{using} werden Bibliotheken von C\# und Unity importiert. Diese werden benötigt, um zum Beispiel Arrays zu erstellen und verwalten zu können. Mit den Bibliotheken von Unity kann man auf die Objekte in Unity per Skript zugreifen und diese verwalten oder bearbeiten.\\

\begin{lstlisting}[language=csh, caption={MenuPickLevel.cs Klasse Menü Variablen}]
public class MenuPickLevel : MonoBehaviour
{
	public Button[] buttonsMenu = new Button[7];

	public enum LvlType{
		BACK = 1,
		QUANTITIES = 2,
		MATHOPERATIONS = 3,
		TRIANGLE = 4,
		LIGHTNIGVIEW = 5,
		PATTER = 6,
		MIXEDWORDS = 7
	}

	public static int loadButtonsPrefab = 0;
\end{lstlisting}

Nachdem alle notwendigen Bibliotheken eingebunden wurden, wird die Klasse deklariert. Die Klasse bekommt hinten \textit{MonoBehaviour} angehängt, dies ermöglicht unteranderem Funktionen wie \textit{void start()} und \textit{void() update} zu nutzen. Wenn ein Skript diese Funktionen nicht benötigt, weil es zum Beispiel nur etwas ausrechen muss, so kann das \textit{MonoBehaviour} dahinter weggelassen werden.\\
In den darauf folgenden Zeilen werden die Variablen deklariert die für das Skript notwendig sind. Zuerst werden die Buttons deklariert. Der auskommentierte Code, welcher in der Abbildung für eine bessere Übersichtlichkeit nicht dargestellt wird, stellt eine alternative Lösung dar. In Unity können Buttons in das Skript eingefügt werden, welches man in Abbildung 5.16 sehen kann. Dadurch können wir im Code dann auf das Objekt des jeweiligen Buttons zugreifen und mit diesen arbeiten. Es gibt zwei Möglichkeiten mehrere Buttons zu implementieren. Entweder man deklariert die anzahl an Buttons selbst oder man nutzt ein Array dafür. Beides ist in den folgenden Abbildungen zu sehen.
\begin{figure}[htbp]
  \centering
  \includegraphics[width=0.65\textwidth,height=0.55\textheight,keepaspectratio]{images/buttonZuweisen.PNG}
  \caption{Buttons in Unity einzeln zuweisen}
  \label{singleButtonUnity}
\end{figure}
\begin{figure}[htbp]
  \centering
  \includegraphics[width=0.65\textwidth,height=0.55\textheight,keepaspectratio]{images/buttonsIntoArray.PNG}
  \caption{Buttons in Unity in Array einfügen}
  \label{arrayButtonUnity}
\end{figure}
In C\# gibt es standardmäßig nicht die Möglchkeit ein \textit{Const Array} zu erstellen. Für jeden Button wird in diesem Skript eine Konstante benötigt, denn diese soll dem nachfolgenden Skript weiterleiten welcher Button gedrückt wurde. Dies ist wichtig, da die Level unterschiedliche Varianten und Schwierikeitsgrade beinhalten und dementsprechend unterschiedliche Buttons angezeigt werden. Um ein Array zu deklarieren, welches ähnlich zu einer Konstanten funktioniert, kann ein Array welches \textit{readonly} ist oder eine Enumeration verwendet werden. Eine Enumeration ist eine Reihe benannter Konstanten, das heißt alle Komma die von Hand einzeln erzeugt wurden, können in die Enumeration eingespeichert werden. Da diese Lösung weniger Code benötigt und für mehrere Buttons effektiver ist, wurde diese Möglichkeit auch umgesetzt. In der Menü-Szene sind beide Möglichkeiten im Code geschrieben, aber die Umsetzung alle Buttons und Konstanten einzeln zu deklarieren wurde auskommentiert.\\
Als letztes wird eine statische Variable deklariert, welche später den Wert einer Konstanten zugewiesen bekommt. In der nächsten Szene kann dann auf diese zugegriffen werden und somit ist dem nachfolgenden Skript dann klar, welche Buttons für die nächste Szene geladen werden müssen. In Unity gibt es unterschiedliche Möglichkeiten, um Werte zwischen Skripten transferieren zu können. Einmal wären das static Variablen. Unity selbst bietet aber auch die Möglichkeit einen Wert in eine Varibale in einem file zu speichern, dafür wird der \textit{PlayerPrefs.setInt/String/Float('variablen name', variablen wert);} verwendet. Diese Variante wird in diesem Skript nicht verwendet, da nur die Information aus dem alten Skript benötigt wird und diese nur für diese beiden Skripte notwendig ist. Wenn die Information für mehrere Skripte notwendig wäre, könnte man diese durch den \textit{PlayerPrefs} Befehl speichern.\\

\begin{lstlisting}[language=csh, caption={MenuPickLevel.cs Klasse Menü Start-Funktion}]
	void Start()
	{
		Debug.Log(buttonsMenu.Length);
		for (int i = 0; i < buttonsMenu.Length; i++){
			int temp = i;
			buttonsMenu[temp].onClick.AddListener(() => loadAdvancedOptions((LvlType)temp + 1));
		}
	}
\end{lstlisting}
In der \textit{Start} Funktion werden jedem Button ein \textit{Listener} hinzugefügt. Es gibt auch hier wieder die Möglichkeit dies für jeden Button einzeln zu erledigen oder mithilfe eines Arrays und der Enumeration dies in einer \textit{FOR}-Schleife durchzuführen.\\
Die Schleife startet bei 0 und läuft bis sie durch alle Buttons durch iteriert hat. Dabei kann es zu einem Problem kommen, das die Variable i gefangen nimmt. Das bedeutet, dass egal welcher Button gedrückt wird die letzte Möglichkeit der Iteration geladen wird. Auf das Projekt übertragen würde das bedeuten, dass jeder Button zum selben Level führt. Um dieses Problem zu verhindern, wird eine temporäre Variable \textit{temp} deklariert, die den Wert von i zugewiesen bekommt. Jetzt wird jedem Button aus dem Array an der Position \textit{temp} ein \textit{onClick-Event} hinzugefügt, welches die Möglichkeit bietet, wenn der Button gedrückt wurde etwas oder eine Funktion auszuführen. In diesem Fall wird die \textit{AddLister Funktion} aufgerufen. Mit den leeren Klammern und dem Pfeil (() =>) wird eine Funktion deklariert, die ausgeführt wird, wenn der Button gedrückt wurde. In diesem Fall wird die Funktion, die Szene mit den jeweiligen Varianzen an Modi und Schwierigkeiten lädt aufgerufen. Hierbei wird der Integer Wert aus \textit{temp} verwendet. Dieser muss in den Enumerationstypen umgewandelt werden, diese Umwandlung wird casten genannt. Dabei wird vor die Variable der gewünschte Typ in Klammern geschrieben (LvlTyp). Danach wird die Variable \textit{temp + 1} verwendet, das hat den Grund, dass später mit dieser Variablen gearbeitet wird und das Skript für das erweiterte Menü diese Variable ab der Stelle eins bearbeitet. In einer früheren Version gab es in dieser Szene einen Menü-Button, der die Stelle null belegt hat und später entfernt wurde. Die Skripte funktionieren aber ohne diesen Button ohne Probleme, deshalb wurde es nicht angepasst.\\

\begin{lstlisting}[language=csh, caption={MenuPickLevel.cs Klasse Menü loadAdvancedOptions- Funktion}]
	public void loadAdvancedOptions(LvlType choice){
		switch (choice)
		{
			case LvlType.BACK:
				loadButtonsPrefab = (int)LvlType.BACK;
				SceneManager.LoadScene("MainMenu");
				break;
			case LvlType.QUANTITIES:
				loadButtonsPrefab = (int)LvlType.QUANTITIES;
				SceneManager.LoadScene("MenuLearingAdvancedOptions");
				break;
			case LvlType.MATHOPERATIONS:
				loadButtonsPrefab = (int)LvlType.MATHOPERATIONS;
				SceneManager.LoadScene("MenuLearingAdvancedOptions");
				break;
			case LvlType.TRIANGLE:
				loadButtonsPrefab = (int)LvlType.TRIANGLE;
				SceneManager.LoadScene("MenuLearingAdvancedOptions");
				break;
			case LvlType.LIGHTNIGVIEW:
				loadButtonsPrefab = (int)LvlType.LIGHTNIGVIEW;
				SceneManager.LoadScene("MenuLearingAdvancedOptions");
				break;
			case LvlType.PATTER:
				loadButtonsPrefab = (int)LvlType.PATTER;
				SceneManager.LoadScene("MenuLearingAdvancedOptions");
				break;
			case LvlType.MIXEDWORDS:
				loadButtonsPrefab = (int)LvlType.MIXEDWORDS;
				SceneManager.LoadScene("MenuLearingAdvancedOptions");
				break;

			default:
				break;
		}
	}
\end{lstlisting}

In Zeile 46 beginnt die Funktion, die die nächste Szene lädt. Die Szene unterscheidet mit einem \textit{switch-case}-Statement welcher Button gedrückt wurde und speichert den passenden Wert in die static Variable ein. Der Wert aus der Enumeration wird hierbei in einen Integer Wert umgewandelt. Danach wird die nächste Szene geladen. Der Zurück-Button leitet den Nutzer zum Hauptmenü zurück und die anderen Buttons laden die neue Szene, also das erweiterte Menü. Das erweiterte Menü ist für alle Buttons die gleiche Szene, da diese sich nur in der Anzahl an Auswahlmöglichkeiten unterscheidet. Dies wird aber im nächsten Abschnitt genauer erläutert.\\
In dieser Szene ist noch anzumerken, dass es einen Fehler bei den Konstanten gab, so dass im erweiterten Menü zweimal dasselbe Level geladen wurde. Wie dieser genau zustande kam und wie er gelöst wurde wird im nächsten Abschnitt genauer erläutert.

\subsection{Szene - Erweitertes Menü}
\subsubsection{Erweitertes Menü - Design}
In diesem Abschnitt, wird das Design des erweiterten Menüs erklärt. In diesem Menü gibt es wieder einen Zurück-Button, der in das vorherige Menü zurückleitet. Auf der linken Seite des Menüs werden die alternativen Spielmodi angezeigt und auf der rechten Seite die Schwierigkeitsgraden. Zu sehen ist dies in der Abbildung \ref{AdvancedMenü}. Die Alternativen sind nicht für jedes Level vorhanden oder gleich, wenn es keine zur Auswahl gibt, ist die linke Hälfte  leer. Für die beiden Seiten wurden auch wieder mehrere Button-Designs angefertigt.
\begin{figure}[htbp]
  \centering
  \includegraphics[width=0.65\textwidth,height=0.55\textheight,keepaspectratio]{images/basicOperatorsAdvanced.PNG}
  \caption{Auswahl der einzelnen Optionen eines Levels}
  \label{AdvancedMenü}
\end{figure}
Um das ausgewählte Spiel starten zu können, muss auf beiden Seiten eine Option gewählt werden. Solange nicht auf beiden Seiten ein Button ausgewählt wurde, ist der Start-Button deaktiviert. Wenn ein Button auf einer Seite ausgewählt, wurde wird dieser keaktiviert und kann nicht wiederholt ausgewählt werden. Sobald eine andere Alternative ausgewält wird, wird diese deaktiviert und die andere wieder aktiviert. So wird verhindert, dass nie mehr als eine möglichkeit aktiviert sein kann. Wenn ein Level ohne Alternativenwahl ausgewählt wurde, muss nur eine Schwierigkeit ausgewählt werden, um das Spiel zu starten. In der Abbildung \ref{singleChoice} ist eine Auswahl zu sehen. Der Start-Button ist hier noch deaktiviert, weil keine Schwierigkeit gewählt wurde. 
\begin{figure}[htbp]
  \centering
  \includegraphics[width=0.65\textwidth,height=0.55\textheight,keepaspectratio]{images/basicOperatorsAdvancedSingleChoice.PNG}
  \caption{Einzelne Auswahl}
  \label{singleChoice}
\end{figure}
Nachdem auf jeder Seite ein Button gewählt wurde, ist der Start-Button nicht mehr deaktiviert und der Spieler kann das Spiel starten.
\begin{figure}[htbp]
  \centering
  \includegraphics[width=0.65\textwidth,height=0.55\textheight,keepaspectratio]{images/basicOperatorsAdvancedFinish.PNG}
  \caption{Auswahlbedingung um das Level zu laden}
  \label{startLevel}
\end{figure}
\subsubsection{Erweitertes Menü - Skript}
In diesem Abschnitt, wird das Skript erklärt. Da die Imports die selben sind, werden diese nicht erneut erklärt.
\begin{lstlisting}[language=csh, caption={MenuPickLevelAdvanced.cs Variablendeklaration}]
public class MenuPickLevelAdvanced : MonoBehaviour
{

	//Buttons
	public Button back;
	public Button start;

	public List <Button> leftSideList = new List<Button>();
	private List<Button> rightSideList = new List<Button>();

	public GameObject buttonPref;
	public GameObject spawnerLeft;
	public GameObject spawnerRight;
	private GameObject spawnedObject;

	public Sprite[] quantitiesModes = new Sprite[2];
	public Sprite[] mathModes = new Sprite[4];
	public Sprite[] difficulty = new Sprite[3];

	public static int maxNumberStatic = 0;
	public static int lvlAmmountStatic = 0;
	public static int fourChoices = 0;
	public static int wallSize = 0;

	private int loadButtonsNumber = 0;
	private bool leftSide = false;
	private bool rightSide = false;
	private Vector3 scaleSize = new Vector3 (1.0f, 1.0f, 1.0f);
\end{lstlisting}
%TODO Variablen in den Skripten Checken
Die Klasse beinhaltet mehr Variablen und Objekte. In diesem Fall wurden zwei Buttons deklariert, da diese in der Szene immer vorkommen. Die Buttons für die linke und rechte Seite werden in Listen gespeichert, also nicht in Unity festgelegt. Um mehrere Buttons per Skript zu spawnen, werden nach den Listen Objekte mit einem Prefab eines Buttons und den beiden bereichen hier \textit{spawnerLeft} und \textit{spawnerRight} deklariert. Diese drei sind public, da sie in Unity den Objekten zugewiesen werden. Das Objekt \textit{spawnedObject} wird hier deklariert und kann private bleiben, da es nur im Skript verwenden wird. Der Gebrauch wird in den unteren Funktionen erklärt.\\
Da die Buttons für die Alternativen und die Schwierigkeit unterschiedlich aussehen, werden zunächst drei \textit{Sprite Arrays} erzeugt, in welche Bilder der vergefertigten Buttons gezogen werden. So kann auf die Bilder in der richtigen Reihnfolge später zugegriffen und anschließend können diese verwendet werden.\\
Als nächstes werden static Variablen benötigt mit welchen die Werte gespeichert werden mit denen das Level initialisiert wird. Die \textit{maxNumberStatic} speichert die Zahl ein die später maximal zufällig generiert werden kann. Je nach Schwierigkeit ist diese größer oder kleiner. Die \textit{lvlAmmountStatic} legt fest, wie viele Level das Kind spielen muss bevor er dieses abgeschlossen hat. Die \textit{fourChoices Variable} speichert die linke Auswahl ein. Der Name ist ein wenig irreführend, da es nur einspeichern soll welcher Button links ausgewählt wurde. Das sind die Möglichkeiten im Spiel 'größer, kleiner, gleich' oder bei den 'Grundrechenarten'. Also speichert diese die Zahlen von eins bis vier ein. Die \textit{wallSize} ist nur für die Rechenmauer notwendig, je nach der Schwierigkeit, ist die Mauer unten drei, vier oder fünf Steine breit.\\
Als letztes werden noch vier private Variablen benötigt. Der Variable \textit{loadButtonsNumber} wird die static Variable aus der vorherigen Szene zugewiesen. Die beiden Boolean Variablen werden benötigt, um die linke und rechte Seite der Buttons zu überprüfen. Da die Buttons, wenn sie in einem Canvas erzeugt werden, kleinerskaliert werden müssen diese mithilfe des Vektors wieder auf die Originalgröße skaliert werden.\\
\begin{lstlisting}[language=csh, caption={MenuPickLevelAdvanced.cs Start-Funktion}]
	void Start()
	{
		loadButtonsNumber = MenuPickLevel.loadButtonsPrefab;
		back.onClick.AddListener(() => GoBack());
		start.onClick.AddListener(() => LoadGame());
		SpawnButtonsRight();
		if(loadButtonsNumber == 2 || loadButtonsNumber == 3){
			SpawnButtonsLeft();
		}
		else{
			leftSide = true;
		}
	}
\end{lstlisting}
Die Start-Funktion wird sobald die Szene geladen. Diese speichert den Wert der static Variablen aus dem vorherigen Skript in eine Variable. Da die Variable public war, kann mit dem Namen des Skriptes und dem Namen der Variablen auf diese zugegriffen werden.
Danach bekommen die Buttons, um zurückzugehen und um das Level zu starten, ihre Funktion zugewiesen. Da die Buttons auf der rechten seite immer gleich sind, werden diese als nächstes erzeugt, dies wird in der Funktion \textit{SpawnButtonsRight} erledigt. Danach muss für die linke Seite überprüft werden, ob es überhaupt Alternativen gibt, wenn dies nicht der Fall wäre, würde der Boolean für die linke Seite auf \textit{true} gesetzt werden. Um zu überprüfen welche Alterantiven benötigt werden, wird die \textit{loadButtonsNumber} überprüft. Wenn diese den Wert zwei oder drei zugewiesen bekam, entspricht dies dem Spiel  größer, kleiner, gleich oder Grundrechenarten. Wenn eine dieser beiden Möglichkeiten besteht, werden die Buttons auf der linken Seite erzeugt.\\
\begin{lstlisting}[language=csh, caption={MenuPickLevelAdvanced.cs Update-Funktion}]
	void Update(){
		if(!leftSide || !rightSide){
			start.interactable = false;
		}
		if(leftSide && rightSide){
			start.interactable = true;
		}
	}
\end{lstlisting}
Die Update-Funktion von Unity wird jeden Frame aufgerufen und überprüft ob die Booleans für die linke und rechte Seite \textit{true} sind oder ob eine Seite noch nicht ausgewählt wurde. Wenn auf beiden Seiten etwas ausgewählt wurde, kann das Spiel gestartet werden. Wenn eine der beiden Variablen \textit{false} ist, ist der Start-Button deaktiviert. Um einen Button zu aktivieren oder zu deaktivieren wird die Variable \textit{interactable} des Buttons auf \textit{false} oder \textit{true} gesetzt. Je nachdem ist der Button im Spiel dann ausgegraut oder eben nicht.\\
\begin{lstlisting}[language=csh, caption={MenuPickLevelAdvanced.cs GoBack-Funktion}]
	public void GoBack(){
		MenuPickLevel.loadButtonsPrefab = 0;
		SceneManager.LoadScene("MenuLearning");
	}
\end{lstlisting}
Die GoBack-Funktion wird aufgerufen, wenn der Zurück-Button gedrückt wurde. In dieser wird die static Variable aus dem vorherigen Skript wieder auf null gesetzt, damit es zu keinem Fehler in der Levelauswahl kommt, da diese sonst den vorherigen Wert weiterhin besitzen würde. Danach wird mithilfe des Aufrufes \textit{SceneManager.LoadScene(''MenuLearning'')} die Menüauswahl-Szene aufgerufen.
\begin{lstlisting}[language=csh, caption={MenuPickLevelAdvanced.cs LoadGame-Funktion}]
	public void LoadGame(){
		switch (loadButtonsNumber)
		{
			case 2:
				SceneManager.LoadScene("CompareQuantities");
				break;
			case 3:
				SceneManager.LoadScene("basicOperations");
				break;
			case 4:
				SceneManager.LoadScene("triangle");
				break;
			case 5:
				SceneManager.LoadScene("lightningView");
				break;
			case 6:
				SceneManager.LoadScene("finishPattern");
				break;
			case 7:
				SceneManager.LoadScene("MixedWords");
				break;
			default:
				break;
		}
	}
\end{lstlisting}
Die Funktion \textit{LoadGame} kontrolliert mithilfe eines \textit{switch-case}-statements, welcher Button in der Szene davor gedrückt wurde, um anschließend die passende Szene zu laden\\
Das Problem, was nach der Verbesserung des ersten Skripts aufgetreten ist, war dass die Szene mit den Grundrechenarten die Zahl vier zugewiesen bekommen sollte, diese war aber der dritte Fall im switch-case-statement und somit wurde in dieser Szene immer die Rechenmauer geladen. Die Zahlen wurden dann im Skript davor angepasst und die Grundrechenarten bekamen die nummer drei zugewiesen. Dies führte in der Funktion \textit{SpawnButtonsLeft} zu Problemen.\\
\begin{lstlisting}[language=csh, caption={MenuPickLevelAdvanced.cs-SafeOptionsRight Funktion}]
	public void SafeOptionsRight(int whichButton){
		for (int i = 0; i < rightSideList.Count; i++){
			if(i != whichButton){
				rightSideList[i].interactable = true;
				continue;
			}
			rightSide = true;
			rightSideList[i].interactable = false;
		}
		switch (whichButton)
		{
			case 0:
				lvlAmmountStatic = 10;
				maxNumberStatic = 10;
				wallSize = 3;
				break;
			case 1:
				lvlAmmountStatic = 15;
				maxNumberStatic = 15;
				wallSize = 4;
				break;
			case 2:
				lvlAmmountStatic = 20;
				maxNumberStatic = 20;
				wallSize = 5;
				break;
			default:
				break;
		}
	}
\end{lstlisting}
In dieser Funktion wird die Auswahl der Schwierigkeit des Levels gespeichert. Diese bekommt eine Zahl zwischen null und zwei zugewiesen. In einer \textit{For}-Schleife wird dann überprüft welcher Button, gedrückt wurde. Wenn ein Button der ungleich i gedrückt wurde, wird dieser aktiviert und die Schleife springt einen Durchgang weiter. Wenn i die gleiche Zahl wie die Variable \textit{whichButton} ist, wird der Boolean auf \textit{true} gesetzt und der aktuell gedrückte Button wird deaktiviert.\\
Nachdem die Schleife durchgelaufen ist, wird mithilfe eines \textit{switch-case}-statements überprüfut welcher der drei Buttons ausgewählt wurde und je nachdem eine Levelanzahl, eine maximale random Nummer und die Mauergröße gespeichert.\\
\begin{lstlisting}[language=csh, caption={MenuPickLevelAdvanced.cs SafeOptionsLeft-Funktion}]
	public void SafeOptionsLeft(int whichButton){
		for (int i = 0; i < leftSideList.Count; i++){
			if(i != whichButton){
				leftSideList[i].interactable = true;
				continue;
			}
			leftSide = true;
			leftSideList[i].interactable = false;
		}
		if(loadButtonsNumber == 2){
			setQuantitiesOptions(whichButton);
		}
		else if(loadButtonsNumber == 3){
			setMathOptions(whichButton);
		}

	}
\end{lstlisting}
Die Funktion SafeOptionsLeft Speichert welche der Alternativen links gespeichert wurde. Also wird einmal wie in der Funktion \textit{SafeOptionsRight} mithilfe einer \{For}-Schleife kontrolliert welcher Button gedrückt wurde und dieser wird dann deaktiviert. Die Boolean Variable für links wird dann auch auf \textit{true} gesetzt. Da es in zwei Leveln Alternativen gibt, wird überprüft welches Level gewählt wurde. Im Fall der Zahl Zwei wird die Funktion aufgerufen, die Speichern soll ob die Kinder Objekte oder Zahlen vergleichen sollen. Im Fall der Drei wird die Funktion aufgerufen, die speichert welche der vier Grundrechenarten das Kind spielen möchte. 
\begin{lstlisting}[language=csh, caption={MenuPickLevelAdvanced.cs setQuantitiesOptions-Funktion}]
	public void setQuantitiesOptions(int whichButton){
		switch (whichButton)
		{
			case 0:
				fourChoices = 1;
				break;
			case 1:
				fourChoices = 2;
				break;
			default:
				break;
		}
	}
\end{lstlisting}
Diese Funktion überprüft mithilfe eines \textit{switch-case}-statements welche der beiden Varianten ausgewählt wurde und speichert diese in die Variable \textit{fourChoices}. Durch diese weiß das Spiel 'größer, kleiner, gleich' ob es Zahlen oder Objekte anzeigen soll.
\begin{lstlisting}[language=csh, caption={MenuPickLevelAdvanced.cs setMathOptions-Funktion}]
	public void setMathOptions(int whichButton){
		switch (whichButton)
		{
			case 0:
				fourChoices = 1;
				break;
			case 1:
				fourChoices = 2;
				break;
			case 2:
				fourChoices = 3;
				break;
			case 3:
				fourChoices = 4;
				break;
			default:
				break;
		}
	}
\end{lstlisting}
Dieses Skript arbeitet gleich wie das Skript davor. Es überprüft welche der vier Grundrechenarten ausgewählt werden sollte und speichert diese ab, damit das Skript der Grundrechenarten weiß ob das Kind addieren, subtrahieren, multiplizieren oder dividieren möchte.
\begin{lstlisting}[language=csh, caption={MenuPickLevelAdvanced.cs SpawnButtonsLeft-Funktion}]
	public void SpawnButtonsLeft(){
		int size = 0;
		if(loadButtonsNumber == 3) size = 4;
		else size = 2;
		for (int i = 0; i < size; i++){
			int copy = i;
			spawnedObject = Instantiate(buttonPref, spawnerLeft.transform.position, Quaternion.identity);
			spawnedObject.name += i;
			spawnedObject.transform.SetParent(spawnerLeft.transform);
			spawnedObject.transform.localScale = scaleSize;
			spawnedObject.GetComponent<Image>().preserveAspect = true;
			leftSideList.Add(spawnedObject.GetComponent<Button>());
			leftSideList[i].onClick.AddListener(() => SafeOptionsLeft(copy));
			if(size == 2){
				leftSideList[i].GetComponentInChildren<Image>().sprite = quantitiesModes[i];
				// leftSideList[i].GetComponentInChildren<Image>().preserveAspect = true;
				continue;
			}
			else if(size == 4){
				leftSideList[i].GetComponentInChildren<Image>().sprite = mathModes[i];
			}
		}
	}
\end{lstlisting}
In diesem Skript kam es zu einem Problem, nachdem das erste Menü Skript angepasst wurde. Dieses Skript läuft mit einer \textit{For}-Schleife durch und erzeugt je nachdem welche Zahl in der Variablen \textit{loadButtonsNumber} steht die richtige Anzahl an Buttons. Für das Spiel 'größer, kleiner, gleich' erzeugt es zwei Buttons und für die Grundrechenarten vier. Da aber die Grundrechenarten die Nummer Drei bekamen wurde ein Button zu wenig erzeugt. Um dies zu verhindern, wird am Anfang einmal überprüft ob die \textit{loadButtonsNumber} zu diesem Zeitpunkt eine drei beinhaltet und wenn ja wird die Größe auf vier gesetzt, wenn das nicht der Fall ist wird diese auf zwei gesetzt. Somit wird die richtige Anzahl an Buttons für die Alternativen erzeugt.\\
Die \textit{For}-Schleife muss auch hier wieder eine temporäre Variable erzeugen, hier wurde sie \textit{copy} genannt. Um Objekte durch ein Skript zu erzeugen, wird in das Objekt \textit{spawnedObject} mithilfe der Instantiate-Methode, dies ist eine Methode um ein Objekt zu spawnen der Button erzeugt. Die Instantiate-Methode bekommt ein \textit{GameObject} das sie spawnen soll, in diesem Fall den Button Prefab. Dann benötigt sie eine Position, das wäre die Position des Spawners auf der linken Seite. Diese bekommt man durch den Aufruf \textit{spawnerLeft.transform.position}. Dieser gibt die Position des \textit{GameObject} auf der linken Seite zurück und an dieser Stelle soll der Button erzeugt werden. Da sie keine Rotation benötigt, wird als letzer Übergabewert \textit{Quaternion.identity} verwendet.
Damit in Unity die Objekte in der Hierarchie unterscheidbar waren, bekamen sie an den Namen ein 'i' angehängt(spawnedObject0).\\
Um die Buttons richtig anzuordnen wurde ein Overlay mit vertikalen und horizontalen Layoutgroups angelegt. In der Abbildung \ref{layoutAdvancedMenu} ist dies zu sehen. Das Objekt \textit{Choices}  eine bekam eine horizontale Layoutgruppe, die Objekte \textit{leftSite} und \textit{rightSite} bekamen eine vertikale Layoutgruppe. 
%TODO Maybe noch ein Bild
Damit die Buttons in diesen Objekten zu sehen sind, wird dem \textit{spawnedObjekt} ein Elternobjekt zugewiesen. Um dies zu erreichen, wird \textit{spawnedObject.transform.SetParent} aufgerufen. Danach wird das Objekt an der richtigen Stelle angezeigt, aber nicht in der richtigen Größe. Unity skaliert das Objekt kleiner, da es hierbei die Größe des Objekts nimmt in dem das neue Objekt erzeugt wird. Damit das Objekt die Originalgröße behält, wird für das erzeugte Objekt die Skalierung auf die originale Größe gesetzt. Auch dies ist mithilfe des \textit{.transfom.localScale} Befehls möglich.\\
Bilder in Unity werden unscharf dargestellt. Dies kann verhindert werden, wenn die Option \textit{preserveAspect} auf \textit{true} gesetzt wird. Dies wird auch für jedes erzeugte Objekt in dieser Funktion erledigt.\\
Nachdem alle Buttons für die linke Seite erzeugt wurden, werden diese in die Liste \textit{LeftSideList} gespeichert und bekommen danach den Listener zugewiesen. Dieser ruft dann die Funktion auf, die speichert welche Alterantive des Spiels gespeichert werden soll.\\
Nachdem die Buttons links erzeugt wurden, muss die Funktion in einer \textit{If} und \textit{else if}-Abfrage überprüfen welche Bilder den Buttons zugewiesen werden sollen, sonst würde auf den Buttons nichts stehen. Wenn die Variable \textit{size} die Zahl Zwei beinhaltet, wird auf den Button Links in den Komponenten Image das passende Bild aus dem Array gespeichert. Der Befehl \textit{leftSideList[i].GetComponentInChildren<Image>().sprite = quantitiesModes[i];} nimmt die Komponeten des Buttons aus der Liste \textit{leftSideListe} und fügt das Bild aus dem Array an der Stelle i ein.\\
Für die Grunrechenarten wird dasselbe angewandt, in dem fall ist die Variable gleich vier ist.
\begin{figure}[htbp]
  \centering
  \includegraphics[width=0.65\textwidth,height=0.55\textheight,keepaspectratio]{images/LayoutMenuAdvanced.PNG}
  \caption{Layout Erweitertes-Menü}
  \label{layoutAdvancedMenu}
\end{figure}
\begin{lstlisting}[language=csh, caption={MenuPickLevelAdvanced.cs SpawnButtonsRight-Funktion}]
	public void SpawnButtonsRight(){
		for (int i = 0; i < 3; i++){
			int copy = i;
			spawnedObject = Instantiate(buttonPref, spawnerRight.transform.position, Quaternion.identity);
			spawnedObject.name += i;
			spawnedObject.transform.SetParent(spawnerRight.transform);
			spawnedObject.transform.localScale = scaleSize;
			spawnedObject.GetComponent<Image>().preserveAspect = true;
			rightSideList.Add(spawnedObject.GetComponent<Button>());
			rightSideList[i].onClick.AddListener(() => SafeOptionsRight(copy));
			rightSideList[i].GetComponentInChildren<Image>().sprite = difficulty[i];
			// rightSideList[i].GetComponentInChildren<Image>().preserveAspect = true;
		}
	}
\end{lstlisting}
Die Funktion \textit{SpawnButtonsRight} arbeitet gleich wie die SpawnButtonsLeft-Funktion, nur dass in diesem Fall immer drei Buttons gespawnt, auf die richtige größe skaliert und in eine Liste eingefügt werden. Diesen werden danach dann die Bilder der unterschiedlichen Schwierkeitsgrade hinzugefügt.
\subsection{Szene - Grundrechenarten}
\subsubsection{Grundrechenarten - Design}
\begin{figure}[htbp]
  \centering
  \includegraphics[width=0.65\textwidth,height=0.55\textheight,keepaspectratio]{images/addition.jpg}
  \caption{Grundrechenarten - Addition}
  \label{basicOperation}
\end{figure}
In der Szene der Grundrechenarten gibt es ein Textfeld, welches mit einem grünen Pfeil markiert wurde. In diesem steht der Operator, also ob das Kind addieren, subtrahieren, multiplizieren oder dividieren soll. Die Szene kann mit dem Menü-Button oben links verlassen werden. Die Überschrift der Szene sagt dem Kind, dass es den Term ausrechnen soll. Ganz rechts oben steht das aktuelle Level und wie viele es insgesamt gibt. Dies wird, sobald das Kind eine Rechnung richtig hat, um eins hochgezählt. In der Mitte der Szene steht links die erste Zahl, dann der Operator, dann die zweite Zahl. Danach kommt das Gleichheitszeichen und ein Eingabefeld für die Lösung des Nutzers. Dieses Feld wird grün umrandet, wenn die Lösung richtig ist und rot, wenn diese falsch ist. Nach einer falschen Eingabe wird das Eingabefeld wieder geleert. In der Abbildung \ref{greenBorder} und \ref{redBorder} ist der grüne und rote Rand zu sehen. Dieser wird in diesem Fall aus der Szene 'größer, kleiner, gleich' sein, ist aber in den Szenen, in denen er verwendet wird, gleich. 
\begin{figure}[htbp]
  \centering
  \includegraphics[width=0.5\textwidth,height=0.30\textheight,keepaspectratio]{images/rightChoiceQuantities.PNG}
  \caption{Grüner Rahmen}
  \label{greenBorder}
\end{figure}
\begin{figure}[htbp]
  \centering
  \includegraphics[width=0.5\textwidth,height=0.30\textheight,keepaspectratio]{images/wrongChoiceQuantites.PNG}
  \caption{Roter Rahmen}
  \label{redBorder}
\end{figure}
Der Button \textit{Testen} löst die Funktion aus, die kontrolliert, ob richtig gerechnet wurde. Dieser ist deaktiviert, solange keine Zahl eingeben wurde und wird aktiviert, sobald eine Zahl darin steht.
\subsubsection{Grundrechenarten - Skript}
\begin{lstlisting}[language=csh, caption={MathOperations.cs Variablen-Deklaration}]
public class MathOperations : MonoBehaviour
{

	public Text TextNumberLeft;
	public Text TextNumberRight;
	public Text operatorSymbole;
	public Text showLevelNumber;

	public InputField solutionNumber;

	public Button checkButton;
	public Button menu;

	public Image imageColor;

	private int leftNumber;
	private int rightNumber;
	private int solution;
	private int maxRandomNumber = 10;
	private int minRandomNumber = 0;
	private int lvlNumber = 10;
	private int countLvl = 0;
	//add = 1, sub = 2, mult = 3, div = 4
	private int gameMode = 3;
	private int wrongSolution = 0;
\end{lstlisting}
In dem Skript für die Szene werden zuerst vier public Textfelder deklariert. Diese zeigen verschiedene Informationen an. Eines zeigt das aktuelle Level und wie viele es insgesamt gibt. Dann zeigt eines der Textfelder den Operator an, mit dem das Kind rechnen soll. Die letzten zwei Textfelder zeigen die Zahlen an, die addiert, subtrahiert, multipliziert oder dividiert werden sollen. In dieser Szene benötigt der Schüler/die Schülerin ein Eingabefeld, in das seine/ihre Lösung geschrieben werden kann. Das Skript bekommt auch zwei Buttons zugewiesen, einen der in das erste Menü zurückführt und einen, der das Ergebnis überprüft. Um den Kontrollrahmen zu erzeugen, wird ein Objekt vom Typ Image benötigt, dieses liegt hinter dem Eingabefeld und ist ein wenig größer als das Input-Feld selbst. Als Letztes benötigt das Skript noch ein paar Variablen um die Zahlen links und rechts zu speichern. Das Skript benötigt eine Variable, die das Ergebnis der Rechnung speichert. Des Weiteren sind zwei Variablen deklariert, um die zufällig generierten Zahlen zu begrenzen. Es gibt eine Variabel, die die maximale Anzahl an Leveln festlegt und einen Zähler, der nach oben gezählt wird. Es gibt noch zwei letzte Variablen im Skript. Die eine legt den Modus fest, also ob addiert, subtrahiert, multipliziert oder dividiert wird und die andere zählt die Anzahl der Fehler, die das Kind macht.\\
\begin{lstlisting}[language=csh, caption={MathOperations.cs Start-Funktion}]
	void Start()
	{
		menu.onClick.AddListener(() => GoBack());
		lvlNumber = MenuPickLevelAdvanced.lvlAmmountStatic;
		gameMode = MenuPickLevelAdvanced.fourChoices;
		checkButton.onClick.AddListener(() => buttonClick());
		checkButton.interactable = false;
		solutionNumber.onValueChanged.AddListener(delegate {EnableButton(); });
		PlayGame(gameMode);
	}
\end{lstlisting}
In der Start-Funktion der Szene werden am Anfang ein Listener für den Menü-Button und einen für den Test-Button zugewiesen. In dieser Methode, werden die statischen Variablen, die in der Szene davor gesetzt wurden aufgerufen und in die passenden Variablen gespeichert.
So weiß das Skript nun, ob es \textit{Addieren, Subtrahieren, Multiplizieren oder Dividieren} laden soll. Der Button zum Testen wird am Anfang deaktiviert, da wenn das Eingabefeld leer ist und der Button gedrückt wird die Szene einfriert. Damit dies verhindert wird, wird der Button beim Start der Szene deaktiviert. Wenn das Kind eine Zahl in das Eingabefeld eingibt, wird der Button aktiviert. Dies ist mit der onValueChange Methode für das Eingabefeld möglich. Diese bekommt einen Listener, der die Funktion \textit{EnableButton} aufruft und dann den Button aktiviert oder deaktiviert. Als Letztes ruft die Start-Funktion die Funktion \textit{PlayGame} auf. Diese wird mit dem Übergabeparameter des Gamemodes aufgerufen.\\
\begin{lstlisting}[language=csh, caption={MathOperations.cs EnableButton-Funktion}]
	public void EnableButton(){
		if(string.IsNullOrEmpty(solutionNumber.text)){
			checkButton.interactable = false;
			solutionNumber.interactable = true;
		}
		else{
			checkButton.interactable = true;
		}
	}
\end{lstlisting}
Die EnableButton-Funktion überprüft bei jeder Änderung des Wertes, ob das Input Feld leer ist oder nicht. Dies erfolgt mit der IsNullOrEmpty-Abfrage, da aber eine Integerzahl eingegeben wird, muss diese mit \textit{string.} in einen String umgewandelt werden. Wenn das Feld leer ist, wird der Button deaktiviert und das Eingabefeld aktiviert, sodass der Schüler/die Schülerin eine Eingabe tätigen kann. Wenn das Feld gefüllt ist, wird der Testbutton aktiviert. Damit in den jeweiligen Szenen keine falsche Eingabe gemacht werden kann, also bei den mathematischen Spielen kein Text und bei dem Deutsch-Spiel keine Zahl eingegeben werden kann, wird in Unity die Eingabe auf den jeweiligen Wert festgelegt. Dies ist in Abbildung \ref{inputField} zu sehen.
\begin{figure}[htbp]
  \centering
  \includegraphics[width=0.65\textwidth,height=0.55\textheight,keepaspectratio]{images/inputType.PNG}
  \caption{Festlegen welcher Wert in das Input-Feld geschrieben werden darf}
  \label{inputField}
\end{figure}
Die Erklärung des Codes der Funktion, die in das Menü zurückkehrt, wird nicht eingefügt, da etwas in dieser Art schon erklärt wurde. Dieser setzt die statischen Variablen auf den Ursprungswert zurück und lädt die Menüszene.\\
\begin{lstlisting}[language=csh, caption={MathOperations.cs PlayGame-Funktion}]
public void PlayGame(int gameMode){
		countLvl++;
		if(countLvl > lvlNumber){
			PlayerPrefs.SetInt("wrongAnswers", wrongSolution);
			SceneManager.LoadScene("LearnFinishScreen");
			Debug.Log("Game Vorbei \n" + "Anzahl Fehler: " + wrongSolution);
		}
		//change color to white
		imageColor.color = new Color32(255, 255, 255, 255);

		if(countLvl <= lvlNumber)	showLevelNumber.text = "Level: " + countLvl + "/" + lvlNumber;
		solutionNumber.text = "";
		if(gameMode == 4){
			operatorSymbole.text = "/";
			minRandomNumber = 1;
			leftNumber = RandomNumbers(minRandomNumber, maxRandomNumber);
			rightNumber = RandomNumbers(minRandomNumber, maxRandomNumber);
			solution = leftNumber * rightNumber;
			SetNumbersLeftAndRight(solution, rightNumber);
			return;
		}
		else if(gameMode == 2){
			minRandomNumber = 0;
			leftNumber = RandomNumbers(minRandomNumber, maxRandomNumber);
			rightNumber = RandomNumbers(minRandomNumber, maxRandomNumber);
			if(leftNumber < rightNumber){
				Debug.Log(" leftNumber: " + leftNumber + " rightNumber: " + rightNumber + "zweite If");
				SetNumbersLeftAndRight(rightNumber, leftNumber);
				solution = rightNumber - leftNumber;
			}
			else{
				Debug.Log(" leftNumber: " + leftNumber + " rightNumber: " + rightNumber + "zweite If");
				SetNumbersLeftAndRight(leftNumber, rightNumber);
				solution = leftNumber - rightNumber;
			}
			operatorSymbole.text = "-";
			return;
		}
		else{
			minRandomNumber = 0;
			leftNumber = RandomNumbers(minRandomNumber, maxRandomNumber);
			rightNumber = RandomNumbers(minRandomNumber, maxRandomNumber);
			SetNumbersLeftAndRight(leftNumber, rightNumber);
			if(gameMode == 1){
				operatorSymbole.text = "+";
				solution = leftNumber + rightNumber;
			}
			else{
				operatorSymbole.text = "*";
				solution = leftNumber * rightNumber;
			}
			return;
		}
	}
\end{lstlisting}
Die PlayGame-Funktion, zählt den Zähler für die Anzahl der Level bei jedem Aufruf um eins nach oben. Das Skript testet dann, ob der Counter über der Anzahl zu spielenden Level ist und wenn dies der Fall ist, wird in einer Datei die Anzahl an Fehlern gespeichert und die Szene, die den \textit{finishScreen} lädt, geladen. In diesem Fall wird die Anzahl an Fehlern nicht in einer statischen Variable gespeichert, da diese im Skript der End Szene deklariert werden müsste. Die statische Variable muss dann nach jedem aufruf der Szene \textit{finishScreen} wieder auf null gesetzt werden. Durch das speichern in einer Datei, wird diese einfach mit dem neuen wert überspeichert.\\
Das Skript muss den Rahmen des Eingabefeldes beim Aufruf der PlayGame-Funktion wieder auf die Farbe weiß setzen. Das Skript setzt die Anzahl an Leveln in das passende Textfeld, solange der Zähler nicht größer als die maximale Anzahl an Leveln ist. Bei jedem Aufruf der Funktion wird das Eingabefeld geleert, sodass keine Zahl in dem Feld steht.\\
Für jede Art der Grundrechenarten werden andere Vorbereitungen benötigt, um eine korrekte Rechnung zu erstellen. Wenn das Kind dividieren soll, muss das Operator-Textfeld das Symbol '/' anzeigen. Damit eine Division entsteht, die mathematisch korrekt ist, wird die kleinste Zahl, die zufällig generiert werden kann, auf eins gesetzt. Danach werden zwei Zahlen für das linke und rechte Textfeld generiert und die Lösungsvariable \textit{solution} wird druch eine Multiplikation berechnet. Der Grund dafür ist, dass wenn eine Zahl durch eine Multiplikation erzeugt wird und diese durch eine der beiden zufälligen Zahlen dividiert wird, auf jeden Fall keine Dezimalzahl als Ergebnis herauskommt. Daher wird das Ergebnis der Rechnung in das linke Feld und die rechten Zahlen in das rechte Feld geschrieben. Das Ergebnis ist in diesem Fall dann immer die Zahl, die für die linke Seite erzeugt wurde.\\
Im Falle einer Subtraktion werden wieder zufällige Zahlen erzeugt, dieses Mal ist die kleinstmögliche Zahl allerdings eine Null. Damit das Ergebnis keine negative Zahl werden kann,, wird überprüft welche Zahl die größere ist. Diese wird in die linke Seite geschrieben und die kleinere in die rechte Seite. Das Ergebnis wird dann entweder \textit{linke Zahl - rechte Zahl} oder \textit{rechte Zahl - linke Zahl} berechnet.\\
Für den Fall der Addition und Multiplikation, können beide Zahlen unbesorgt für die linke und rechte Seite erzeugt werden. Hierbei sollte überprüft werden, ob im Feld für den Operator ein Plus oder ein Asterisk(Stern) zu sehen ist und wie die Lösung berechnet werden muss.\\
\begin{lstlisting}[language=csh, caption={MathOperations.cs RandomNumber-Funktion}]
	public int RandomNumbers(int minRandomNumber, int maxRandomNumber){
		return UnityEngine.Random.Range(minRandomNumber, maxRandomNumber);
	}
\end{lstlisting}
In dieser Funktion wird eine zufällige Zahl mithilfe der von Unity bereitgestellten Funktion generiert. Die Funktion erzeugt eine zufällige Zahl in der Reichweite einer kleinstmöglichen und einer maximalen Zahl.\\
\begin{lstlisting}[language=csh, caption={MathOperations.cs SetTextLeftAndRight-Funktion}]
	public void SetNumbersLeftAndRight(int left, int right){
			TextNumberLeft.text = left.ToString();
			TextNumberRight.text = right.ToString();
	}
\end{lstlisting}
Diese Funktion setzt die zwei übergebenen Zahlen in die linke und rechte Seite der Textfelder. Indem von dem Objekt \textit{Textfield} die Komponente \textit{.text} aufgerufen wird, kann in diese die Integerzahl, nachdem sie in einen String umgewandelt wurde, einspeichert werden.\\
\begin{lstlisting}[language=csh, caption={MathOperations.cs ButtonClick-Funktion}]
	public void buttonClick(){
		checkButton.interactable = false;
		if(gameMode == 4){
			StartCoroutine(waiterDiv(1));
		}
		else{
			StartCoroutine(waiter(1));
		}
	}
\end{lstlisting}
Wenn der Test-Button gedrückt wurde, wird je nach Game Mode eine \textit{Coroutine} aufgerufen. Diese wird verwendet, um zwischen den Leveln eine Sekunde zu warten und dem Kind zu zeigen, dass das Ergebnis richtig oder falsch ist. Da die Division anders überprüft werden muss, als die anderen drei Rechenarten, bekam diese eine eigene \textit{Coroutine}.\\
\begin{lstlisting}[language=csh, caption={MathOperations.cs waiter-Coroutine}]
	IEnumerator waiter(int sec){
		solutionNumber.interactable = false;
		if(countLvl <= lvlNumber){
			if(int.Parse(solutionNumber.text) == solution){
				//change color green
				imageColor.color = new Color32(37, 250, 53, 255);
				yield return new WaitForSeconds(sec);
				PlayGame(gameMode);
			}
			else{
				wrongSolution++;
				//change color red
				imageColor.color = new Color32(251, 37, 37, 255);
				yield return new WaitForSeconds(sec);
				//change color to white
				imageColor.color = new Color32(255, 255, 255, 255);
				solutionNumber.text = "";
			}
		}
	}
\end{lstlisting}
Dies ist die \textit{Coroutine} für alle Grundrechenarten außer der Division. Die beiden Funktionen sind sich aber sehr ähnlich, daher wird nur eine erklärt. Als Erstes wird das Eingabefeld deaktiviert, da schnell schreibende Kinder während die Lösung angezeigt, dass das Ergebnis falsch ist. Schon die richtige Zahl eingeben können ohne zu warten bis dieser Ablauf einmal durchgelaufen ist. Danach wird überprüft, ob das Kind überhaupt noch Level zu spielen hat. Wenn dies der Fall ist, wird das Ergebnis mit der Eingabe verglichen. Wenn diese gleich ist, wird der Rahmen des Eingabefeldes für eine Sekunde grün, danach wird das nächste Level geladen. Wenn das Kind falsch gerechnet hat, wird der Zähler für die Fehler hochgezählt und die Farbe des Rahmens wird für eine Sekunde rot, danach wird dieser wieder weiß gefärbt. Die Farbe eines Objektes kann geändert werden, indem von dem Objekt die Komponente \textit{.color} aufgerufen wird und es eine neue Farbe zugewiesen bekommt, wie zum Beispiel \textit{new Color32(255, 255, 255, 255);} für die weiße Farbe. Das Eingabefeld wird geleert, aber keine neue Rechnung geladen.\\
Der Unterschied zur Überprüfung der Division ist, in der Abfrage, anstatt zu testen ob die Eingabe das gleiche ist wie die Lösung, wird getestet, ob die Eingabe das gleiche wie die Zahl, die für das linke Feld generiert wurde ist.
\subsection{Szene - Mengen vergleichen}
\subsubsection{Mengen vergleichen - Design}
\begin{figure}[htbp]
  \centering
  \includegraphics[width=0.65\textwidth,height=0.55\textheight,keepaspectratio]{images/quantitiesObjects.PNG}
  \caption{größer, kleiner, gleich}
  \label{quantities}
\end{figure}
In dieser Szene gibt es zwei Möglichkeiten das Spiel zu spielen. Entweder werden in dem großen Feldern links und rechts mittig Zahlen oder mithilfe einer \textit{Layout Group} Objekte in vier Reihen eingefügt. Der Header der Szene beinhaltet einen Menü-Bbutton, eine kleine Aufgabenstellung und die Level-Anzeige. In der Mitte der Szene befindet sich ein leeres Viereck, welches anzeigt,welcher der unteren Buttons gedrückt wurde und ob die Lösung richtig oder falsch ist. Dies ist wieder mit einem grünen oder roten Rahmen umgesetzt. Ganz unten in der Szene gibt es drei verschiedene Buttons für die jeweilige Option 'größer', 'kleiner' oder 'gleich'. Die Buttons werden nach dem Auswählen einer Option deaktiviert, bis das nächste Level geladen hat oder bis das Kind erneut versuchen kann, die richtig Lösung zu bestimmten. Die Buttons mussten deaktiviert werden, da bei der richtigen Lösung auf den zutreffenden Button zehnmal gedrückt werden könnte und somit das Spiel nach einer Runde beendet wäre.\\
\subsubsection{Mengen vergleichen - Skript}
\begin{lstlisting}[language=csh, caption={GameQuantities.cs Variablen-Deklaration}]
public class GameQuantities : MonoBehaviour {
	//the Textfields I fill
	public Text TextfieldLeft;
	public Text TextfieldRight;
	public Text TextfieldLevel;
	public Text Symbol;

	//collider for Objects
	public Collider2D spawnerLeftCollider;
	public Collider2D spawnerRightCollider;

	//my Gameobjects to compare Quantities
	public GameObject objectLeft;
	public GameObject objectRight;
	public GameObject spawnerLeft;
	public GameObject spawnerRight;
	private GameObject spawnedObject;

	//array with gameObjects
	private List<GameObject> alleQuantitiesObjects = new List<GameObject>(40);

	public Image imageColor;

	//Buttons
	public Button lessBtn;
	public Button equalsBtn;
	public Button greaterBtn;
	public Button menu;

	//variables for compare Quantities
	private int numberLeft = 0;
	private int numberRight = 0;
	private int randomNumber = 0;
	private int counterRound = 0;
	private int counterWrongChoice = 0;
	private Vector3 scaleSize = new Vector3 (1.0f, 1.0f, 1.0f);

	//number of lvls
	private int lvlNumber = 10;
	private int maxNumber = 0;

	//1 für mit objekten, 2 für mit zahlen
	private int lvlChoice = 1;
\end{lstlisting}
In dieser Szenen werden viele Variablen benötigt, diese einzeln zu erklären dauert zu lange. Die Szene benötigt, zwei Textfelder für die Zahlen links und rechts, in diesen Vierecken existieren auch zwei leere Felder, die als Spawner für die Objekte dienen. Ein Textfeld dient dem Zeichen des Vergleiches, also 'größer', 'kleiner', 'gleich'. In diese wird eine dafür vorhergesehene Variable geschrieben. Das Skript benötigt vier Buttons und eine Liste für die erzeugten Objekte. Um die Zahlen, Objekte und den Spielmodus zu vergleichen, werden Variablen deklariert, die in den einzelnen Funktionen verwendet werden. Da auch in dieser Szene die erzeugten Objekte wieder zu klein erzeugt werden, wird ein weiterer Vektor benötigt, der diese auf die Originalgröße zurückskaliert.\\
\begin{lstlisting}[language=csh, caption={GameQuantities.cs Start-Funktion}]
	void Start() {
		menu.onClick.AddListener(() => GoBack());
		maxNumber = MenuPickLevelAdvanced.maxNumberStatic;
		lvlNumber = MenuPickLevelAdvanced.lvlAmmountStatic;
		lvlChoice = MenuPickLevelAdvanced.fourChoices;
		lessBtn.onClick.AddListener(() => SetSymbol('<'));
		equalsBtn.onClick.AddListener(() => SetSymbol('='));
		Debug.Log("Lvl Ammount: " + lvlNumber);
		greaterBtn.onClick.AddListener(() => SetSymbol('>'));
		PlayGame();
	}
\end{lstlisting}
Die Start Funktion erzeugt wieder für den jeweiligen Button einen Listener. Für die Buttons, mit der Auswahl der drei Zeichen, wird eine Funktion aufgerufen, die das aktuelle Symbol in das Textfeld in der Mitte setzt. Die statischen Variablen aus dem Menü davor werden wieder für die Anzahl an Leveln verwendet und um zu kontrollieren, ob Zahlen oder Objekte erzeugt werden.\\

Die Funktion für das zurückkehren in das Menü wurde in den vorherigen Kapiteln schon erklärt.\\

\begin{lstlisting}[language=csh, caption={GameQuantities.cs SetSymbol-Funktion}]
	public void SetSymbol(char symbol){
		lessBtn.interactable = false;
		equalsBtn.interactable = false;
		greaterBtn.interactable = false;
		Symbol.text = symbol.ToString();
		StartCoroutine(waiter(1, symbol));
	}
\end{lstlisting}
Die Funktion bekommt das Zeichen '<', '=' oder '>' übergeben. Sobald einer der drei Buttons gedrückt wurde, werden alle drei Buttons deaktiviert, das Textfeld gesetzt und die \textit{Coroutine} zur Überprüfung des Ergebnisses aufgerufen.\\
\begin{lstlisting}[language=csh, caption={GameQuantities.cs PlayGame-Funktion}]
	public void PlayGame() {
		counterRound++;
		if(counterRound > lvlNumber){
			PlayerPrefs.SetInt("wrongAnswers", counterWrongChoice);
			SceneManager.LoadScene("LearnFinishScreen");
			Debug.Log("Game Vorbei \n" + "Anzahl Fehler: " + counterWrongChoice);
		}
		Debug.Log("Round: " + counterRound);
		lessBtn.interactable = true;
		equalsBtn.interactable = true;
		greaterBtn.interactable = true;
		//change color to white
		imageColor.color = new Color32(255, 255, 255, 255);
		if(counterRound <= lvlNumber)	SetLevelText();
		Symbol.text = "";
		if (lvlChoice == 2) {
			numberLeft = GenerateNumber(maxNumber);
			numberRight = GenerateNumber(maxNumber);
			SetTextLeftAndRight(numberLeft, numberRight);
		} else if (lvlChoice == 1) {
			numberLeft = GenerateNumber(16);
			numberRight = GenerateNumber(16);
			SpawnQuantitiesObjects(numberLeft, spawnerLeft, objectLeft);
			SpawnQuantitiesObjects(numberRight, spawnerRight, objectRight);
		}
	}
\end{lstlisting}
Die PlayGame-Funktion zählt wieder für jeden Aufruf den Level-Zähler nach oben und überprüft, ob genug Level gespielt wurden. Wenn das nicht der Fall ist, wird die aktuelle Level-Zahl angezeigt und die Buttons werden wieder aktiviert. Die Funktion ändert die Rahmenfarbe auf weiß und leert das Feld mit dem Symbol wieder. Je nachdem welcher Modus gewählt wurde, werden die zwei zufällig generierten Zahlen als Zahl in die Felder geschrieben oder es werden maximal 16 Objekte für jede Seite erzeugt und angezeigt.\\
\begin{lstlisting}[language=csh, caption={GameQuantities.cs SpawnQuantitiesObjects-Funktion}]
	public void SpawnQuantitiesObjects(int number, GameObject spawner, GameObject side) {
		for (int i = 0; i < number; i++) {
			spawnedObject = Instantiate(side, spawner.transform.position, Quaternion.identity);
			spawnedObject.name += i;
			spawnedObject.transform.SetParent(spawner.transform.GetChild(i / 4));
			spawnedObject.transform.localScale = scaleSize;
			alleQuantitiesObjects.Add(spawnedObject);
		}
	}
\end{lstlisting}
Die Funktion bekommt als Übergabe eine Zahl, einen der beiden Bereiche, in denen die Objekte erzeugt werden sollen und das Objekt, das erzeugt werden soll, da dieses für links ein anderes als für rechts ist. Das Skript läuft eine \textit{for}-Schleife durch, die die gewünschte Zahl an Objekten erzeugen soll. Da in dieser Szene in jeder der Reihen nur vier Objekte passen, wird in der \textit{Layout-Group} an der Stelle $\frac{i}{4}$ das Objekt gesetzt, so wird nach vier Objekten die nächste Reihe verwendet. Am Schluss der Funktion werden die Objekte in eine Liste gespeichert.\\
\begin{lstlisting}[language=csh, caption={GameQuantities.cs Delete-Funktion}]
	public void DeleteObjects() {
		foreach (GameObject child in alleQuantitiesObjects) {
			Destroy(child);
		}
	}
\end{lstlisting}
Die Objekte, die in die Liste gespeichert werden, nach dem Lösen eines Levels wieder gelöscht. Die Funktion iteriert mit einer \textit{foreach}-Schleife durch alle Objekte der Liste und löscht diese aus der Szene. Somit werden nicht immer nur mehr Objekte erzeugt.\\

Die nächsten drei Funktionen des Skriptes würden nur wieder zeigen, wie die Zufallszahlen erzeugt werden und wie die Textfelder der Levelanzahl und die Zahlen angezeigt werden sollen.\\

\begin{lstlisting}[language=csh, caption={GameQuantities.cs waiter-Coroutine}]
IEnumerator waiter(int sec, char symbol){
		if(counterRound <= lvlNumber ){
			if(symbol.Equals('<')){
				Symbol.text = '<'.ToString();
				if (numberLeft < numberRight) {
					Debug.Log("Richtig");
					//change color green
					imageColor.color = new Color32(37, 250, 53, 255);
					yield return new WaitForSeconds(sec);
					DeleteObjects();
					PlayGame();
				} else {
					counterWrongChoice++;
					//change color red
					imageColor.color = new Color32(251, 37, 37, 255);
					yield return new WaitForSeconds(sec);
					EnableButtons();
					imageColor.color = new Color32(255, 255, 255, 255);
					Symbol.text = "";
					Debug.Log("Nicht richtig");
				}
			}
			if(symbol.Equals('=')){
				Symbol.text = '='.ToString();
				if (numberLeft == numberRight) {
					Debug.Log("Richtig");
					//change color green
					imageColor.color = new Color32(37, 250, 53, 255);
					yield return new WaitForSeconds(sec);
					DeleteObjects();
					PlayGame();
				} else {
					counterWrongChoice++;
					//change color red
					imageColor.color = new Color32(251, 37, 37, 255);
					yield return new WaitForSeconds(sec);
					EnableButtons();
					imageColor.color = new Color32(255, 255, 255, 255);
					Symbol.text = "";
					Debug.Log("Nicht richtig");
				}
			}
			if(symbol.Equals('>')){
				Symbol.text = '>'.ToString();
				if (numberLeft > numberRight) {
					Debug.Log("Richtig");
					//change color green
					imageColor.color = new Color32(37, 250, 53, 255);
					yield return new WaitForSeconds(sec);
					DeleteObjects();
					PlayGame();
				} else {
					counterWrongChoice++;
					//change color red
					imageColor.color = new Color32(251, 37, 37, 255);
					yield return new WaitForSeconds(sec);
					EnableButtons();
					imageColor.color = new Color32(255, 255, 255, 255);
					Symbol.text = "";
					Debug.Log("Nicht richtig");
				}
			}
		}
	}
\end{lstlisting}
Die \textit{Coroutine} ist dafür zuständig zu überprüfen, ob die Zahl im linken Feld kleiner, gleich oder größer der Zahl im rechten Feld ist. Die Funktion überprüft zuerst, welches Symbol ausgewählt wurde, indem es das eingespeicherte Symbol mit den drei Möglichkeiten abgleicht. Danach wird geprüft, ob das Symbol für die jeweiligen erzeugten Zahlen korrekt ist. Wenn dies der Fall ist, wird der Rahmen des Symbolfeldes grün und nach einer Sekunde weiß. Danach werden die Objekte gelöscht und eine neue Runde wird geladen. Wenn das Kind die falsche Wahl getroffen hat, wird das Feld rot und die Fehleranzahl um eins hochgezählt. Nach einer Sekunde werden die Buttons wieder aktiviert und die Farbe des Rahmens zurückgesetzt. Das Symbolfeld wird wieder geleert und das Kind kann erneut versuchen das Problem zu lösen.\\
\begin{lstlisting}[language=csh, caption={GameQuantities.cs EnableButtons-Funktion}]
	public void EnableButtons(){
		lessBtn.interactable = true;
		equalsBtn.interactable = true;
		greaterBtn.interactable = true;
	}
\end{lstlisting}
Diese Funktion aktiviert alle drei Buttons und wird aufgerufen nachdem das Kind den falschen Button ausgewählt hat.
\subsection{Szene - Rechenmauer}
\subsubsection{Rechenmauer - Design}
\begin{figure}[htbp]
  \centering
  \includegraphics[width=0.65\textwidth,height=0.55\textheight,keepaspectratio]{images/bottom5Tiles.PNG}
  \caption{Rechenmauer mit fünf Steinen unten}
  \label{triangle}
\end{figure}
Die Rechenmauer kann in drei verschiedenen Varianten gespielt werden. Je nach dem Schwierigkeitsgrad, wird diese mit fünf, vier oder drei Steinen in der untersten Reihe erzeugt. Im Header ist die Szene wie die anderen aufgebaut. In der Mitte wird die Mauer angezeigt, für welche ein Prefab eines Steines angelegt wurde, dieses wird dann mithilfe einer \textit{Layout Group} an die richtige Position gesetzt. Die Felder, die die unterste Reihe bilden, werden deaktiviert, damit der Nutzer diese nicht verändern kann. Wenn das Kind den Test Button drückt, werden die richtigen Felder auch deaktiviert, damit diese nicht wieder verfälscht werden können. Die falschen Steine werden geleert und das Kind muss die Lösung erneut eintragen. Dies ist in der Abbildung \ref{wrongAndRightStone} zu sehen.\\
\begin{figure}[htbp]
  \centering
  \includegraphics[width=0.65\textwidth,height=0.55\textheight,keepaspectratio]{images/rechenmauerRichtigUndFalsch.PNG}
  \caption{Richtig und falsch gelöste Steine}
  \label{wrongAndRightStone}
\end{figure}
Wenn der Schüler/die Schülerin die Rechenmauer komplett richtig gelöst hat, wird das anhand des Feuerwerks, das um den Test Button auftaucht angezeigt. Die Pyramide wird geleert und eine neue unterste Reihe wird erzeugt. Das Feuerwerk wird auch in dem Spiel 'Führe die Reihe fort' angewendet und ist in Abbildung \ref{firework} zu sehen.
\begin{figure}[htbp]
  \centering
  \includegraphics[width=0.65\textwidth,height=0.55\textheight,keepaspectratio]{images/finisheTriangle.PNG}
  \caption{Feuerwerk nach richtigem Lösen}
  \label{firework}
\end{figure}
\subsubsection{Rechenmauer - Skript}
\begin{lstlisting}[language=csh, caption={Triangle.cs Variablen-Deklaration}]
public class Triangle : MonoBehaviour
{

	public GameObject brick;

	//rows of the triangle
	public GameObject triangleGameObject;

	public Text lvlText;


	private GameObject spawnedObject;

	public GameObject leftFirework;
	public GameObject rightFirework;

	public GameObject fireworkPrefab;

	//3 = bottom 3 bricks, 4 = bottom 4 bricks, 5 = bottom 5 bricks
	private int triangleSize = 3;

	private int[][] jaggedSolution = new int[5][];

	private InputField[][] jaggedInputs = new InputField[5][];

	private int[] bottomNumbers;

	public Button testSolution;
	public Button menu;

	private int lvlCounter = 0;
	private int wrongChoice = 0;
	private int lvlNumber = 10;
	private int maxNumber = 0;
	private Vector3 scaleSize = new Vector3 (1.0f, 1.0f, 1.0f);
\end{lstlisting}
Im Skript der Rechenmauer benötigen wir eine Variable für das Prefab des einzelnen Steines. Die Rechenmauer wird in einer \textit{Layout Group} gespawnt, welche sich in einem Leeren Objekt befindet. Es gibt auch hier wieder ein Textfeld, um das aktuelle Level anzuzeigen und ein Objekt, in das die Blöcke erzeugt werden. Für das Feuerwerk werden auch ein Objekt und zwei Positionen benötigt. Für die einzelnen Reihen gibt es zweidimensionale Arrays, um die Lösung zu berechnen und um die Eingaben zu speichern. In der Szene gibt es zwei Buttons. Zum Abschluss benötigen wir noch private Variablen, die notwendig sind um das Level auszuführen. Diese wären ein Zähler für das Level, ein Zähler für die Fehler, die Anzahl an Leveln, die maximal größte Zahl und ein Vektor um die Steine zur Originalgröße zu skalieren.\\
\begin{lstlisting}[language=csh, caption={Triangle.cs Start-Funktion}]
	void Start()
	{
		menu.onClick.AddListener(() => GoBack());
		maxNumber = MenuPickLevelAdvanced.maxNumberStatic;
		lvlNumber = MenuPickLevelAdvanced.lvlAmmountStatic;
		triangleSize = MenuPickLevelAdvanced.wallSize;
		testSolution.interactable = false;
		testSolution.onClick.AddListener(() => CheckSolution());
		jaggedInputs[0] = new InputField[1];
		jaggedInputs[1] = new InputField[2];
		jaggedSolution[0] = new int[1];
		jaggedSolution[1] = new int[2];
		GenerateSolutionArray();
		SpawnPyramidRows();
		PlayGame();
	}
\end{lstlisting}
Die Start-Funktion weißt den Buttons den Listener zu und deaktiviert den Test-Button. Die Spitze und die zwei Felder der Mauer, die unter dem obersten Stein liegen, werden immer erzeugt, da diese immer verwendet werden. Die zweidimensionalen Arrays werden an der ersten und zweiten Stelle in der richtigen Größe initalisiert. Danach wird die Funktion \textit{GenerateSolutionArray()} aufgerufen, diese initailisert die restlichen Ebenen des Lösungsarrays und des Eingabearrays. Nachdem das Lösungsarray generiert wurde, werden die Reihen der Pyramide in die Szene geladen. Am Schluss wird die Funktion \textit{PlayGame()} aufgerufen, diese startet das Spiel für die Kinder.\\
\begin{lstlisting}[language=csh, caption={Triangle.cs Update-Funktion}]
	void Update(){
		bool enabel = false;
		for (int i = 0; i < triangleSize - 1; i++){
			for (int j = 0; j < jaggedInputs[i].Length; j++) {
				if(string.IsNullOrEmpty(jaggedInputs[i][j].text)){
					enabel = true;
					testSolution.interactable = false;

				}
			}
		}
		if(!enabel){
			testSolution.interactable = true;
		}
	}
\end{lstlisting}
In dem Skript für die Rechenmauer und in dem für 'Führe die Reihe fort', wird die Update-Funktion genutzt, um zu überprüfen ob alle Eingabefelder gefüllt sind. Wenn eines der Felder leer ist, wird der Test-Button nicht aktiviert. In anderen Szenen wurde dies mit der Funktion \textit{onValueChanged} gelöst, aber um mehrere Felder zu überprüfen ist die Variante in der Update-Funktion angenehmer. In diesem Skript läuft in der Update-Funktion eine doppelte \textit{for}-Schleife durch. Diese läuft für jede Zeile die Länge ab und kontrolliert ob eines der Felder leer ist. Wenn ein Feld leer ist, wird die \textit{boolean} Variable auf \textit{true} gesetzt, solange diese nicht \textit{false} ist, wird der Test-Button nicht aktiviert.\\
\begin{lstlisting}[language=csh, caption={Triangle.cs GenerateSolutionArray-Funktion}]
	public void GenerateSolutionArray(){
		if(triangleSize >= 3){
			jaggedSolution[2] = new int[3];
			jaggedInputs[2] = new InputField[3];
			if(triangleSize >= 4){
				jaggedSolution[3] = new int[4];
				jaggedInputs[3] = new InputField[4];
				if(triangleSize == 5){
					jaggedSolution[4] = new int[5];
					jaggedInputs[4] = new InputField[5];
				}
			}
		}
	}
\end{lstlisting}
In dieser Funktion wird kontrolliert, welche Größe die unterste Reihe der Pyramide hat. Je nach der Größe werden die einzelnen Ebenen des zweidimensionales Arrays mit der passenden Größe deklariert. Zeitgleich wird das Array für die Eingaben des Kindes erzeugt, somit sind die beiden Arrays gleich groß.\\
\begin{lstlisting}[language=csh, caption={Triangle.cs ShowBottomLineNumbers-Funktion}]
	public void ShowBottomLineNumbers(){
		for (int i = 0; i < bottomNumbers.Length; i++){
			if(triangleSize == 3){
				jaggedSolution[2][i] = bottomNumbers[i];
				jaggedInputs[2][i].text = bottomNumbers[i].ToString();
				jaggedInputs[2][i].interactable = false;
			}
			else if(triangleSize == 4){
				jaggedSolution[3][i] = bottomNumbers[i];
				jaggedInputs[3][i].text = bottomNumbers[i].ToString();
				jaggedInputs[3][i].interactable = false;
			}
			else if(triangleSize == 5){
				jaggedSolution[4][i] = bottomNumbers[i];
				jaggedInputs[4][i].text = bottomNumbers[i].ToString();
				jaggedInputs[4][i].interactable = false;
			}
		}
\end{lstlisting}
In dieser Funktion wird die unterste Reihe gefüllt. Es gibt drei \textit{If}-statements, die die breite der Mauer in der untersten Reihe überpüfen. Diese statements befinden sich in einern \textit{for}-Schleife und füllen je nach breite die Unterste reihe mit den zufällig generierten Zahlen auf. Die Funktion dafür wird später erklärt. Die Nummer aus dem Array wird auch sofort in das Eingabearray eingefügt, da diese Reihe für Kinder nicht bearbeitbar ist, aber die Kontrolle zwischen dem Eingabearray und dem Lösungsarray so einfacher ist. Jedes Feld aus dem Eingabearray, in dem eine Zahl aus der untersten Reihe steht, wird deaktiviert.\\
\begin{lstlisting}[language=csh, caption={Triangle.cs FillSolutionArray-Funktion}]
	public void FillSolutionArray(){
		for (int i = triangleSize - 2; i >= 0; i--){
			for (int j = 0; j < jaggedSolution[i].Length; j++){
				jaggedSolution[i][j] = jaggedSolution[i + 1][j] + jaggedSolution[i + 1][j + 1];
			}
		}
	}
\end{lstlisting}
In dieser Funktion werden die Lösungen berechnet und an die richtige Position gespeichert. Dafür werden zwei \textit{For}-Schleifen benötigt. Um die Lösung zu berechnen, werden zwei nebeneinaderliegende Felder verwendet. Die äußere \textit{for}-Schleife zählt die Variable i nach unten, die innere zählt nach oben. Dies wurde so implementiert, da die Pyramide von unten nach oben berechnen werden müssen. Deswegen wurde i mit der Pyramidengröße - 2 belegt, weil die Schleife bis zur Zahl Null durchlaufen muss und weil die unterste Reihe schon gefüllt ist. Die zweite Schleife, geht die einzelnen Reihen durch und berechnet die Lösungen.\\
Die Funktion, die für den Test-Button aufgerufen wird, steht in einer Zeile, diese startet die \textit{Coroutine}, um die Ergebnisse zu überprüfen.\\
\begin{lstlisting}[language=csh, caption={Triangle.cs PlayGame-Funktion}]
	public void PlayGame(){
		lvlCounter++;
		if(lvlCounter > lvlNumber){
			PlayerPrefs.SetInt("wrongAnswers", wrongChoice);
			SceneManager.LoadScene("LearnFinishScreen");
			Debug.Log("Game Vorbei \n" + "Anzahl Fehler: " + wrongChoice);
		}
		if(lvlCounter != 1){
			Debug.Log("hier war ich");
			ClearArray();
		}
		if(lvlCounter <= lvlNumber)	lvlText.text = "Level: " + lvlCounter + "/" + lvlNumber;
		GenerateNumbers();
		ShowBottomLineNumbers();
		FillSolutionArray();
	}
\end{lstlisting}
Diese Funktion zählt einen Zähler nach oben, um zu sehen welches Level das Kind spielt, um die Szene nach Abschluss des Spieles zu laden. Wenn der Zähler größer als die Zahl Eins ist, wird die Mauer mit jedem Aufruf der PlayGame-Funktion geleert. Wenn der Zähler kleiner als die maximale Anzahl an Leveln ist, wird das Textfeld aktualisiert und die zufälligen Zahlen werden erzeugt. Danach wird die unterste Reihe gefüllt und das Lösungsarray berechnet.\begin{lstlisting}[language=csh, caption={Triangle.cs ClearArray-Funktion}]
	public void ClearArray(){
		for (int i = 0; i < triangleSize - 1; i++){
			for (int j = 0; j < jaggedInputs[i].Length; j++) {
				jaggedInputs[i][j].text = "";
				jaggedInputs[i][j].interactable = true;
			}
		}
	}
\end{lstlisting}
Um das Array zu leeren, wird mithilfe einer doppelten \textit{for}-Schleife jedes Feld der Mauer durchgegangen, diese werden anschließend geleert und wieder aktiviert, sodass das Kind diese wieder füllen kann.\\
\begin{lstlisting}[language=csh, caption={Triangle.cs SpawnPyramidRows-Funktion}]
	public void SpawnPyramidRows(){
		for (int i = 0; i < triangleSize; i++){
			for (int j = 0; j < i + 1; j++){
				spawnedObject = Instantiate(brick, triangleGameObject.transform.position, Quaternion.identity);
				spawnedObject.name += i;
				spawnedObject.transform.SetParent(triangleGameObject.transform.GetChild(i));
				spawnedObject.transform.localScale = scaleSize;
				jaggedInputs[i][j] =spawnedObject.GetComponent<InputField>();
			}
		}
	}
\end{lstlisting}
Um die Reihen der Mauer zu erzeugen, wird wieder eine doppelte \textit{for}-Schleife benötigt. Diese läuft die Variable \textit{triangleSize} durch und fügt in jedes freie Feld der \textit{Layout Group} einen Stein ein. Das Erzeugen der Mauer läuft nach dem gleichen Prinzip ab nachdem die Objekte bei den Mengenvergleichen oder bei dem Spiel Blitzblick erzeugt werden. Erst werden diese instantiiert, dafür wird das \textit{GameObject} des Steines benötigt. Danach wird ein Feld benötigt, in welches der Stein gesetzt werden soll, hierbei ist die Rotation wieder unwichtig. Die Objekte bekommen für die Hierarchie einen Namen und werden an die passende Kinderposition gesetzt. Da die Steine zu klein skaliert werden, wird die Skalierung auf eins zurückgesetzt. Als Letztes werden die Prefabs der Eingabefelder in das Array aus Eingabefeldern eingefügt.\\
\begin{lstlisting}[language=csh, caption={Triangle.cs GenerateNumbers-Funktion}]
	public void GenerateNumbers(){
		bottomNumbers = new int[triangleSize];
		for (int i = 0; i < bottomNumbers.Length; i++) {
			bottomNumbers[i] = UnityEngine.Random.Range(0, maxNumber);
			jaggedSolution[triangleSize - 1][i] = bottomNumbers[i];
		}
	}
\end{lstlisting}
Diese Funktion deklariert die Größe eines normalen Arrays, welches die Größe der untersten Reihe beinhaltet. In einer \textit{for}-Schleife wird für jedes Feld in diesem Array eine zufällige Zahl mit der UnitEngine.Random.Range-Funktion generiert. Diese wird in die unterste Reihe des Arrays, das alle Lösungen speichert, eingefügt.\\
\begin{lstlisting}[language=csh, caption={Triangle.cs SpawnFirework-Funktion}]
	public void SpawnFirework(){
		spawnedObject = Instantiate(fireworkPrefab, leftFirework.transform.position, leftFirework.transform.rotation);
		spawnedObject.transform.SetParent(leftFirework.transform);
		spawnedObject.transform.localScale = scaleSize;
		spawnedObject = Instantiate(fireworkPrefab, rightFirework.transform.position, rightFirework.transform.rotation);
		spawnedObject.transform.SetParent(rightFirework.transform);
		spawnedObject.transform.localScale = scaleSize;
	}
\end{lstlisting}
In dieser Funktion werden die Prefabs des Feuerwerks erzeugt. Dafür werden die Objekte so wie in den anderen Skripten erzeut. Der Unterschied hierbei ist, dass das Feuerwerk eine Rotation benötigt um nicht gerade aufzutauchen. Daher werden aus den Bereichen links und rechts um den Test-Button die Werte der Rotation genommen und das Feuerwerk-Objekt rotiert. Um an die Werte zu kommen, wird \textit{leftFirework.transfom.rotation} aufgerufen. Die beiden Feuerwerk Objekte wurden auch in diesem Fall wieder zu klein skaliert und müssen auch wieder auf die Originalgröße skaliert werden.\\
\begin{lstlisting}[language=csh, caption={Triangle.cs DeleteFirework-Funktion}]
	public void DeleteFirework(){
		Destroy(leftFirework.transform.GetChild(0).gameObject);
		Destroy(rightFirework.transform.GetChild(0).gameObject);
	}
\end{lstlisting}
Damit das Feuerwerk nach einer Sekunde wieder verschwindet, wird die Funktion \textit{DeleteFirework} aufgerufen. Diese löscht aus den beiden eigentlich leeren Objekten \textit{leftFirework} und \textit{rightFirework} die Prefabs der Feuerwerke.\\
\begin{lstlisting}[language=csh, caption={Triangle.cs Waiter-Coroutine}]
	IEnumerator Waiter(int sec){
		bool finished = false;
		if(lvlCounter <= lvlNumber){
			for (int i = triangleSize - 2; i >= 0; i--){
				for (int j = 0; j < jaggedSolution[i].Length; j++){
					if(jaggedSolution[i][j] == int.Parse(jaggedInputs[i][j].text)){
						jaggedInputs[i][j].interactable = false;
						continue;
					}
					else{
						jaggedInputs[i][j].text = "";
						wrongChoice++;
						finished = true;
					}
				}
			}
			if(!finished){
				SpawnFirework();
				yield return new WaitForSeconds(sec);
				DeleteFirework();
				PlayGame();
			}
		}
	}
\end{lstlisting}
Hier wird überprüft, ob die Mauer komplett richtig gerechnet wurde. Dafür wird eine boolean Variable benötigt, diese ist standardmäßig mit \textit{false} implementiert. Nachdem die Funktion getestet hat, ob das Kind noch nicht alle Level gespielt hat, wird mit einer doppelten \textit{for}-Schleife überprüft, ob die Eingabe des Kindes die gleiche ist wie die Zahlen in dem Lösungsarray. Wenn ein Ergebnis richtig ist, wird das Eingabefeld deaktiviert und die Schleife springt einen Durchgang weiter. Wenn eine der Eingaben nicht richtig ist, wird das Feld geleert, der Fehlerzähler wird eins nach oben gezählt und die Variable \textit{finished} wird auf \textit{true} gesetzt. Wenn die Variable \textit{false} ist, wurde die gesamte Mauer gelöst, also wird das Feuerwerk mit dem Aufruf der Funktion \textit{SpawnFirework()} erzeugt. Dann wartet das Skript mithilfe des \textit{yield return new WaitForSeconds(sec)} eine Sekunde. Danach wird das Feuerwerk mit der Funktion \textit{DeleteFirework()} gelöscht. Als Letztes wird die Funktion \textit{PlayGame()} aufgerufen und damit eine neue Mauer generiert.\\
\subsection{Szene - Blitzblick}
\subsubsection{Blitzblick - Design}
\begin{figure}[htbp]
  \centering
  \includegraphics[width=0.65\textwidth,height=0.55\textheight,keepaspectratio]{images/lightningViewWithoutCover.PNG}
  \caption{Blitzblick ohne Vorhang}
  \label{withoutCover}
\end{figure}
\begin{figure}[htbp]
  \centering
  \includegraphics[width=0.65\textwidth,height=0.55\textheight,keepaspectratio]{images/lightningViewCover.PNG}
  \caption{Blitzblick mit Vorhang}
  \label{withCover}
\end{figure}
Die Szene ist hier in zwei Bildern gezeigt. In der Abbildung \ref{withoutCover} ist die Szene ohne den Vorhang zu sehen und in der Abbildung \ref{withCover} mit dem Vorhang. Der Vorhang wurde in Blau aus dem Internet heruntergeladen und in Unity wurde er dann Grün gefärbt\autocite{curtain}. Das Spiel Blitzblick zeigt dem Schüler/der Schülerin eine Anzahl an Objekten und verdeckt diese danach. Das Kind muss dann sehr schnell sehen wie viele der Objekte angezeigt wurden und diese in ein Textfeld eingeben. Wenn das Kind den Test-Button auswählt, wird die Zahl überprüft und wenn es richtig lag, wird eine neue Runde geladen. Wenn das Kind die falsche Anzahl angibt, bekommt es die Objekte noch einmal für eine Sekunde zu sehen. Die Zeit, wie lange die Objekte zu sehen sind, wird nicht verringert oder erhöht, da eine Sekunde für die vorhergesehene Anzahl an Objekten einen angemessenen Schwierigkeitsgrad hat.\\
Die Szene hat wie jede andere einen Header. In diesem ist ein Menü-Button, eine Überschrift und die Levelanzahl zu sehen. In der Mitte der Szene ist das Feld, in dem die Objekte auftauchen, welches von einem grünen Vorhang verdeckt wird. Rechts davon ist das Eingabefeld für den Nutzer. Ganz unten in der Szene ist der Button zum Testen der Lösung. Dieser ist, solange keine Zahl eingegeben wurde, deaktiviert. In dieser Szene wird dem Schüler/der Schülerin durch einen grünen Rahmen angezeigt, ob die Lösung richtig oder falsch ist.\\
\subsubsection{Blitzblick - Skript}
\begin{lstlisting}[language=csh, caption={lightningView.cs Variablendeklaration}]
public class lightningView : MonoBehaviour {

	public Text showLevel;

	public InputField solution;

	public Image imageColor;

	public GameObject spawnedObject;
	public GameObject spawner;
	public GameObject cover;
	private GameObject allObjects;

	public GameObject spawnerCoverLeft;

	public Button checkButton;
	public Button menu;

	private List<Vector2> allPositions = new List<Vector2>(10);

	//Variabln for lightningView
	private int numberLeft;
	private int counterRound = 0;
	private int wrongChoice = 0;
	private int lvlNumber = 10;
	private float minDistance = 0f;
	private bool hitPosition = false;
	private Vector3 scaleSize = new Vector3 (1.0f, 1.0f, 1.0f);
\end{lstlisting}
Für dieses Spiel wird ein Textfeld benötigt, um das Level anzuzeigen. Die Szene benötigt auch ein Eingabefeld und ein Objekt, um die Farbe des Rahmens zu ändern. Da in diesem Skript wieder Objekte erzeugt werden, wird ein Bereich, in dem diese auftauchen sollen, festgelegt. In Unity wurde ein Prefab der Objekte vorbereitet. Damit der Vorhang über den Objekten zu sehen ist, wird dieser auch durch ein Prefab geladen. In diesem Skript werden die Objekte an zufälligen Positionen erzeugt. Damit diese nicht aufeinander liegen, wird die Position in eine Vektor-Liste eingespeichert. Um das Spiel zu verwalten, werden einife \textit{private} Attribute benötigt. Um zu unterscheiden, wie groß die minimale Distanz zwischen den Objekten und zum Rand sein soll, wurde eine Variable deklariert. Um eine Überschneidung zu finden, benötigt das Skript eine boolean Variable. Die Objekte werden auch in diesem Fall wieder zu klein skaliert, deswegen wird wieder ein Vektor mit der Standardgröße festgelegt.\\
\begin{lstlisting}[language=csh, caption={lightningView.cs Start-Funktion}]
	void Start() {
		minDistance = spawnedObject.GetComponent<RectTransform>().rect.width / 2;
		spawner.transform.localScale = scaleSize;
		checkButton.interactable = false;
		menu.onClick.AddListener(() => GoBack());
		lvlNumber = MenuPickLevelAdvanced.lvlAmmountStatic;
		checkButton.onClick.AddListener(() => buttonClick());
		solution.onValueChanged.AddListener(delegate {EnableButton(); });
		playGame();
	}
\end{lstlisting}
Diese Start Funktion legt als Erstes die minimale Distanz fest, die ein Objekt zu einem anderen Objekt besitzen darf. Diese wird wie folgt berechnet. Das Skript holt sich von dem Prefab Objekt \textit{spawnedObject} ein Rechteck. Der Befehl \textit{GetComponent<RectTransform>().rect.width} sucht aus dem Prefab das Rechteck und nimmt von diesem die Breite. Die minimale Distanz zwischen den Objekten ist die Hälfte des bekommenen Rechtecks. Danach wird der Bereich für die Objekte auf die Originalgröße skaliert, da dieser nicht mitskaliert wurde. Das heißt, im Vollbildmodus wurde dieser nicht größer. Der Test-Button wird wieder deaktiviert. In dieser Szene werden von den statischen Variablen nur die Levelanzahl verwendet. Das Feld der Objekte kann nur eine gewisse Anzahl an Objekten beinhalten und diese Zahl darf auch für die Kinder nicht zu groß werden. Um zu überprüfen, ob das Kind etwas in das Eingabefeld eingegeben hat, wird die onValueChange Methode verwendet. Als Letztes wird die Funktion \textit{playGame()} aufgerufen.\\
Die nächsten zwei Funktionen des Skriptes laden das Menü und aktivieren oder deaktivieren den Test-Button. Diese sind ähnlich zu den vorigen Funktionen und werden deswegen nicht genauer erläutert.\\
\begin{lstlisting}[language=csh, caption={lightningView.cs buttonClick-Funktion}]
	public void buttonClick(){
		checkButton.interactable = false;
		foreach (Transform child in spawnerCoverLeft.transform) {
			Destroy(child.gameObject);
		}
		StartCoroutine(solutionWaiter(1));
	}
\end{lstlisting}
Diese Funktion wird beim Aufruf des Test-Buttons durchlaufen. Da in dieser Szene über den Objekten ein Vorhang schwebt, wird dieser einmal gelöscht und die \textit{Coroutine}, die die Lösung überprüft wird aufgerufen.\\
Auch die Funktion für die Zufallszahlen wird nicht genauer erklärt. In diesem Spiel wurde die maximale Zahl der Objekte auf zehn gesetzt, da das Feld nicht mehr Objekte beinhalten kann und Kinder nicht zu viele Objekte in so kurzer Zeit erkennen können.\\
\begin{lstlisting}[language=csh, caption={hideCircle.cs buttonClick-Funktion}]
	public void hideCircle(){
		StartCoroutine(waiter(1));
	}
	
	IEnumerator waiter(int sec){
		solution.interactable = false;
		yield return new WaitForSeconds(sec);
		GameObject hideCircles = Instantiate(cover, spawnerCoverLeft.transform.position, Quaternion.identity);
		hideCircles.transform.SetParent(spawnerCoverLeft.transform);
		hideCircles.transform.localScale = scaleSize * (0.75f);
		solution.interactable = true;
	}
\end{lstlisting}
Um den Vorhang nicht direkt am Anfang zu erzeugen, wird eine weitere \textit{Coroutine} geladen. Diese deaktiviert das Eingabefeld für den Nutzer so lange bis der Vorhang aufgetaucht ist. Dies hat den Vorteil, dass der Nutzer/Nutzerin nicht schon direkt die Zahl eintragen kann, sondern wirklich eine Sekunde warten muss. Danach wird, wie in anderen Szenen, das Objekt erzeugt. Da der Vorhang aber erst zu klein war und in der Originalgröße zu groß ist, wird diese mit 0.75 multipliziert.\\
\begin{lstlisting}[language=csh, caption={hideCircle.cs setObjects-Funktion}]
	public void setObjects(){
		Debug.Log(numberLeft);
		for (int i = 0; i < numberLeft; i++) {
			allObjects = Instantiate(spawnedObject, spawner.transform.position, Quaternion.identity);
			allObjects.name += i;
			allObjects.transform.SetParent(spawner.transform);
			allObjects.transform.localScale = scaleSize;
			ChangePosition(i);
		}
	}
\end{lstlisting}
Hier werden für die zufällig generierte Zahl die Objekte mithilfe einer \textit{for}-Schleife generiert. Diese werden auf die normale Größe skaliert und danach wird die Funktion \textit{ChangePosition(i)} aufgerufen.\\
\begin{lstlisting}[language=csh, caption={hideCircle.cs ChangePosition-Funktion}]
	public void ChangePosition(int firstNumber){
		RectTransform spawnRect = spawner.GetComponent<RectTransform>();
		float width = spawnRect.rect.width;
		float height = spawnRect.rect.height;
		float xPos = UnityEngine.Random.Range(-width/2 + minDistance, width/2 - minDistance);
		float yPos = UnityEngine.Random.Range(-height/2 + minDistance, height/2 - minDistance);
		Vector2 newPos = new Vector2(xPos, yPos);
		Debug.Log("x:" + xPos);

		if(firstNumber == 0){
			allPositions.Add(newPos);
			allObjects.transform.localPosition = newPos;
			return;
		}


		for (int i = 0; i < allPositions.Count; i++){
			float distance = Vector2.Distance(new Vector2(allPositions[i].x, allPositions[i].y), new Vector2(newPos.x, newPos.y));
			if(distance >= minDistance){
				hitPosition = true;
			}
			else{
				hitPosition = false;
				break;
			}

		}
		if(hitPosition){
			allPositions.Add(newPos);
			allObjects.transform.localPosition = newPos;

		}
		else{
			ChangePosition(firstNumber);
		}

	}
\end{lstlisting}
Hier wird eine zufällige Position in dem Rechteck \textit{spawner} erzeugt. Dafür bekommt ein Objekt \textit{RectTransform} das Objekt aus \textit{spawner} zugewiesen. Der Vorteil hierbei ist, dass aus dem Objekt \textit{RectTransform} die höhe und breite ausgelesen werden kann. Diese werden in Variablen gespeichert. Dann wird für die x- und y-Position eine zufällige Zahl erzeugt. Die x-Position wird aus der Breite berechnet. Die kleinstmögliche  Zahl befindet sich an der Stelle \textit{(-width/2) + minDistance}, die maximale Zahl befindet sich an der Stelle \textit{(width/2) - minDistance}. Es muss die halbe negative Breite sein, da das Skript vom Mittelpunkt des Objektes ausgeht. Also muss erst die negative breite geteilt werden und danach die minimale Distanz darauf gerechnet werden. Das ergebnis ist die zahl die die Objekte nach links begrenzt. Für den rechten Rand des Rechtecks wird die positive Breite halbiert und minus der minimalen Distanz gerechnet. Für die y-Position wird die gleiche Rechnung benötigt. Nur dass anstatt der Breite die Höhe des Rechtecks verwendet wird. Nachdem die beiden Positionen zufällig erzeugt wurden, werden diese in einen Vektor gespeichert. Wenn die Übergabevariable eine Null ist, also wenn die \textit{for}-Schleife bei dem ersten Durchlaufen ist, wird die Position, die im Vektor \textit{newPos} gespeichert wurde, in eine Liste gespeichert. Danach wird das Objekt an diese Position bewegt und die Funktion mithilfe des Befehls \textit{return} verlassen.\\
Wenn das nicht der erste Durchlauf ist, wird mithilfe einer \textit{for}-Schleife, für das aktuelle Objekt zuerst die Distanz zwischen allen Objekten und der neuen Position berechnet.
Die Methode \textit{Distance} errechnet aus zwei Positionen die distanz zwischen diesen. Das ergebnis wird in einen Vektor gespeichert. Die Funktion benötigt die Positionen in Form eines Vektors um die Distanz zu berechnen. Wenn die Distanz dann größer oder gleich der minimalen Distanz ist, wird eine Position gefunden, die genutzt werden kann, um das Objekt zu platzieren. Wenn das nicht der Fall ist, wird die boolean Variable auf \textit{false} gesetzt und die Schleife verlassen. In diesem Fall wird durch eine Abfrage die Funktion \textit{ChangePosition(firstNumber)} erneut aufgerufen. Wenn die Variable \textit{hitPosition} \textit{true} ist, wird die neu gefundene Position in die Liste gespeichert und das Objekt an die besagte Position bewegt.\\
Die Funktion, die die Objekte wieder löscht, ist dieselbe, die die Objekte in dem Spiel 'größer', 'kleiner', 'gleich' löscht. Diese geht mithilfe einer \textit{foreach}-Schleife alle Objekte durch und löscht diese.\\
\begin{lstlisting}[language=csh, caption={hideCircle.cs playGame-Funktion}]
	public void playGame(){
		counterRound++;
		if(counterRound > lvlNumber){
			PlayerPrefs.SetInt("wrongAnswers", wrongChoice);
			SceneManager.LoadScene("LearnFinishScreen");
			Debug.Log("Game Vorbei \n" + "Anzahl Fehler: " + wrongChoice);
		}
		//change color to white
		imageColor.color = new Color32(255, 255, 255, 255);
		if(counterRound <= lvlNumber)	showLevel.text = "Level: " + counterRound + "/" + lvlNumber;
		numberLeft = randomNumber();
		Debug.Log(numberLeft);
		solution.text = "";
		setObjects();
		hideCircle();
	}
\end{lstlisting}
Die Funktion \textit{playGame} arbeitet ähnlich zu den anderen. Zuerst wird der Zähler für die anzahl an Runden hochgezählt und getestet, ob das Kind noch Level spielen sollte. Wenn nicht, wird die Szene für den Abschluss geladen. Die Farbe des Rahmens wird wieder auf weiß gesetzt und ein neues Level wird generiert. Dafür muss das Eingabefeld geleert werden und neue Objekte zufällig erzeugt werden. Zum Schluss taucht der Vorhang wieder aus.\\
\begin{lstlisting}[language=csh, caption={hideCircle.cs solutionWaiter-Funktion}]
	IEnumerator solutionWaiter(int sec){
		solution.interactable = false;
		if(counterRound <= lvlNumber){
			if(int.Parse(solution.text) == numberLeft){
				//change color green
				imageColor.color = new Color32(37, 250, 53, 255);
				yield return new WaitForSeconds(sec);
				deletObjects();
				playGame();

			}
			else{
				//change color red
				imageColor.color = new Color32(251, 37, 37, 255);
				yield return new WaitForSeconds(sec);
				wrongChoice++;
				//change color to white
				imageColor.color = new Color32(255, 255, 255, 255);
				solution.text = "";
				hideCircle();
			}
		}
	}
\end{lstlisting}
Diese \textit{Coroutine} deaktiviert das Eingabefeld und überprüft, ob die eingegebene Zahl dieselbe ist, wie die zufällig generierte. Wenn das der Fall ist, wird der Rahmen grün und nach einer Sekunde werden die Objekte gelöscht und eine neue Runde mit der playGame Funktion gestartet. Wenn das Kind falsch liegt, wird der Rahmen für eine Sekunde rot und danach wieder weiß, das Eingabefeld wird geleert und die Objekte werden wieder versteckt.
\subsection{Szene - Führe die Reihe fort}
\subsubsection{Führe die Reihe fort - Design}
\begin{figure}[htbp]
  \centering
  \includegraphics[width=0.65\textwidth,height=0.55\textheight,keepaspectratio]{images/finishPattern.PNG}
  \caption{Führe die Reihe fort}
  \label{finishPattern}
\end{figure}
In dieser Szene muss das Kind eine vorgegebene Reihe erkennen und fortführen. Den Kindern werden fünf Zahlen vorgegeben und aus dieser Reihe müssen sie das Muster der Zahlen erkennen und anwenden. Die Szene hat einen Header, der gleich aufgebaut ist, wie in den anderen Szenen. In der Mitte der Szene befinden sich insgesamt zehn Felder, in der oberen Reihe sind dies fünf Textfelder, welche das Zahlenmuster anzeigen. In der unteren Reihe befinden sich fünf Textfelder, diese haben einen Platzhalter, in diesem steht das Wort Zahl. In diese können die Kinder ihre Zahlen eingeben. Der Test-button wird auch in dieser Szene erst aktiviert, wenn alle Eingabefelder eine Zahl beinhalten. Diese Felder können keine anderen Zeichen als Zahlen enthalten. Dem Schüler/der Schülerinnen wird auch in dieser Szene anhand eines Feuerwekes gezeigt, ob es alle Felder richtig ausgefüllt hat. In Abbildung \ref{firework} ist das erwähnte Feuerwerk zu sehen. Das Feuerwerk wurde aus dem Internet als Icon heruntergeladen\autocite{firework}. Das Spiel wurde so implementiert, dass eine zufällige Zahl erzeugt wird und mit zwei weiteren Zufallszahlen die Reihe fortgesetzt wird. Es wurde sich gegen noch mehr Zufallszahlen entschieden, damit die Schüler/Schülerinnen keine zu komplexen Reihen bekommen. Es wurde aber so implementiert, dass in den unterschiedlichen Schwierigkeitsgraden die Anfangszahl größer werden kann und auch die Zahlen, die die Reihe fortzusetzen größer werden.
\subsubsection{Führe die Reihe fort - Skript}
\begin{lstlisting}[language=csh, caption={hideCircle.cs Variablendeklaration}]
public class FinishPattern : MonoBehaviour {

	public Text lvlText;

	//random Numbers
	private int randomNumber;
	private int FirstRandomNumber;
	private int SecondRandomNumber;

	//lvl variabls
	private int lvlNumber = 10;
	private int wrongAnswers;
	private int lvlCounter = 0;
	private int maxNumbers = 20;

	//int start start pattern
	private int[] patternNumbers = new int[5];

	//int solutions
	private int[] solutionsNumbers = new int[5];


	//Button test
	public Button checkSolution;
	public Button menu;

	//Firework
	private GameObject spawnedObject;
	public GameObject leftFirework;
	public GameObject rightFirework;
	public GameObject fireworkPrefab;

	public InputField[] arrayInputs = new InputField[5];
	public Text[] arrayTextFields = new Text[5];
\end{lstlisting}
Für dieses Spiel werden mehrere zufällige Zahlen benötigt. Eine um die Reihe zu starten und zwei um die Reihe fortzusetzen.  Es werden auch wieder Variablen benötigt, um das Spiel auszuführen. Wie die Levelanzahl, die Fehleranzahl und die maximal größte Zahl. Für die Reihe und die Lösungen werden zwei Arrays deklariert. Diese bekommen die größe fünf. Die Szene benötigt zwei Objekte für die Buttons und die Objekte für das Feuerwerk. In dieser Szene wurden anfangs die Objekte für die Eingaben der Kinder und die Textfelder einzeln deklariert, in der aktuellsten Umsetzung wurden auch dafür zwei Arrays deklariert. Diese sind \textit{public} und somit können die einzelnen Objekte in Unity in das Skript von Hand eingefügt werden. Wie dies abläuft, ist in Abbildung \ref{singleButtonUnity} zu sehen.\\
\begin{lstlisting}[language=csh, caption={hideCircle.cs Start-Funktion}]
	void Start() {
		menu.onClick.AddListener(() => GoBack());
		lvlNumber = MenuPickLevelAdvanced.lvlAmmountStatic;
		maxNumbers = MenuPickLevelAdvanced.maxNumberStatic;
		checkSolution.interactable = false;
		checkSolution.onClick.AddListener(() => clickButton());
		for (int i = 0; i < arrayInputs.Length; i++){
			arrayInputs[i].onValueChanged.AddListener(delegate { EnableButton(); });
		}
		PlayGame();
	}
\end{lstlisting}
In der Start Funktion werden die Listener für die Buttons deklariert. Die größtmögliche Nummer, die zufällig erzeugt werden kann, wird festgelegt und ebenso die Anzahl an Leveln.  Der Test-Button wird deaktiviert. Um eine andere Art der Überprüfung, ob ein Eingabefeld leer ist zu verwenden, wurde jedem der Eingabefelder über die Methode \textit{onValueChanged} ein Listener hinzugefügt. Diese Art ist zwar schon bekannt, wurde in diesem Teil des Projektes aber noch nicht mit einem Array aus Objekten verwendet. Als Letztes wurde die Funktion, die das Spiel startet und immer wieder neu erzeugt aufgerufen, die \textit{PlayGame}-Funktion.\\
\begin{lstlisting}[language=csh, caption={hideCircle.cs EnableButton-Funktion}]
	public void EnableButton() {
		if (string.IsNullOrEmpty(arrayInputs[0].text) || string.IsNullOrEmpty(arrayInputs[1].text) || string.IsNullOrEmpty(arrayInputs[2].text) || string.IsNullOrEmpty(arrayInputs[3].text) || string.IsNullOrEmpty(arrayInputs[4].text)) {
			checkSolution.interactable = false;
		} else {
			checkSolution.interactable = true;
		}
	}
\end{lstlisting}
Hier ist die Funktion zu sehen, die den Test-Button aktiviert oder deaktiviert. Da es mehrere Felder sind, können diese mithilfe einer \textit{for}-Schleife abgefragt werden. Diese würde dann durch alle Eingabefelder iterieren und überprüfen, ob an dieser Stelle eine Eingabe ist. Wenn dies der Fall ist, könnte nach der gesamten Überprüfung der Test-Button aktiviert werden, wenn nicht, dann würde dieser deaktiviert bleiben. Da es aber mit der Schleife mehr Code wäre, wurde in einem \textit{if}-Statement mithilfe der Oder-Verknüpfung überprüft, ob eines der Felder leer ist. Wenn ja, wird der Button nicht aktiviert oder erneut deaktiviert, falls das Kind die Eingabe aus einem der Felder löscht. Wenn alle Felder gefüllt sind, wird in einer \textit{else} der Button aktiviert.\\
Die nächsten vier Funktionen des Skriptes werden nicht gezeigt. Diese werden einmal aufgerufen, wenn der Menü-Button oder der Test-Button gedrückt wurde. Diese laden die Menü-Szene beziehungsweise starten die Abfrage, ob die eingegebene Lösung des Kindes richtig ist. Die letzten zwei Funktionen sind für die zufälligen Zahlen zuständig. Die erste der beiden erzeugt eine zufällige Zahl zwischen null und der maximal möglichen Zahl. Die andere wird später zweimal aufgerufen, um die Zahlen für das Muster zu erzeugen. Diese werden zwischen eins und der maximal möglichsten Zahl erzeugt. Der Grund hierfür ist, dass es für die Kinder unspannend wäre, wenn sich das Muster um null verändern würde.\\
\begin{lstlisting}[language=csh, caption={hideCircle.cs PlayGame-Funktion}]
	public void PlayGame() {
		lvlCounter++;
		if(lvlCounter > lvlNumber){
			PlayerPrefs.SetInt("wrongAnswers", wrongAnswers);
			SceneManager.LoadScene("LearnFinishScreen");
			Debug.Log("Game Vorbei \n" + "Anzahl Fehler: " + wrongAnswers);
		}
		ResetInputFields();
		if(lvlCounter <= lvlNumber)	lvlText.text = "Level: " + lvlCounter + "/" + lvlNumber;;
		randomNumber = RandomBeginningNumber(maxNumbers);
		FirstRandomNumber = RandomPatternNumbers();
		SecondRandomNumber = RandomPatternNumbers();
		SetStartPattern();
		SetSolution();
		SetPattern();
	}
\end{lstlisting}
Diese Funktion ist wieder ähnlich zu den anderen. Sie überprüft wie viele Level noch gespielt werden müssen und beendet das Spiel oder setzt es fort. Wenn das Kind noch spielen soll, wird der Text um das Level anzuzeigen aktualisiert. Die Eingabefelder werden zurückgesetzt und die zufälligen Zahlen werden erzeugt und in die dazugehörige Variable gespeichert. Danach wird das Muster erzeugt und die daraus folgende Lösung berechnet. Als Letztes wird das Muster in die Textfelder geschrieben.\\
\begin{lstlisting}[language=csh, caption={hideCircle.cs SetStartPattern- und SetSolution-Funktion}]
	public void SetStartPattern() {
		patternNumbers[0] = randomNumber;
		patternNumbers[1] = patternNumbers[0] + FirstRandomNumber;
		patternNumbers[2] = patternNumbers[1] + SecondRandomNumber;
		patternNumbers[3] = patternNumbers[2] + FirstRandomNumber;
		patternNumbers[4] = patternNumbers[3] + SecondRandomNumber;
	}

	public void SetSolution() {
		solutionsNumbers[0] = patternNumbers[4] + FirstRandomNumber;
		solutionsNumbers[1] = solutionsNumbers[0] + SecondRandomNumber;
		solutionsNumbers[2] = solutionsNumbers[1] + FirstRandomNumber;
		solutionsNumbers[3] = solutionsNumbers[2] + SecondRandomNumber;
		solutionsNumbers[4] = solutionsNumbers[3] + FirstRandomNumber;
	}
\end{lstlisting}
Diese Funktionen könnten auch in einer Funktion stattfinden, wurde aber für die Übersichtlichkeit in zwei unterschiedlichen implementiert. Die SetStartPattern Funktion, speichert an die erste Stelle die Start-Zahl ein. Danach wird mithilfe der zwei anderen Zufallszahlen, hier \textit{FirstRandomNumber} und \textit{SecondRandomNumber}, die Reihe berechnet. Zuerst wird die erste Zahl auf die Start-Zahl addiert, danach die zweite und so fortführend die weiteren Zahlen. Nachdem das Muster berechnet wurde, wird die Funktion \textit{SetSolution} aufgerufen, diese berechnet die Lösung, die das Kind eintippen sollte. Dafür wird die letzte Zahl des Musters verwendet, auf welche die erste Zufallszahl addiert wird und anschließend auf dieses Ergebnis die zweite Zufallszahl addiert. Das wird so lange wiederholt, bis das Array gefüllt ist.\\
\begin{lstlisting}[language=csh, caption={hideCircle.cs SetPattern-Funktion}]
	public void SetPattern() {
		for (int i = 0; i < arrayInputs.Length; i++){
			arrayTextFields[i].text = patternNumbers[i].ToString();
		}
	}
\end{lstlisting}
Diese Funktion iteriert mit einer \textit{for}-Schleife durch das Textfeld-Array und weist jedem der Felder die passende Nummer zu. Die Integerzahl wird auch hier wieder in einen String umgewandelt.\\
Die Funktionen, um das Feuerwerk zu erzeugen und zu löschen, sind dieselben wie in dem Skript für die Rechenmauer.\\ %TODO nenne wo diese stehen
\begin{lstlisting}[language=csh, caption={hideCircle.cs Solution-Funktion}]
	IEnumerator Solution(int sec){
		bool skip = false;
		if (lvlCounter <= lvlNumber) {
			for (int i = 0; i < 5; i++) {
				if (solutionsNumbers[i] == int.Parse(arrayInputs[i].text)) {
					arrayInputs[i].interactable = false;
					continue;
				}
				else{
					skip = true;
					wrongAnswers++;
					arrayInputs[i].text = "";
				}
			}
			if (!skip) {
				SpawnFirework();
				yield return new WaitForSeconds(sec);
				DeleteFirework();
				PlayGame();
			}
		}
	}
\end{lstlisting}
Diese Funktion überprüft die Lösungen des Kindes. Dafür wird eine boolean Variable benötigt. Danach wird mit einer \textit{for}-Schleife durch fünf Schritte iteriert. Da die Arrays eine feste Größe besitzen, ist es möglich dort die Fünf fest vorzugeben. In diesem Fall kann auch die Größe des Eingabearrays oder des Lösungsarrays verwendet werden. In der Schleife werden die Eingaben des Schülers/der Schülerin mit denen aus dem Lösungsarray verglichen. Sobald eine Zahl gleich ist, wird die boolean Variable auf \textit{false} gesetzt und die Schleife springt eine Iteration weiter. Wenn eine Eingabe nicht mit der Lösung übereinstimmt, wird die boolean Variable auf \textit{true} gesetzt und der Zähler für die Fehler wird um eins erhöht. Das dazugehörige Eingabefeld wird geleert, somit sollte das Kind wissen, dass dieses Feld nicht richtig gelöst war. Wenn die Variable \textit{skip} am Ende der Schleife den Wert \textit{false} hat, sind alle Eingabefelder richtig gelöst. Daraufhin wird mithilfe der Funktion \textit{SpawnFirework} das Feuerwerk für den Nutzer/die Nutzerin erzeugt. Nach einer Sekunde wird dieses durch die Funktion \textit{DeleteFirework} wieder gelöscht. Als letzter Schritt des Skriptes wird die Funktion \textit{PlayGame} aufgerufen. \\
\begin{lstlisting}[language=csh, caption={hideCircle.cs ResetInputFields-Funktion}]
	public void ResetInputFields(){
		for (int i = 0; i < arrayInputs.Length; i++){
			arrayInputs[i].text = "";
			arrayInputs[i].interactable = true;
		}
	}
\end{lstlisting}
Diese Funktion löscht nach erfolgreichem Lösen der Reihe die Eingabe aus allen Eingabefeldern. Das wird mit einer \textit{for}-Schleife gelöst. Diese leert das Eingabefeld und aktiviert es für den Nutzer wieder.
\subsection{Szene - Buchstabensalat}
\subsubsection{Buchstabensalat - Design}
\begin{figure}[htbp]
  \centering
  \includegraphics[width=0.65\textwidth,height=0.55\textheight,keepaspectratio]{images/wordMix.PNG}
  \caption{Buchstabensalat}
  \label{wordMix}
\end{figure}
Dies ist die einzige Szene, die ein Spiel beinhaltet, um die Kompetenzen im Fach Deutsch zu verbessern. Die Szene hat einen Header, der sich nicht von den anderen unterscheidet. Dieser beinhaltet einen Menü-Button, eine Überschrift und die Levelanzeige.\\
In der Mitte gibt es ein großes Rechteck, in welchem das Wort, in dem die Buchstaben durcheinander gemischt werden, angezeigt wird. Dazwischen befindet sich das Bild eines Pfeiles\autocite{arrow}
Ganz rechts ist das Eingabefeld für den Nutzer. Unten in der Mitte befindet sich der Button, um die Lösung zu überprüfen.
\subsubsection{Buchstabensalat - Skript}
Als Erstes wird auf die Datei, die Wörter beinhaltet, eingegangen. Um eine Liste mit vielen deutschen Wörtern zu bekommen, kann eine \textit{API} verwendet werden. Da aber sicher gegangen werden sollte, dass die Wörter für Kinder in der ersten bis vierten Klasse nicht zu anspruchsvoll sind und die Kinder diese Wörter auf jeden Fall kennen, wurde von der Internetseite \textit{Grundschulkönig}.de aus dem Themenbereich Deutsch der ersten und zweiten Klasse eine Liste mit dem Grundwortschatz heruntergeladen.\autocite{Grundschulkönig}. Es wurden keine Wörter für die dritte und vierte Klasse ausgewählt, da die Datei mit den ersten beiden Klassen sehr viele Worte beinhaltet. Diese Datei kann aber einfach erweitert werden. Im folgenden Abschnitt wird ein Teilausschnitt der Datei gezeigt. 
\begin{lstlisting}[language=csh, caption={Wortdatei}]
{
	"words": [
		"Abend",
		"Abende",
		"acht",
		"alle",
		"alles",
		"alt",
		"aelter",
		"Ampel",
		"Apfel",
		"Aepfel",
		"April",
\end{lstlisting}
Die Datei ist angelegt wie eine JSON-Datei und wurde auch mit der gleichen Endung abgespeichert. Das ermöglicht dem Skript darauf zuzugreifen wie als wäre es eine Tabelle in einer JSON-Datenbank.\\
\begin{lstlisting}[language=csh, caption={WordMix.cs Variablendeklaration}]
public class WordMix : MonoBehaviour
{

	public Text wordLeft;
	public Text lvlNumber;

	public InputField solutionChild;

	private string mixedWord;
	private string solution;
	public Image imageColor;

	private int lvlCount = 0;
	private int lvlAmount = 10;
	private int wrongChoices = 0;
	private int randomNumber = 0;

	//json file
	public TextAsset textJson;
	public WordList list = new WordList();

	//button
	public Button testWord;
	public Button menu;
\end{lstlisting}
Für die Szene werden Textfelder für das Level, sowie das gemischte Wort, Buttons für das Menü und zum Testen und Variablen, die die Levelanzahl begrenzen und die Zufallszahl abspeichern benötigt. Die Level und die Fehler müssen auch gezählt werden. Das Skript muss die Lösung und das gemischte Wort abspeichern. Um mit der JSON-Datei zu arbeiten, werden zwei Objekte benötigt. Eine \textit{WordList} und ein \textit{TextAsset} auf das Objekt \textit{TextAsset} wird die JSON-Datei mit allen Wörtern zugewiesen. Dies ist zu sehen in Abbildung \ref{jsonUnity}. 
\begin{figure}[htbp]
  \centering
  \includegraphics[width=0.65\textwidth,height=0.55\textheight,keepaspectratio]{images/JsonFile.PNG}
  \caption{JSON-File in das Skript importieren}
  \label{jsonUnity}
\end{figure}
Wie dann auf dieses File zugegriffen wird, wird später erklärt.\\
\begin{lstlisting}[language=csh, caption={WordMix.cs Start-Funktion}]
	void Start()
	{
		testWord.interactable = false;
		menu.onClick.AddListener(() => GoBack());
		lvlAmount = MenuPickLevelAdvanced.lvlAmmountStatic;
		testWord.onClick.AddListener(() => ButtonClicked());
		solutionChild.onValueChanged.AddListener(delegate {EnableButton(); });
		ReadJsonFile();
		PlayGame();
	}
\end{lstlisting}
Diese Funktion deaktiviert den Test-Button und fügt die Listener hinzu. Die Levelanzahl wird abgespeichert und ein Listener wird hinzugefügt, der sich bei einer Eingabe in das Eingabefeld aktiviert und eine Funktion aufruft, die testet, ob der Test-Button aktiviert werden soll oder nicht. Danach wird die Funktion aufgerufen, die die JSON-Datei liest und die Worte daraus in eine Liste speichert. Als Letztes wird die Funktion \textit{PlayGame} aufgerufen und das Spiel wird initialisiert.\\
\begin{lstlisting}[language=csh, caption={WordMix.cs ReadJsonFile-Funktion}]
	public void ReadJsonFile(){
		list = JsonUtility.FromJson<WordList>(textJson.text);
	}
	[System.Serializable]
	public class WordList
	{
	public string[] words;
	}
\end{lstlisting}
Um Daten aus einer JSON-Datei zu lesen, wird eine extra Klasse benötigt. Diese kann serialisiert werden, aber nicht vererbt. Das wird mit dem Keyword \textit{System.Serializable} angezeigt und umgesetzt. Diese Klasse beinhaltet ein String-Array \textit{words}. Um nun auf die Daten zu kommen, wird aus dem Objekt dem die JSON-Datei zugewiesen wurde, die Wörter in die Liste \textit{list} gespeichert. Dafür muss diese jedes Wort in einen String umwandeln und auch nur die einzelnen Wörter ausgeben. Mit \textit{JsonUtility.FromJson<WordList>textJson.text);} bekommt das Skript ein Objekt der Klasse \textit{WordList}, welches ein Array aus Wörtern ist. Diese können in die Liste gespeichert werden, darum ist unwichtig, wie groß die Datei ist, die Liste wächst mit. Um auf die einzelnen Wörter aus der Datei zuzugreifen wird das Objekt, das die Datei beinhaltet, mit \textit{.text} aufgerufen. Diese Methode gibt die einzelnen Wörter aus der Datei aus. Diese werden in das Array \textit{words} gespeichert und dieses Array wird in eine Liste gespeichert. Daher hat das Skript nun eine Liste mit allen Wörtern, die in der Datei vorhanden sind.\\
Die Funktionen für den Menü-Button und um den Test-Button zu aktivieren, arbeiten gleich wie in den anderen Szenen. Der Menü-Button lädt die Szene des Menüs und setzt die Variablen. Die Funktion für den Test-Button überprüft nur das Eingabefeld und aktiviert oder deaktiviert diesen.\\
Die Funktion \textit{PlayGame} arbeitet gleich wie alle anderen. Sie zählt den Zähler hoch und lädt die Szene \textit{LearnFinishScreen} oder bereitet die Szene für die nächste Runde vor. In dieser Szene wird wieder durch einen farbigen Rand angezeigt, ob das Kind richtig lag oder nicht. Das heißt, dieser Rahmen muss wieder auf die Farbe weiß gesetzt werden. Danach wird wieder ein neues Wort aus der Liste geholt und gemischt. Als Letztes wird das Wort, wenn es gemischt wurde, in das linke Textfeld geschrieben.\\
\begin{lstlisting}[language=csh, caption={WordMix.cs GetWord-Funktion}]
	public void GetWord(){
		randomNumber = GetRandomNumbers();
		solution = list.words.GetValue(randomNumber).ToString();
		solutionChild.characterLimit = solution.Length;
		mixedWord = mixWord(solution);
	}
\end{lstlisting}
Diese Funktion erzeugt eine zufällige Zahl mit der Funktion \textit{GetRandomNumbers}, danach wird mit dieser aus der Liste an der Stelle der zufälligen Zahl ein Wort geholt. Dieses wird in die Variable für die Lösung gespeichert. Damit das Kind keinen zu langen Text in das Textfeld schreiben kann, wurde die Länge für die möglichen einzugebenden Zeichen im Eingabefeld auf die Länge des Wortes gesetzt. Als Letztes wird die Funktion \textit{mixWord(solution)} aufgerufen, diese gibt dem Skript ein Wort zurück, das gemischt ist. Als Übergabe in diese Funktion wird das aus der Liste geholte Wort übergeben.\\
\begin{lstlisting}[language=csh, caption={WordMix.cs mixWord-Funktion}]
	public string mixWord(string word){
		string finishedMixing = "";
		System.Random rnd = new System.Random();
		SortedList<int,char> list = new SortedList<int,char>();
		foreach(char c in word)
			list.Add(rnd.Next(), c);
		foreach(var x in list){
			finishedMixing += x.Value.ToString();
		}
		if(word.Equals(finishedMixing)){
			mixWord(word);
		}
		return finishedMixing;
	}
\end{lstlisting}
Diese Funktion erzeugt eine temporäre Variable \textit{finishedMixing} , welche mit einem leeren String gefüllt ist. Danach wird eine zufällige Zahl erzeugt. Dafür wird nicht die Methode von Unity verwendet, sondern die die in C\# verwendet werden kann. Dafür wird ein Objekt der Klasse \textit{System.Random} erzeugt. Danach wird eine sortierte Liste erzeugt. In diese wird das Wort gespeichert. Um das Wort zufällig in die Liste zu bekommen, wird eine \textit{foreach}-Schleife genutzt. Diese fügt an eine zufällige Zahl einen Buchstaben aus dem Wort ein. Der Aufruf \textit{rnd.Next()} gibt eine zufällige Zahl aus, die in der Reichweite eines Integers ist. Das heißt es können sehr kleine aber auch sehr große Zahlen herauskommen. Danach muss mit einer weiteren \textit{foreach}-Schleife das Wort zu einem String zusammengefügt werden. Dafür wird an jeder Stelle der Liste das Zeichen an den leeren String gehangen. Da diese Methode nicht garantiert, dass das Wort nicht zufällig gleich zusammengesetzt wird, wird in einem \textit{If}-Statement abgefragt, ob das neue Wort das gleiche wie das Wort ist, was in die Funktion übergeben wird. Wenn das Wort dasselbe ist, wird die Funktion mit dem gleichen Wort noch einmal aufgerufen. Wenn das Wort unterschiedlich ist, wird dieses zurückgegeben.\\
\begin{lstlisting}[language=csh, caption={WordMix.cs GetRandomNumbers-Funktion}]
	public int GetRandomNumbers(){
		if (list.words.Length != 0) {
			return UnityEngine.Random.Range(0, list.words.Length);
		}
		else{
			return -1;
		}
	}
\end{lstlisting}
Hier wird eine zufällige Zahl erzeugt. Die Liste startet an Index null und endet mit der Zahl, die der Länge der Liste entspricht. Also wird erst abgefragt, ob die Liste leer ist, wenn das der Fall ist, wird minus eins zurückgegeben. Wenn in der Liste aber mindestens ein Wort steht, wird mit dem Zufallszahlgenerator von Unity eine Zufallszahl zwischen null und der maximalen Länge der Liste erzeugt. Diese Funktion muss nicht angepasst werden, wenn die Liste wächst. So können immer mehr Wörter hinzugefügt werden.\\
Die Funktion, die beim Drücken des Test-Buttons aufgerufen wird, deaktiviert den Button und ruft die \textit{Coroutine}, die das Ergebnis überprüft und verarbeitet, auf.\\
\begin{lstlisting}[language=csh, caption={WordMix.cs waiter-Funktion}]
	IEnumerator waiter(int sec){
		solutionChild.interactable = false;
		if(lvlCount <= lvlAmount){
			if(solution.Equals(solutionChild.text)){
				//change color green
				imageColor.color = new Color32(37, 250, 53, 255);
				yield return new WaitForSeconds(sec);
				solutionChild.text = "";
				PlayGame();
			}
			else{
				//change color red
				imageColor.color = new Color32(251, 37, 37, 255);
				yield return new WaitForSeconds(sec);
				//change color to white
				imageColor.color = new Color32(255, 255, 255, 255);
				solutionChild.text = "";
				wrongChoices++;
			}
		}
	}
}
\end{lstlisting}
Zuerst wird das Eingabefeld deaktiviert. Danach wird, wie sonst auch, zuerst überprüft ob noch Level gespielt werden sollen. Danach wird, wenn das eingegebene Wort richtig ist, der Rahmen grün und nach einer Sekunde wird das nächste Wort geladen. Dafür muss das Feld geleert werden und die Funktion \textit{PlayGame} aufgerufen werden. Wenn das Kind falsch liegt, wird der Rahmen rot und nach einer Sekunde wieder weiß. Das Textfeld wird geleert und der Zähler für die Fehler wird nach oben gezählt.
\subsection{Szene - Nach Abschluss des Levels}
\subsubsection{Nach Abschluss des Levels - Design}
\begin{figure}[htbp]
  \centering
  \includegraphics[width=0.65\textwidth,height=0.55\textheight,keepaspectratio]{images/finishScreen.PNG}
  \caption{Endszene nach Abschluss eines Levels}
  \label{finishSzene}
\end{figure}
Diese Szene wird geladen, wenn das Kind ein Level beendet. Die Szene bekommt dort wo die Überschrift \textit{Header} steht, eine Botschaft vom Skript übergeben. Je nachdem wie viele Fehler das Kind hat, wird dort wo \textit{Glückwunschtext} steht eine andere Nachricht ausgegeben. Dort wo die drei weißen Vierecke sind werden aus einem Prefab ein Stern erzeugt. Diese kann das Kind sammeln und im Hauptmenü angezeigt bekommen. Ganz unten befindet sich ein Knopf, der in das Hauptmenü zurückführt. Somit kann das Kind dann ein neues Level auswählen. In der Abbildung \ref{withoutError} ist die Szene zu sehen, wenn das Kind keine Fehler gemacht hat.
\begin{figure}[htbp]
  \centering
  \includegraphics[width=0.65\textwidth,height=0.55\textheight,keepaspectratio]{images/finishScreen0Mistakes.PNG}
  \caption{Endszene mit null Fehlern}
  \label{withoutError}
\end{figure}
Es werden keine Sterne erzeugt, wenn der Schüler/die Schülerin mehr als vier Fehler gemacht hat. Bei genau vier oder weniger Fehlern wird der Stern auf der linken Seite erzeugt. Wenn der Nutzer/die Nutzerin genau zwei oder weniger Fehler hat, wird der Stern auf der rechten Seite erzeugt. Der Stern in der Mitte wird nur bei genau null Fehlern erzeugt.
\subsubsection{Nach Abschluss des Levels - Skript}
\begin{lstlisting}[language=csh, caption={FinishScreen.cs Variablendeklaration}]
public class FinishScreen : MonoBehaviour
{
	public Text TextfieldMistakes;
	public Text head;
	private GameObject spawnedStarLeft;
	private GameObject spawnedStarCenter;
	private GameObject spawnedStarRight;
	public GameObject LeftStarSpawn;
	public GameObject CenterStarSpawn;
	public GameObject RightStarSpawn;
	public GameObject star;
	private int mistakes;

	public Button menu;
\end{lstlisting}
Das letzte Skript benötigt zwei Textfeld-Objekte für die Überschrift und den Text, der je nach Fehleranzahl unterschiedlich ist. Um die Sterne zu erzeugen, werden drei \textit{GameObjects} benötigt. In diese werden die Sterne instantiiert. Es werden drei leere Bereiche benötigt, um den einzelnen Sternen eine Position zuzuweisen. Damit auch ein Stern erscheint, wird ein \textit{GameObjekt} mit dem Prefab des Sternes benötigt. Zuletzt braucht es eine Variable für die Fehler und ein Objekt für den Menü-Button.\\
\begin{lstlisting}[language=csh, caption={FinishScreen.cs Start-Funktion}]
	void Start()
	{
		menu.onClick.AddListener(() => GoMenu());
		mistakes = PlayerPrefs.GetInt("wrongAnswers");
		SpawnStars();
	}
\end{lstlisting}
Die Start Funktion fügt dem Menü-Button einen Listener zu. Danach holt es sich aus der Datei die Variable, die die Anzahl an Fehlern gespeichert hat. Als Letztes wird eine Funktion aufgerufen, die die Sterne erzeugen soll.\\
Die Funktion für das Menü ist schnell erklärt und wird nicht gezeigt. Diese setzt alle statischen Variablen auf null zurück und lädt die Menüszene mit dem \textit{SzeneManager}.\\
\begin{lstlisting}[language=csh, caption={SpawnStars.cs Start-Funktion}]
	public void SpawnStars(){
		head.text = ("Schade!");
		TextfieldMistakes.text = ("Naechstes Mal klappt's!");
		if(mistakes <= 4){
			SaveDataManager.RiddleSaveData.stars++;
			spawnedStarLeft = Instantiate(star, LeftStarSpawn.transform.position, Quaternion.identity);
			spawnedStarLeft.transform.SetParent(LeftStarSpawn.transform);
			head.text = ("Glueckwunsch!");
			TextfieldMistakes.text = ("Nur " + mistakes + " Fehler!");
			if(mistakes <= 2){
				SaveDataManager.RiddleSaveData.stars++;
				spawnedStarCenter = Instantiate(star, CenterStarSpawn.transform.position, Quaternion.identity);
				spawnedStarCenter.transform.SetParent(CenterStarSpawn.transform);
				head.text = ("Glueckwunsch!");
				TextfieldMistakes.text = ("Nur " + mistakes + " Fehler!");
				if(mistakes == 0){
					SaveDataManager.RiddleSaveData.stars++;
					spawnedStarRight = Instantiate(star, RightStarSpawn.transform.position, Quaternion.identity);
					spawnedStarRight.transform.SetParent(RightStarSpawn.transform);
					head.text = ("Glueckwunsch!");
					TextfieldMistakes.text = ("Super, alles richtig!!!");
				}
			}
		}
		SaveDataManager.SaveGame();
	}
\end{lstlisting}
Diese Funktion besteht aus verknüpften \textit{if}-Statement. Als Erstes wird davon ausgegangen, dass das Kind zu viele Fehler für Sterne gemacht hat. Deshalb wird die Überschrift auf 'Schade!' gesetzt. Die Botschaft darunter wird auf einen motivierenden Satz gesetzt. In dem Fall, dass zu viele Fehler gemacht wurden, wäre dieser 'Nächstes Mal klappt's'. Wenn das Kind unter vier Fehler gemacht hat, wird die Zahl der gespeicherten Sterne um eins erhöht, dafür wird die Variable aus dem Skript um die Rätselergebnisse zu speichern aufgerufen und die dort angelegte Variable für die Anzahl der Sterne um eins erhöht. Danach wird der erste Stern erzeugt. Die Texte der Überschrift werden ab dem ersten Stern immer 'Glückwunsch!' anzeigen und für den Text für die Fehler wird außer bei null Fehlern der Text 'Nur [Anzahl an Fehler] Fehler!' anzeigen. Wenn das Kind dann unter zwei Fehler gemacht hat, wird der zweite Stern erzeugt und die Variable erneut erhöht. Als letzte Abfrage wird überprüft, ob das Kind keine Fehler gemacht hat. In diesem Fall wird ein dritter Stern erzeugt und die Variable zum Zählen der Sterne um eins erhöht. Der Text unter der Überschrift der Szene zeigt nun 'Super, alles richtig!!!'. Nachdem diese Abfragen durchgelaufen sind, muss noch die Funktion aus dem Skript aufgerufen werden, die die Daten speichert. dem Befehl \textit{SaveDataManger.SaveGame()} zu erreichen.\\
%---
\chapter{Inbetriebnahme}
\label{cha:inbetriebnahme}

Aufgabe des Kapitels Inbetriebnahme ist es, die Überführung der in
Kapitel \ref{cha:implementierung} entwickelte Lösung in das betriebliche
Umfeld aufzuzeigen. Dabei wird beispielsweise die Inbetriebnahme eines
Programms beschrieben oder die Integration eines erstellten
Programmodules dargestellt.

Bei der Software-Erstellung entspricht dieses Kapitel der
Auslieferungsphase (Deployment) im \ac{rup}.

%---
\chapter{Evaluierung}
Um zu nennen, welche Ziele des Projektes erreicht wurden und welche nicht erreicht wurden, werden diese in vier Kategorien eingeteilt. Diese lauten "Must Have"\ , "\ Should Have"\, "Could Have"\ und "Nice to Have"\ eingeteilt.\\
Die Kategorie "Must Have" beinhaltete alle Ziele die auf jeden Fall erfüllt sein müssen. In diesem Projekt wären das die Lernspiele und die Rätsel. Diese sollten auch ein Menü zur auswahl der Spiele besitzen. Diese Ziel wurde erreicht, es gibt in beiden Bereichen eine Auswahl an unterschiedlichen Spielen. Die Spiele wurden durch ein Menü eingeteilt und können so bequem ausgewählt werden.\\
In der Kategorie "Should Have"\ wurde für das Projekt ein Hauptmenü vorhergesehen. Auch diese Ziele wurde erreicht. Es gibt eine Szene in dieser kann das Kind zwischen den Rätseln und den Lernspielen entscheiden. Hierbei kann das Kind sehen, wie viele Punkte es in dem Abschnitt der Rätsel bekommen hat und wie viele Sterne es durch das Spielen der Lernspiele schon gesammelt hat. Das Kind kann diese Anzahl auch mit einem Button zurücksetzten. Und das Spiel beenden.\\
Zu den "Could Have"\ gehören schwierigere Level, eine Online Rangliste und das Spiel auch für Handys zu implementieren. Davon wurden nur die schwierigeren Level umgesetzt.\\
In der letzten Kategorie den sogenannten "Nice to Have"\, gehören ein Profil, Achievements und eine Evaluation. Ein Profil und Achievements wurden nicht implementiert, aber eine sehr vereinfachtet Art der Evaluation hat statt gefunden. Das Projekt wurde von Bekannten und Freunden getestet. Dies hat dabei geholfen Fehler zu finden und Ideen zu sammeln um das Projekt zu erweitert.\\

%---
\chapter{Zusammenfassung und Ausblick}
\label{cha:zusammenfassung}

\section{Erreichte Ergebnisse}
\label{sec:ergebnisse}

Die Zusammenfassung dient dazu, die wesentlichen Ergebnisse des
Praktikums und vor allem die entwickelte Problemlösung und den
erreichten Fortschritt darzustellen. (Sie haben Ihr Ziel erreicht und
dies nachgewiesen).

\section{Ausblick}
\label{sec:ausblick}

Das gesamte Projekt könnte noch um ein Profil erweitert werden. Die Kinder könnten sich einen Namen aussuchen, ein Bild für ihr Profil auswählen. Durch das Profil könnten Errungenschaften beinhalten. Das Projekt könnte auch den Kindern ermöglichen sich mit anderen Kindern zu vergleichen, dass heißt sie könnten die Errungenschaften und den Fortschritt sehen. Die Kinder könnten sich auch als Freunde hinzufügen. Außerdem könnten Events eingebaut werden in denen die Kinder durch das Spielen von Leveln Punkte sammeln und dann in einer Rangliste aufsteigen. Es könnte Musik eingebaut werden die eine angenehme Atmosphäre bieten. Diese könnte in den Einstellungen leiser oder lauter gestellt werden. \\
Die Rätsel können durch noch mehr unterschiedliche Rätselarten erweitert werden. Außerdem könnte ein Hinweissystem eingefügt werden. Nach der Eingabe der Ergebnisse, könnte mithilfe einer Animation zu dem Sound der abgespielt wurde dem Kind zeigen ob es richtig oder falsch lag.\\
Die Lernspiele könnten mit neuen Spielen erweitert werden, außerdem könnte die Anzahl der Level nicht mehr mit der Schwierigkeit hochgehen, sonder im dem Menü mitangegeben werden. Es kann ein Menü das mithilfe von Bildern den Kinder erklärt wie sie die Spiele spielen sollen eingebaut werden. Das Menü kann um eine Auswahl der Klassen erweitert werden, so können die Spiele besser für die einzelnen Klassen gefiltert werden. Das Menü könnte dann auch zwischen den unterschiedlichen Schulfächern unterscheiden um so die Übersichtlichkeit zu verbessern.\\

\subsection{Erweiterbarkeit der Ergebnisse}
\label{sub:erweiterbarkeit}

%TODO Räsel erweiterbarkeit
Um ein Profil einzubauen, wird eine Szene dafür benötigt. Diese benötigt dann einen passenden Aufbau. Ein Textfeld um den Namen einzugeben, ein Feld um ein Bild hochzuladen. Das Profil benötigt eine Liste mit Errungenschaften. Die Errungenschaften bräuchten einen Text und ein Bild. Die Szene würde auch ein Button benötigen um die Änderungen zu speichern.
Die Rätsel wurden so Implementiert, dass ohne Probleme mehrere erzeugt werden können. Es besteht auch kein Problem ein Hinweissystem einzubauen. Dafür müsste es die Möglichkeit geben eine Währung zu sammeln, diese könnte das Kind durch das Lösen der Rätsel oder sogar durch das Lösen der Lernspiele erhalten. Mit dieser Währung könnte das Kind sich dann einen Hinweis kaufen um das Rätsel lösen zu können. Es muss nur überlegt werden wie viele Hinweise das Kind pro Rätsel kaufen darf. Eine Animation einzubauen ist auch nicht Problematisch, dafür werden zwei Sounds benötigt. Einer für das richtige Lösen und einer für das Falsche Lösen. Des weiteren wird eine Animation benötigt die dem Kind visuell zeigen soll, ob es das Rätsel richtig oder falsch gelöst hat. Diese muss nicht komplex sein, dafür würde auch eine Anreihung von Bildern reichen. \\
Die Lernspiele können um neue Spiele erweitert werden. Die Spiele die zurzeit Implementiert sind, sind für die erste bis vierte Klasse. Der Lehrplan gibt aber noch mehr Stoff her, so können sich also dafür auch neue Spiele ausgedacht werden. Die Menü-Szenen einzubauen ist auch kein Problem, dafür muss nur ein Design überlegt werden, dass zum restlichen Projekt passt. Eine Anleitung der Spiele, kann in eines der Menüs eingebaut werden. Am besten in das Menü das dann die Spiele auch anbietet. Wenn ein Menü eingebaut wird, welches zum Beispiel zwischen den Spielen für das Fach Deutsch und Mathe unterscheidet, dann sollte sich die Anleitung im jeweiligen Menü befinden, in dem auch die Spiele vorkommen.\\

%-----------------------------------------------------------------------
\appendix

%---
\printbibliography[heading=bibintoc]



\end{document}